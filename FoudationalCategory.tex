%\documentclass[aps,twocolumn,secnumarabic,graphics,flotfix,graphicx,
%url,bm,tightenlines,nobibnotes,nobalancelastpage,amsmath,amssymb,
%nofootinbib]{article}
\documentclass[aps,twocolumn,secnumarabic,nobalancelastpage,amsmath,amssymb,
amsthm,nofootinbib,parskip=full]{revtex4}

% standard graphics specifications
% alternative graphics specifications
% helps with long table options
% for on-line citations
% special 'bold-math' package

\usepackage{utf8}{inputenc}
%\usepackage[bottom]{footmisc}
\usepackage{titlesec}
\usepackage{mathtools}
\usepackage{graphics}      % standard graphics specifications
\usepackage{graphicx}      % alternative graphics specifications
\usepackage{longtable}     % helps with long table options
\usepackage{url}           % for on-line citations
\usepackage{bm}            % special 'bold-math' package
%\usepackage{unicode-math}
\usepackage{stmaryrd}
\usepackage{mathrsfs}
%\usepackage{mathabx}
%\setmathfont{XITS Math}
%\setmathfont[version=setB,StylisticSet=1]{XITS Math}
\usepackage[utf8]{inputenc}
\usepackage[english]{babel}
\usepackage{pgf, tikz, tikz-cd}
\usepackage{chngcntr}
\counterwithin{figure}{section}
\usepackage{calrsfs}
\usepackage{stringstrings}
\usepackage{xstring}

\DeclareMathAlphabet{\pazocal}{OMS}{zplm}{m}{n}
%\DeclareMathAlphabet{\eurocal}{OMS}{zplm}{m}{n}
\usepackage{calligra}
\usepackage[T1]{fontenc}
\DeclareFontShape{T1}{calligra}{m}{n}{<->s*[1.2]callig15}{}
\DeclareMathAlphabet{\classscr}{T1}{calligra}{m}{n}
\usepackage[mathscr]{euscript}
\let\euscr\mathscr \let\mathscr\relax% just so we can load this and rsfs
\usepackage[scr]{rsfso}
\usepackage{relsize}

\usetikzlibrary{matrix,arrows, decorations,
  positioning, automata, calc, fit, shapes.geometric}
\newtheorem{theorem}{Theorem}[section]
\newtheorem{proposition}{Proposition}[section]
\newtheorem{corollary}{Corollary}[theorem]
\newtheorem{lemma}[theorem]{Lemma}
\newtheorem{proof}{Proof}[theorem]
\newtheorem{axiom}{Axiom}

\numberwithin{equation}{section}

%%\theoremstype{definition}
\newtheorem{definition}{Definition}[section]
%\newtheorem*{remark}{Remark}

%\setlength{\parskip}{\baselineskip}
%\setlength{\parskip}{4em}

\tikzset{every loop/.style={min distance=12mm,looseness=10}}
\tikzset{place/.style={circle,thick,minimum size=8mm}}
\tikzset{
modal/.style={>=stealth’,shorten >=1pt,shorten <=1pt,auto,node distance=1.5cm,
semithick},
world/.style={circle,draw,minimum size=0.5cm,fill=gray!15},
point/.style={circle,draw,inner sep=0.5mm,fill=black},
reflexive above/.style={->,loop,looseness=7,in=120,out=60},
reflexive below/.style={->,loop,looseness=7,in=240,out=300},
reflexive left/.style={->,loop,looseness=7,in=150,out=210},
reflexive right/.style={->,loop,looseness=7,in=30,out=330}
}

\newcommand*{\xslant}[2][70]{%
  \begingroup
    \sbox0{#2}%
    \pgfmathsetlengthmacro\wdslant{\the\wd0 + cos(#1)*\the\wd0}%
    \leavevmode
    \hbox to \wdslant{\hss
      \tikz[
        baseline=(X.base),
        inner sep=0pt,
        transform canvas={xslant=cos(#1)},
      ] \node (X) {\usebox0};%
      \hss
      \vrule width 0pt height\ht0 depth\dp0 %
    }%
  \endgroup
}

\makeatletter
\newcommand*{\xslantmath}{}
\def\xslantmath#1#{%
  \@xslantmath{#1}%
}
\newcommand*{\@xslantmath}[2]{%
  % #1: optional argument for \xslant including brackets
  % #2: math symbol
  \ensuremath{%
    \mathpalette{\@@xslantmath{#1}}{#2}%
  }%
}
\newcommand*{\@@xslantmath}[3]{%
  % #1: optional argument for \xslant including brackets
  % #2: math style
  % #3: math symbol
  \xslant#1{$#2#3\m@th$}%
}
\makeatother

\newcommand{\nlessdot}{\mathrlap{\lessdot}\;|\;\,}

\newcommand{\gtrplus}{\mathrlap
  {\textbf{\raisebox{1.15pt}{\hspace{2.6pt}\scalebox{.4}{+}}}}>}
\newcommand{\padtxt}[1]{\hspace*{1em}\parsecat{#1}\hspace*{1em}}

\newcommand{\abk}[1]{\xslantmath{\mathfrak{#1}}}
%\newcommand{\subp}[1]{\xslantmath{\mathscr{#1}}}
\newcommand{\subp}[1]{\mathbf{#1}}
\newcommand{\obk}[1]{\mathpzc{#1}}
\newcommand{\diag}[1]{\mathbf{\mathsf{#1}}}
\newcommand{\diagobj}[1]{\diag{#1}_0}
\newcommand{\diagarr}[1]{\diag{#1}_1}
\newcommand{\diagdom}[1]{\oper{dom}\diag{#1}}
\newcommand{\diagcod}[1]{\oper{cod}\diag{#1}}
%\newcommand{\cat}[1]{\pazocal{#1}}
\newcommand{\cat}[1]{\pazocal{#1}}
\newcommand{\iarr}[1]{{\large \abk{i}}_{\obk{#1}}}
\newcommand{\tarr}[1]{{\large \abk{t}}_{\obk{#1}}}
\newcommand{\earr}[1]{{\large \abk{e}}_{\obk{#1}}}
\newcommand{\earrid}[1]{\oper{id}_{\obk{#1}}}
\newcommand{\earrdom}[1]{\oper{dom}\earr{#1}}
\newcommand{\earrcod}[1]{\oper{cod}\earr{#1}}
\newcommand{\arr}[1]{{\large \abk{a}}_{\obk{#1}}}
\newcommand{\arrid}[1]{\oper{id}_{\obk{#1}}}
\newcommand{\arrdom}[1]{\oper{dom}\arr{#1}}
\newcommand{\arrcod}[1]{\oper{cod}\arr{#1}}
\newcommand{\arrrel}[1]{\oper{rel}\arr{#1}}
\newcommand{\oarr}[2]{{\large \abk{a}}_{{\subp{#1},\obk{#2}}}}
\newcommand{\oarrid}[2]{\oper{id}_{\subp{#1},\obk{#2}}}
\newcommand{\oarrdom}[2]{\oper{dom}\oarr{#1}{#2}}
\newcommand{\oarrcod}[2]{\oper{cod}\oarr{#1}{#2}}
\newcommand{\oarrrel}[2]{\oper{rel}\oarr{#1}{#2}}
\newcommand{\arrn}[2]{{\large \abk{a}}^{#2}_{\obk{#1}}}
\newcommand{\ccid}[1]{\cat{#1}_{\oper{id}}}
\newcommand{\ccobj}[1]{\cat{#1}_{\subp{0}}}
\newcommand{\ccarr}[1]{\cat{#1}_{\subp{1}}}
\newcommand{\occarr}[2]{\cat{#2}_{\subp{#1}}}
\newcommand{\ocatarr}[2]{\parsecat{#2}_{\subp{#1}}}
\newcommand{\catarr}[1]{\parsecat{#1}_{\subp{1}}}
\newcommand{\catobj}[1]{\parsecat{#1}_{\subp{0}}}
\newcommand{\owraparr}[2]{(#2)_{\subp{#1}}}
\newcommand{\wrpcobj}[1]{#1_{,\subp{0}}}
\newcommand{\wrpcarr}[1]{#1_{,\subp{1}}}
\newcommand{\owrparr}[2]{{#2}_{\subp{#1}}}
\newcommand{\wrparr}[1]{#1_{\subp{1}}}
\newcommand{\wrpobj}[1]{#1_{\subp{0}}}
\newcommand{\wraparr}[1]{(#1)_{\subp{1}}}
\newcommand{\wrapobj}[1]{(#1)_{\subp{0}}}
\newcommand{\catfull}[1]{\cat{#1}(\ccobj{#1},\ccarr{#1})}
\newcommand{\oper}[1]{\mathbf{#1}\,}
\newcommand{\catN}[1]{\mathbf{\large #1}}
\newcommand{\initcat}{\catN{0}}
\newcommand{\termcat}{\catN{1}}
\newcommand{\singleton}[1]{\catN{1}_{#1}}
\newcommand{\singletoncat}[1]{\singleton{\parsecat{#1}}}
\newcommand{\fst}[1]{\oper{fst}\,#1}
\newcommand{\snd}[1]{\oper{snd}\,#1}
\newcommand{\adj}[2]{#1\dashv #2}
\newcommand{\adjcat}[2]{\pazocal{#1}\dashv\pazocal{#2}}
\newcommand{\adjfunct}[2]{\funct{#1}\dashv\funct{#2}}
\newcommand{\adjnat}[2]{\nfunct{#1}\dashv\nfunct{#2}}
\newcommand{\adjcom}[2]
{\funct{#2}\circ\funct{#1}\dashv\funct{#1}\circ\funct{#2}}
\newcommand{\adjtpl}[3]{(\parsecat{#1},\,\funct{#2},\,\nfunct{#3})}
\newcommand{\adjdbl}[2]{(\funct{#2},\,\nfunct{#3})}
\newcommand{\montpl}[3]{\mathsf{T}(\funct{#1},\,\nfunct{#2},\,\nfunct{#3})}

\newcommand{\efarr}[3]{\oarr{#1}{\epair{#2}{#3}}}
\newcommand{\ofarr}[3]{\oarr{#1}{\ffpair{#2}{#3}}}
\newcommand{\ofarrid}[3]{\langle\arrid{{#1}^*},\,\arrid{#2}\rangle}
\newcommand{\ofarrdom}[3]{\oper{dom}\ofarr{#1}{#2}{#3}}
\newcommand{\ofarrcod}[3]{\oper{cod}\ofarr{#1}{#2}{#3}}
\newcommand{\farr}[2]{\arr{\fpair{#1}{#2}}}
\newcommand{\farrid}[2]{\langle\arrid{#1},\,\arrid{#2}\rangle}
\newcommand{\farrdom}[2]{\oper{dom}\farr{#1}{#2}}
\newcommand{\farrcod}[2]{\oper{cod}\farr{#1}{#2}}
\newcommand{\ffarrdom}[2]{\oper{dom}_{\obk{1}}\farr{#1}{#2}}
\newcommand{\ffarrcod}[2]{\oper{cod}_{\obk{1}}\farr{#1}{#2}}
\newcommand{\arrpidom}[1]{\nprojdown{1}{\oper{dom}}\arr{#1}}
\newcommand{\arrpicod}[1]{\nprojdown{1}{\oper{cod}}\arr{#1}}
\newcommand{\arrpidomn}[2]{\nprojdown{#1}{\oper{dom}}\arr{#2}}
\newcommand{\farrpicodn}[3]{\nprojdown{#1}{\oper{cod}}\oarr{#2}{#3}}
\newcommand{\farrpidomn}[3]{\nprojdown{#1}{\oper{dom}}\oarr{#2}{#3}}
\newcommand{\arrpicodn}[2]{\nprojdown{#1}{\oper{cod}}\arr{#2}}
\newcommand{\farrpidomn}[2]{\arrpidom{\oarr{#1}{#2}}}
\newcommand{\farrpicodn}[2]{\arrpicod{\oarr{#1}{#2}}}
\newcommand{\farrpidom}[2]{\arrpidom{\fpair{#1}{#2}}}
\newcommand{\farrpicod}[2]{\arrpicod{\fpair{#1}{#2}}}
\newcommand{\arrpifst}[2]{\nprojfst{1}{\farr{#1}{#2}}}
\newcommand{\arrpisnd}[2]{\nprojsnd{1}{\farr{#1}{#2}}}
\newcommand{\oarrpiop}[3]{\nprojdown{#1}{\oper{#2}}\,#3}
\newcommand{\arrpiop}[2]{\oarrpiop{1}{#1}{#2}}

\newcommand{\largenat}{\mathpzc{NAT}}
\newcommand{\wrpcat}[1]{#1_{\functscr{O}}}
\newcommand{\wrpfunct}[1]{#1_{\functscr{F}}}
\newcommand{\wrpnat}[1]{#1_{\functscr{N}}}
\newcommand{\largenatcat}{\wrpcat{\largenat}}
\newcommand{\largenatfunct}{\wrpfunct{\largenat}}
\newcommand{\largenatnat}{\wrpnat{\largenat}}
\newcommand{\cnat}[2]{\largenat\cpr{#1}{#2}}
\newcommand{\cfnat}[2]{\largenat\fpair{#1}{#2}}
\newcommand{\natfunct}[2]{\wrpfunct{\cnat{#1}{#2}}}
\newcommand{\natnat}[2]{\wrpnat{\cnat{#1}{#2}}}
\newcommand{\nfunctscr}{\functscr{N}}
\newcommand{\nfunct}[1]{\nfunctscr_{\obk{#1}}}
\newcommand{\nfunctdom}[1]{\oper{dom}\,\nfunct{\obk{#1}}}
\newcommand{\nfunctcod}[1]{\oper{cod}\,\nfunct{\obk{#1}}}
\newcommand{\nfunctobj}[1]{\nfunctscr_{obk{#1},\subp{2}}}
\newcommand{\nfunctarr}[1]{\nfunctscr_{\obk{#1},\subp{3}}}

\newcommand{\cone}[2]{\mathpzc{CONE}\cpr{#1}{#2}}
\newcommand{\conefunct}[2]{{\mathpzc{CONE}\wrpfunct{\cpr{#1}{#2}}}}
\newcommand{\conenat}[2]{{\mathpzc{CONE}\wrpnat{\cpr{#1}{#2}}}}
\newcommand{\cocone}[2]{\mathpzc{COCO}\cpr{#1}{#2}}
\newcommand{\coconefunct}[2]{{\mathpzc{COCO}\wrpfunct{\cpr{#1}{#2}}}}
\newcommand{\coconenat}[2]{{\mathpzc{COCO}\wrpnat{\cpr{#1}{#2}}}}

\newcommand{\functbag}[1]{\llbracket #1\rrbracket}

%{{\raisebox{.1\baselineskip}{\ensuremath{\digamma}}}}
%{\mathrlap{\lessdot}\;|\;\,}
\newcommand{\oprojection}[2]{\xslantmath{\mathsf{P}}_
  {\!\!\overset{#2}{#1}}}
\newcommand{\nprojection}[1]{\xslantmath{\mathsf{P}}_
  {\!\!\shortdownarrow\!\raisebox{.3\baselineskip}{\scalebox{.5}{#1}}}}
\newcommand{\nprojdown}[2]{\nprojection{#1}_{,\obk{#2}}}
\newcommand{\nprojfst}[2]{\oprojection{\shortleftarrow}{#1}\,#2}
\newcommand{\nprojsnd}[2]{\oprojection{\shortrightarrow}{#1}\,#2}
\newcommand{\simplex}[1]{\scalebox{1}[1.4]{\ensuremath{\rhd}}_{\catN{#1}}}
\newcommand{\simplexfunct}[1]{\wrpfunct{\simplex{#1}_{,}}}
\newcommand{\simplexcat}[1]{\wrpcat{\simplex{#1}_{,}}}
\newcommand{\simplexcat}[1]{\wrpcat{\simplex{#1}_{,}}}
\newcommand{\simplexnn}{\scalebox{1}[1.4]{\ensuremath{\rhd}}_{\mathbb{N}}}

\newcommand{\snat}{\mathpzc{Nat}}
\newcommand{\smallnat}[2]{\snat\cpr{#1}{#2}}
\newcommand{\smallnatobj}{\wrpcat{\smallnat}}
\newcommand{\smallnatnat}{\wrpnat{\smallnat}}
\newcommand{\smallnatfunct}{\wrpfunct{\smallnat}}
\newcommand{\snatfunct}[2]{\wrpfunct{\smallnat{#1}{#2}}}
\newcommand{\snatnat}[2]{\wrpnat{\smallnat{#1}{#2}}}
\newcommand{\snfunct}[1]{\nfunctscr_{\obk{#1}}}
\newcommand{\snfunctdom}[1]{\oper{dom}\,\nfunctscr_{\obk{#1}}}
\newcommand{\snfunctcod}[1]{\oper{cod}\,\nfunctscr_{\obk{#1}}}
\newcommand{\snfunctobj}[1]{\nfunctscr_{\obk{#1},\subp{2}}}
\newcommand{\snfunctarr}[1]{\nfunctscr_{\obk{#1},\subp{3}}}

\newcommand{\homfunct}[2]{\left[\parsecat{#1},\,\parsecat{#2}\right]}
\newcommand{\chom}[1]{\left|\carr{#1}\right|}
\newcommand{\homobj}[1]{\wrpobj{\chom{#1}}}
\newcommand{\homarr}[1]{\wrparr{\chom{#1}}}
\newcommand{\projg}[3]{\xslantmath{\mathbf{#1}}^{#3}(\mathsf{#2})}
\newcommand{\ccpairn}[2]{\nprojection{#1}(\parsecat{#2})}
\newcommand{\ccpair}[1]{\nprojection{1}(\parsecat{#1})}
\newcommand{\ccpairg}[1]{\nprojection{1}(\mathsf{#1})}
\newcommand{\apair}[2]{\langle\arr{#1},\,\arr{#2}\rangle}
\newcommand{\epair}[2]{\langle|\obk{#1},\,\obk{#2}|\rangle}
\newcommand{\ffpair}[2]{\langle\!\langle\obk{#1},\,\obk{#2}\rangle\!\rangle}
\newcommand{\fpair}[2]{\langle\obk{#1},\,\obk{#2}\rangle}
\newcommand{\fopair}[2]{\langle #1,\,#2\rangle}
\newcommand{\pair}[2]{(#1,\,#2)}
\newcommand{\cpair}[2]{(\obk{#1},\,\obk{#2})}
\newcommand{\cpr}[2]{(\pazocal{#1},\,\pazocal{#2})}
\newcommand{\ctpair}[4]{(\pair{#1}{#2},\,\pair{#3}{#4})}
\newcommand{\ccrel}[2]{\parsecat{#1}_{\obk{#2}}}
\newcommand{\csub}[2]{#1\lessdot #2}
\newcommand{\carrn}[2]{\parsecat{#1}_{\overset{#2}{\rightarrow}}}
\newcommand{\carrnobj}[2]{\wrapobj{\carrn{#1}{#2}}}
\newcommand{\carrnarr}[2]{\wraparr{\carrn{#1}{#2}}}
\newcommand{\carr}[1]{\parsecat{#1}_{\rightarrow}}
\newcommand{\carrobj}[1]{\carr{#1}_{\subp{,\,0}}}
\newcommand{\carrarr}[1]{\carr{#1}_{\subp{,\,1}}}
\newcommand{\functscr}[1]{\small\xslantmath{\pazocal{#1}}}

\newcommand{\bfunct}[1]{\functscr{F}_{\parsecat{#1}}} % behavior
\newcommand{\bfunctobj}[1]{\functscr{F}_{\parsecat{#1},\subp{0}}} % behavior
\newcommand{\bfunctarr}[1]{\functscr{F}_{\parsecat{#1},\subp{1}}} % behavior

\newcommand{\funct}[1]{\functscr{F}_{\obk{#1}}}
\newcommand{\functobj}[1]{\funct{#1}_{,\subp{1}}}
\newcommand{\functarr}[1]{\funct{#1}_{,\subp{2}}}
\newcommand{\functV}[2]{\funct{#1}(#2)}
\newcommand{\functid}[1]{\oper{id}_{\parsecat{#1}}}
\newcommand{\functdom}[1]{\oper{dom}\funct{#1}}
\newcommand{\functcod}[1]{\oper{cod}\funct{#1}}

\newcommand{\ofunct}[1]{\functscr{F}_{#1}}
\newcommand{\ofunctobj}[1]{\functscr{F}_{#1,\subp{1}}}
\newcommand{\ofunctarr}[1]{\functscr{F}_{#1,\subp{2}}}
\newcommand{\ofunctV}[2]{\ofunct{#1}(#2)}
\newcommand{\ofunctid}[1]{\oper{id}_{#1}}
\newcommand{\ofunctdom}[1]{\oper{dom}\ofunct{#1}}
\newcommand{\ofunctcod}[1]{\oper{cod}\ofunct{#1}}

\newcommand{\umpclass}{\mathcal{F}}
\newcommand{\umpclassp}[1]{{\umpclass}_{\parsecat{#1}}}
\newcommand{\umpclasspc}[2]{\umpclassp{#1}(\parsecat{#2})}
\newcommand{\umpall}{\umpclassp{\forall}}
\newcommand{\umpallc}[1]{\umpall(\parsecat{#1})}
\newcommand{\umpconst}[1]{\umpall_{\parsecat{#1}}}
\newcommand{\umpconstc}[2]{\umpconst{#1}(\parsecat{#2})}
\newcommand{\umpcat}{\mathcal{C}}
\newcommand{\umpcatp}[1]{\umpcat_{\parsecat{#1}}}
\newcommand{\umpdomp}[1]{\oper{dom}\,\umpclassp{#1}}
\newcommand{\umpcodp}[1]{\oper{cod}\,\umpclassp{#1}}
\newcommand{\expab}[2]{\obk{#1}^{\obk{#2}}}
\newcommand{\expxab}[2]{\expab{#2}{#1}\times\obk{#1}}
\newcommand{\mapfab}[3]{{#1}:\,{#2}\longrightarrow {#3}}
\newcommand{\mapaob}[3]{{#1}\xrightarrow{\;{#2}\;}{#3}}

\newcommand{\homcat}{\parsecat{Hom}}
\newcommand{\quivercat}[1]{\parsecat{Quiv}(\parsecat{#1})}
\newcommand{\largepo}{\mathbf{\mathpzc{PO}}}
\newcommand{\largeeq}{\mathbf{\mathpzc{EQUIV}}}
\newcommand{\smalleq}{\mathbf{\mathpzc{Equiv}}}
\newcommand{\poset}{\mathbf{\mathpzc{Poset}}}
\newcommand{\largegrf}{\mathbf{\mathpzc{GRF}}}
\newcommand{\smallgrf}{\mathbf{\mathpzc{Grf}}}
\newcommand{\functcat}{\mathbf{\mathpzc{FUN}}}
\newcommand{\functcatcat}{\mathbf{\mathpzc{FUN}}_{\functscr{C}}}
\newcommand{\functcatfunct}{\mathbf{\mathpzc{FUN}}_{\functscr{F}}}
\newcommand{\smallfunctcat}{\mathbf{\mathpzc{Fun}}}
\newcommand{\smallfunctcatcat}{\smallfunctcat_{\functscr{C}}}
\newcommand{\smallfunctcatfunct}{\smallfunctcat_{\functscr{F}}}
\newcommand{\functcatp}[1]{\functcat_{\parsecat{#1}}}
\newcommand{\functcatcatp}[1]{\functcat_{\parsecat{#1},\functscr{C}}}
\newcommand{\functcatfunctp}[1]{\functcat_{\parsecat{#1},\functscr{F}}}
\newcommand{\wrpcat}[1]{{#1}_{\functscr{C}}}
\newcommand{\wrpfunct}[1]{{#1}_{\functscr{F}}}
\newcommand{\wrpnat}[1]{{#1}_{\functscr{N}}}
\newcommand{\largecat}{\mathbf{\mathpzc{CAT}}}
\newcommand{\largecatobj}{\wrpobj{\largecat}}
\newcommand{\largecatarr}{\wrparr{\largecat}}
\newcommand{\smallcat}{\parsecat{Cat}}
\newcommand{\smallcatobj}{\wrpobj{\smallcat}}
\newcommand{\smallcatarr}{\wrparr{\smallcat}}
\newcommand{\expcat}{\mathpzc{Exp}}
\newcommand{\setcat}{\mathpzc{Set}}
\newcommand{\smallsetcat}{\parsecat{set}}
\newcommand{\fincat}{\parsecat{Fin}}
\newcommand{\forgetful}[1]{|#1|}
\newcommand{\forgetgrf}[1]{\functscr{U}(#1)}
\newcommand{\rffunct}{\functscr{U}^{\Circlearrowright}}
\newcommand{\rforgetgrf}[1]{\rffunct(#1)}
\newcommand{\ffunct}{\funct{U}}
\newcommand{\subcat}[1]{\mathpzc{Sub}\parsecat{#1}}

\newcommand{\rquiver}[1]{\parsecat{Q}^{\Circlearrowright}(\mathsf{#1})}
\newcommand{\wquiver}[1]{\parsecat{Q}(\mathsf{#1})}
\newcommand{\nwquiver}[2]{\parsecat{Q}^{#1}(\mathsf{#2})}
\newcommand{\quiver}[1]{\Gamma(\mathsf{#1})}
\newcommand{\graph}[1]{\mathsf{#1}}
\newcommand{\graphv}[1]{\mathsf{#1}_{\mathbf{v}}}
\newcommand{\graphe}[1]{\mathsf{#1}_{\mathbf{e}}}
\newcommand{\graphfree}[1]{\parsecat{C}(\graph{#1})}
\newcommand{\graphclass}[1]{{\mathcal{F}}\,(\graph{#1})}

\newcommand{\evalxab}[2]{\mapaob{\expxab{#1}{#2}}{\oper{eval}}{\obk{#2}}}

\newcommand{\di}[1]{\boldsymbol #1}
\newcommand{\dI}{\boldsymbol \imath}

\newcommand{\freefunct}{{\raisebox{.1\baselineskip}{\ensuremath{\digamma}}}}
\newcommand{\freeset}[1]{\freefunct(#1)}
\newcommand{\pset}[1]{{\raisebox{.15\baselineskip}{\Large\ensuremath{\wp}}}(#1)}
\newcommand{\scard}[1]{\kappa(#1)}

\newcommand{\numchars}[1]{\noindent The string #1 has \StrLen{#1} characters. }

%\newcommand{\hmult}{\scalebox{.4}{\bigtriangledown}}
\newcommand{\cohopfmult}{\scalebox{.6}[1]{\ensuremath{\Delta}}}
\newcommand{\hopfmult}{\raisebox{\depth}{\scalebox{.6}[-1]{\ensuremath{\Delta}}}}

\makeatletter
\def\instring#1#2{TT\fi\begingroup
  \edef\x{\endgroup\noexpand\in@{#1}{#2}}\x\ifin@}
%
\def\isuppercase#1{%
  \ensuremath{%
  \instring{#1}{AÂBCÇDEFGĞHIİÎJKLMNOÖÔPRSŞTUÜÛVYZ}%
  }
}%
\makeatother

\newcommand{\checkcat}[1]{
\StrBefore[1]{#1}{/}[\Topcat]
\StrBehind[1]{#1}{/}[\Botcat]
\StrLen{\Topcat}[\toplen]
\StrLen{\Botcat}[\botlen]
\IfEq{\toplen}{0}{\mathbf{\mathpzc{#1}}}{(\parseonecat{\Topcat}/\parseonecat{\Botcat})}
}

\newcommand{\UpperCats}{ABCDEFGHIJKLMNOPQRSTUVWXYZ}

\newcommand{\parsecat}[1]{%
    \StrLen{#1}[\slen]
    \ifcase\slen
      #1
    \or\parseonecat{#1}
    \else
      \checkcat{#1}
    \fi
}

\newcommand{\parseonecat}[1]{
  \IfInteger{#1}{\catN{#1}}
    { \IfSubStr{\UpperCats}{#1}{\pazocal{#1}}{\obk{#1}} }
}

%\newcommand{\catcaseset}[1]{
%\if\isuppercase{#1}\cat{#1}\else\obk{#1}\fi
%}

\def\CircleArrowleft{\ensuremath{%
  \reflectbox{\rotatebox[origin=c]{180}{$\circlearrowleft$}}}}
\def\CircleArrowright{\ensuremath{%
  \reflectbox{\rotatebox[origin=c]{180}{$\circlearrowright$}}}}
\def\Circlearrowleft{\ensuremath{%
  \rotatebox[origin=c]{100}{$\circlearrowleft$}}}
\def\Circlearrowright{\ensuremath{%
  \rotatebox[origin=c]{150}{$\circlearrowright$}}}

\setcounter{secnumdepth}{4}

\titleformat{\paragraph}{\normalfont\normalsize\bfseries}{\theparagraph}{1em}{}
\titlespacing*{\paragraph}{0pt}{3.25ex plus 1ex minus .2ex}{1.5ex plus .2ex}

\tikzset{
commutative diagrams/.cd,
%row sep=3cm,
%column sep=3cm,
%matrix scale/.style={/tikz/matrix xscale=3,/tikz/matrix yscale=3},
%matrix scale=2,
arrow style=tikz,
diagrams={>=latex}
}
\tikzset{res/.style={ellipse,draw,minimum height=0.1cm,minimum width=0.1cm}}

\begin{document}
\title{Algebra in Category Theory}
\author         {Wolfgang Kraske}
\email        {wolfkden@gmail.com}
\homepage     {http://www.oviumzone.com}
\affiliation  {OVium Studies in Physics}
\date{\today}

\begin{abstract}
  Category theory establishes a framework for functional programming,
  serving as a methodology for the reliable development of effective algorithms.
  Category theory aligned with algebraic topology conceits the practical
  reification of general mathematical concepts and artificial intelligence
  algorithms. A repertoire for algorithm development progresses from
  contemporary functional to eventual quantum entanglment.
  Category theory provides an essential framemork for use case implementation
  of critical or casual activities in contemporary civilization.
  As practical consequence, efficient lambda and combinatorial algorithms,
  originally conceptualized by Alonzo Church and Haskell Curry, respectively,
  now empower popular and traditional application;
  derived from the concepts of algebra and category theory.
  Category theory provides essential models and 
  programming idioms established from fixed point theory, Lambek's Theorem,
  to inspire algebraically structured algorithm implementations.
  A confluence of emergent models and idiomatic concepts flourish.
  The fixed point affect upon the paradoxes of game theory and
  quantum phenomenon surrender to propitious applications. 
  The structural caveat of Cartesian closed category 
  dictates the design and implementation of programming languages.
  The expression of Yoneda's Lemma in category theory
  facilitates new algorithmic
  reification of problems in physics and algebraic systems.
  Computer analysis increasingly embraces abstract topics in topology,
  geometry and number theory.
  Topos theory now underpins algorithms for intuitive logic and topology;
  eliminating the need for rigidly inefficient and maladroit
  alternatives by outdated set theoretical approaches.
\end{abstract}

\maketitle

\setlength{\parindent}{0em}
\setlength{\parskip}{0.5em}
\renewcommand{\baselinestretch}{1.0}

\section{Introduction}
Category theory emerged as a schematic representation of
algebraic topology in the late 1940's, quickly progressing
to other fields of mathematics therafter.
In topology category theory resigned the tedium of point set analysis to
cleaner limit/cone-colimit/cocone proposition and proof.
The introduction of topos theory by Grothendieck,
\cite{grothendieck1957}, complemented Mac Lane's interpretation of
category theory, extending a program of cohesive analysis to
encompase topology as well as intuitionistic logic.
In the backdrop of Mac Lane's 1947 introduction
of category theory, Kurt G\"{o}del's incompleteness theorem overturned
Hilbrert's second problem; abolishing the conjecture of an absolute
logical foundation to mathematics.
The existence of quantum spin further cut the continuous manifold
hypothesis of the phase space of mathematical physics.
Both concepts assimilate jointly into category theory.
G\"{o}del theorem and quantum spin both share the synoptic
reality of fixed point phenomenon in logic and topology.
In summary, Lawvere, \cite{lawvere1997}, established a common
fixed point idiom within category theory to systematically model set theory 
paradoxes; from Russell, G\"{o}del incompleteness, Cantor cardinalities,
Turing halting, Tarski truth and periphrially the conundrum of quantum spin.
The progenitors, Eilenberg and Mac Lane, developed a categorical proposition to
distill and analyze abstract concepts from the once independent fields
of topology and algebra. The capacity of category theory
to distill concepts and methodology from many fields 
elevated a discipline affecting new fields such as formal language.
The eventuality of these accomplishments propelled
category theory into an independent field of pure mathematics.
Coincidently the field of computer science developed,
underscored by the achievements of Church, Turing and Von Neumann,
as special topic in algebraic topology and as a practical
application for electronic calculating machines.
The ability of category theory to embrace concepts
such as lambda calculus and algebra portend a propitious
future and ongoing fulfillment in computer science and mathematics.
The more recent development of homotopy type theory fully embraces
category theory to combine aspects of algebraic topology, homology and
lambda calculus.

\section{Basic Theory}\label{sec:basictheory}

The predecessor of category theory, set theory,
pursues the deriviation of mathematics with an ardent belief in logic. 
The belief in logic as an inviolable substance pervades the pursuit of
mathematics and even human justice since at least the time of Plato.
At the dawn of the 20\textsuperscript{th} century 
pure logic consumed methods of syntactical proposition and semantic proof
to unify the language of mathematics as derived from ordinal and cardinal sets. 
Extensions of set theory with the lambda calculus of Alonzo Church
and the combinatory logic of Haskell Curry extended the breadth of
tools available although limited by the Kleen-Rosser paradox.
A succession of paradoxes confounded the logical closure of set theory.
The logical contradictions established by Kurt G\"{o}del's completeness theorems
eventually established limits to logical development of mathematics and
a diagnosis of the paradoxes of Cantor, Russell, Tarski etc.
Lawvere later equivalated the paradoxes of set theory as a
common fixed point idiom.
Despite the futility of set theory to overcome the boundaries of logic
capable disaplines developed to navigate paradoxical limits while
extending the foundations for many significant mathematical fields.

Category theory emerged as the bounding firmament of set theory coalesced.
Relative to the ancient studies of logical arithmetic and geometry,
set theory approached mathematics with the logic of the former while category
theory convenes the later with the concept of arrow.
Category theory formalized a schematic representation for the
emergence of algebraic topology and the eventual development of
homological type theory, in complement to the tools of Alozo Church and
Haskell Curry.

The structural development of 
Category theory complements the quantitative 
tools of set theory with an emphasis on meta abstraction and
conceptual framework.
The proposition of an arrow mediates the abstraction and framework of
category theory; a metaphore to the edge of a graph. In this sense
an arrow is an entity that explicitely or implcitely assumes a relation,
to a meta environment within an encompasing category.
Teleological conundra and complexities are implicitly partitioned and
abstracted from areas of specific interest for construction and analysis.

The propositional development of category theory progresses
through iterations, each iteration designates a context.
The method of context iteration navigates logical paradoxes 
to develop a robust and practical theory.
Collection and arrow propositons engender the development of category theory.
Incremental collection and arrow propositions develop through a
succession of contexts.
From the initial context, the arrow introduces the application of algorithms
to develop proof of proposition, type structure and transformation.
Likewise context levels extend the foundational collection model
to higher order transformation behavior, i.e.
categories, functors and natural transforms.
The context approach synoptically develops logic, language and categories,
incrementally exposing the intuitive concepts of a collection and
arrow without absolute assumptions.
Succeeding contexts navigate the edemic conceit inherited from the
historical era proceeding from Plato; anthopomorphism of intuitive axiom as
universal nature or divine guidance.

Proofs of propositions, programs of a type, and mappings between structures.  These three aspects give rise to three sects of worship: Logic, which gives primacy to proofs and propositions; Languages, which gives primacy to programs and types; Categories, which gives primacy to mappings and structures.  The central dogma of computational trinitarianism holds that Logic, Languages, and Categories

Quipper Language

Collection serves the concept of a category theory entity.
Ultimately a category specializes the concept of a collection with
arrow constructs.
By fiat the arrow initially exposes a specific relation with the
concept of a traditional ordered pair.
As distinct associations the components of the pair may be the same
or different collections.
The distinctive behavior of an arrow associates a mapping from the first
collection, a domain to the second collection, a codomain.
The actual association of an arrow from domain to codomain attributes as a
relation.
Operators are special arrows, employed to expose aspects of a domain
collection.
The operators, $\oper{dom}$ and $\oper{cod}$, designate domain and codomain
respectively. The traditional operators, $\oper{fst}$ and $\oper{snd}$,
used in the Haskell programming language are reserved for other designations.
The behavior of an arrow beyond domain and codomain is ascribed by
the properties of relation.
A special relation, the membership arrow, $\in$, relates
a domain collection as a constituent of a codomain collection.
The subcollection arrow, $\lessdot$, implies
that a domain collection consists only of members contained in the codomain
collection.
Membership and subcollection arrows ascribe to different relations.
The trivial collection is the empty collection, $\varnothing$,
consists of no members. The trivial collection is a subcollection
of all collections.
An object refers to a domain collection that is a
member of a codomain collection. In general the relation of
an arrow is a behavior that attributes a collection with properties.
In general an arrow is an object that ascribes a minimal set of attributes;
domain, codomain and relation.

Category theory organizes into a paradigm of context levels.
The foundation context level 
defines collection and essential arrows such as membership and subcollection.
Above the foundation context is the order context releative to an order of
arrows.

Philosophical discourse on human congnition of the physical universe
lays witness to a storied history.
Limitations on the individial and social capacity of human cognition
sharpen and extend through the practic of reification of the
phyiscal universe.
The process of reification manifests
in pure mathematics progressively augmented with computational algorithms.
The human passion of reason animates the reification process of
physics and mathematics in defiance of Cartesian duality.
Evolution of reason from an Aquinian eschatological form of divine substance
toward a logical construct for computation remains an constant pursuit.
Plato's theory of forms idealized human cognition of pure concept
extended with the practice of dialogue to embrace the
social congnition of the ethereal.
Aristotle's hylomorphic approach emphasizes the empirical dialectic
while maintaining a similar teleological maxim.

The philosophy of science and mathematics dictates the process of
concept construction and analysis.
Concepts endure as epistomological models and theories through
empirical validation and rational deduction.
Scientific observation confirms hypothesis while
deductive reasoning upholds mathematical construct as theory.
Abjective reasoning alternately infers hypothetical mathematical constructs
from scientific observation.
In contrast, na\"{i}ve reliance on intuition as absolute concept undermines
scientific underpinings and confounds mathematics with paradox.
As witness, the haunting forms of Plato
indict human nature with intuitive albeit inspirational belief.
Intuition condemns ambiguity to many fundamental definitions and axioms.
Ideally the practice of rational hypothesis and observation
shape intuition and concept.
Ultimately, categorical construct evolves to reify dynamic observation
and hypothesis of the real world to establish a
dialectic for accurate epistemology.
Analysis and construct relative to assumptions direct the establishment of a
framework for deductive reasoning and category theory.

Computational algorithms mathematically process episodes of
assumption with a reasoning model; an epitome for the
recurrent scientific protocol of hypothesis/observation.
Deductive reasoning employs a context of conjugates
to construct complex assumptions from simple inputs.
A context essentially consists of a vocabulary of
fundamental conjugates and assumptions.
Abductive reasoning alternately employs a similar context to
equivalate an input with semantically similar complex assumptions.
Abductuve reasoning supports the scientific development of
complex hypotheses from observation.
The guidance of algorithm implementation within a context establishes the
boundary constraints and reperetoire for deductive reasoning.
Context supports of a construct of complex assumptions
in conformance with various reasoning models.
The pursuit of universal constructs determine
arithmetic, fields of mathematics or models of physical reality.
Hilbertian axiometrics represents an extreme set model of logical analysis
and construction of prior truths.
Historically the second problem in mathematics,
as stated by David Hilbert in 1900,
prognosticated that a logical algorithm with fundamental assumptions would
prove a consistent arithmetic during the 20\textsuperscript{th} century.
Kurt G\"{o}del ultimately proved the mutual exclusion of both
consistency and completeness for any logical construct over axioms,
including arithmetic.
As such G\"{o}del's Theorem imposes boundaries,
often cited as paradoxes, on mathematical construct and analysis.
In corollary, the computational halting problem states the imposibility of 
determining convergence for all algorithms acting on possible input assumptions.
Herin the applicaton of reasoning and structure are applied to navigate
paradoxes in algorithm development.
All reasoning models rely on formal assumptions, possibly axioms,
to structure graphs that uphold a reasoning calculus. 
Formal assumptions establish a contextual foundation to mediate assertions
tested by deductive calculus.

The primitive assumption of order engenders the definition of sequent.
Sequent girds the construct of complex assumptions from priors.
Sequent alludes to a declensional variation of sequence.
The symbol, $\vdash$, denotes a sequent as follows:

\begin{definition}{Sequent}\label{def:sequent},
  denoted with symbol, $\vdash$, relates an
  antecedent, $\obk{a}$, to the consequent, $\obk{c}$.
\begin{equation*}
\obk{a}\vdash\obk{c}
\end{equation*}
A cedent refers either an antecedent or consequent mutually
associated by a sequent
  \begin{equation*}
  \begin{array}{rll}
   Antecedent&-&\text{an assumption} \\[3pt]
   Consequent&-&\text{an assumption dependent on antecedent}
  \end{array}
  \end{equation*}
\end{definition}


A conditional constraint augments a sequent relation.
Minimally, a sequent relates a condition of order 
from antecedent to subsequent assumptions.
Sequent further consists of a nontrival collection of cedents
discriminated by order as antecedent and consequent.

As a construct, a sequent acts as a conjugate relating
a antecedent to consequent to render a complex assumption.
As definition prevails a complex assumption forms a graph of
constituent assumptions.

An existential contraint extends a sequent to declare assumptions.
The elementary declaration of fundamental assumptions requires a
sequent endowed with an existential relation.
A fundamental existential consequent has no antecedent.
Essentially an existential sequent is declares a
consequent from a nothing antecedent.

\begin{definition}{Existential Sequent, $\exists$}
  \label{def:existentialdeclare},
  determines a consequent with the nothing antecedent.
  \begin{equation*}
  \begin{array}{rll}
   &\exists&\obk{a} \\[3pt]
   \exists&\quad&\text{Existential sequent}\\[3pt]
   &\quad&\text{nothing antecedent} \\[3pt]
   \obk{a}&\quad&\text{existentially declared assumption, } \\[3pt]
  \end{array}
  \end{equation*}
\end{definition}

Existential declaration supports the definition of atomic assumptions
with a dependency on the nothing assumption.
Implication declares a useful dependency sequent.
Several sequents combine to existentially a declaration the
implecation sequent.
Fractional notation encapsulates nested sequent relations to
improve the interpretation of deductive structure.

\begin{equation*}
\frac{\obk{a}\vdash\obk{c}}
{\exists\,\obk{a}\Rightarrow\obk{c}}
\end{equation*}  

Implies indicates that the antecedent assumption qualifies as the
consequent. Mutual equivalence is denoted with the symbol, $\Leftrightarrow$.

\begin{equation*}
\frac{\obk{a}\vdash\obk{c},\;\obk{c}\vdash\obk{a}}
{\exists\,\obk{a}\Leftrightarrow\obk{c}}
\end{equation*}  

A collection more general than sequent consists of
assumptions without an order constraint.
A membership sequent declares an antecedent assumption as a
constituent of a collection.
Declaring a collection to contain a member proceeds:

\begin{definition}{Membership Sequent}\label{def:membership},
  implies a nontrivial consequent collection, $\cat{C}$,
  consists of an antecedent assumption, $\obk{a}$,
  \begin{equation*}
  \begin{array}{rll}
   \obk{a}&\in&\cat{C} \\[3pt]
   \in&\quad&\text{membership sequent} \\[3pt]
   \obk{a}&\quad&\text{assumption of membership} \\[3pt]
   \cat{C}&\quad&\text{nontrivial collection} \\[3pt]
   &&\text{containing the antecedent,}
  \end{array}
  \end{equation*}
\end{definition}

Membership is a restriction on the existential declaration.
The antecedent of membership is a constituent of a
nontrivial consequent collection.
Several sequents combine to form a complex declaration.

\begin{equation*}
\frac{\exists\,\obk{a},\;\exists\,\cat{C},\;\obk{a}\vdash\cat{C}}
{\obk{a}\in\cat{C}}
\end{equation*}  

A collection constraint implies a antecedent contained within a consequent; a
collection entity. The collection constraint strictly restricts the collection
to cnosist exclusively of the antecedent.

\begin{definition}{Initial Collection}\label{def:initialcollection},
  A nothing antecedent within a collection context declares the
  initial collection.
  \begin{equation*}
  \begin{array}{rll}
   &\vdash_{\{\}}&\catN{0} \\[3pt]
   &\quad&\text{nothing antecedent} \\[3pt]
   \{\}&\quad&\text{collection context} \\[3pt]
   \catN{0}&\quad&\text{initial collection} \\[3pt]
    &&\text{existentially defined}
  \end{array}
  \end{equation*}
\end{definition}

Elaborating further, the collection constraint applied to an
assumption antecedent implies a singleton collection or a
collection with one member:

\begin{definition}{Singleton Collection}\label{def:singletoncollection},
  An antecedent, $\obk{a}$, within a collection constraint declares the
  singleton collection.
  \begin{equation*}
  \begin{array}{rll}
   &\obk{a}\vdash_{\{\}}&\{\obk{a}\} \\[3pt]
   &\obk{a}\quad&\text{antecedent} \\[3pt]
   \{\}&\quad&\text{collection constraint} \\[3pt]
   \obk{a}&\quad&\text{singleton collection}
  \end{array}
  \end{equation*}
Alternately, denote a singeton collection, $\{\obk{a}\}$,
consisting of one member, $\obk{a}$, as $\singleton{a}$.
\end{definition}

The terminal collection is a special singleton collection consiting of the
initial collection:

\begin{definition}{Terminal Collection}\label{def:singletoncollection},
  An antecedent, $\catN{0}$, within a collection context declares the
  terminal collection.
  \begin{equation*}
  \begin{array}{rll}
   &\catN{0}\vdash_{\{\}}&\catN{1} \\[3pt]
   &\catN{0}\quad&\text{antecedent} \\[3pt]
   \{\}&\quad&\text{collection context} \\[3pt]
   \catN{1}&\quad&\text{terminal collection}
  \end{array}
  \end{equation*}
\end{definition}

The prior introduction of sequents with special constraints
has defined a sample of assumptions both complex and atomic.
The collection of atomic with complex assumptions
collectively appropriates a context.
A suitible context supports reasoning to generate complex assumptions.
A universal collection consists of all contexts and assumptions.
In context a sequent mediates the combinatorial construction of a
consequent assumption from an antecedent collection.
Sequent extends universal collection to an equivalent universal construct.
The concept of variable further implies a symbol to assert any member
of a collection within a specific expression. 
The for all sequent denotes a variable expression universally:

\begin{definition}{For all Sequent, $\forall$}\label{def:forallsequent},
  implies a variable, $\obk{x}$, as any assumption of the universal context.
  \begin{equation*}
  \begin{array}{rll}
   &\forall&\obk{x} \\[3pt]
   \forall&\quad&\text{for all sequent} \\[3pt]
   &\quad&\text{implied universal construct} \\[3pt]
   \obk{x}&\quad&\text{variable assumption}
  \end{array}
  \end{equation*}
\end{definition}

The for all sequent provides a definition of a global condition
throughout a collection.
For example, all assumptions declared as self implied within universal context:

\begin{equation*}
\frac{\forall\obk{x}}{\obk{x}\Leftrightarrow\obk{x}}
\end{equation*}

Collections infer a membership relationship amongst constituent assumptions.
A self sequent relation is a specialized singleton collection
where the unique assumption is associated with a
first and second order characteristic.
More generally sequent defines an ordered collection wherein
assumptions are attributed by order in the collection.

A graph is a collection with order attributes designated by sequents on members.
A member of a graph associated only by a self sequent is isolated.
A collection consisting of assumptions associated by sequents
defines a directed graph.
A graph of assumptions is itself a complex assumption.
Sequent associations determine a concept of order for a collection.
An unordered collection consists exclusuvely of isolated assumptions.
A trivial or empty collection consists of no assumptions, otherwise the
collection consists of one or more assumptions.

In deductive analysis a context validates an antecedent graph.
Alternately a context provides deductive reasoning to construct
a consequent graph from an antecedent collection.
A graph limited by rules of deduction exists within a context.
A context of deduction attributes a common methodology for the
construction or analysis of graphs.
Analysis refers to a process of interpretation of a graph within a context.
Ideally the deductive  construction or analysis of a graph conforms to a
consistent context.
The context of natural deduction establishes the idea
of reasoning from a context.

\begin{definition}{Context}\label{def:context},
  a specialization, $R$, of a sequent
\begin{equation*}
\frac{R,\,\vdash}{\vdash_R}
\end{equation*}  
A metaconcept specializing a sequent is a \textbf{context}.
The antecedent, $R,\,\vdash$, renders a consequent, $\vdash_R$.
\end{definition}

A context of a sequent interpretation supports the deductive
construction or analysis of a directed graph, a deductive graph.
A consequent assumption conforms to an antecedent collection of assumptions
relative to a context for deductive reasoning.
A graph of assumptions that conform to a formal context denotes a proposition.
An independent proposition denotes an axiom, an intuitive truth.

Initial constructs begin with a nothing context to introduce
axioms and special operations.
Incremental extensions of context organize a structured development
of mathematical construction or analysis.

Other than context, the specialization of a sequent serves to
define various transformations and relations between cedents to
extend the syntax of deductive context.
Fractional notation introduces the the concept of algorithmic structure
and procedures for deductive reasoning bounded within a context.
A membership specialization implies a nontrivial collection.

A collection context associates an antecedent with a collection
construct.
A trivial antecedent relates to the construction of the
trivial collection, $\bot$.

\begin{definition}{Trivial Collection}\label{def:trivialcollection},
  consists of a nothing antecedent with collection context to yield
  an initial or trivial collection, $\bot$.
  \begin{equation*}
  \begin{array}{rll}
   &\vdash_{\{\}}&\bot \\[3pt]
   \{\}&\quad&\text{collection context} \\[3pt]
   &\quad&\text{trivial antecedent and context} \\[3pt]
   \bot&\quad&\text{initial (bottom) collection}
  \end{array}
  \end{equation*}
\end{definition}

Combining membershio and collection context defines any
assumption as a member of a singleton collection.

\begin{definition}{Singleton Collection}\label{def:singletoncollection},
  consists of an arbitrary antecedent with collection context to yield
  a singleton collection consisting of a single member, the antecedent.
  \begin{equation*}
  \begin{array}{rll}
   &\vdash_{\{\}}&\bot \\[3pt]
   \{\}&\quad&\text{collection context} \\[3pt]
   \obk{a}&\quad&\text{antecedent assumption} \\[3pt]
   \{\obk{a}\}&\quad&\text{singleton collection} \\[3pt]
   &&\quad\text{consisting of singlular member, $\obk{a}$}
  \end{array}
  \end{equation*}
\end{definition}

Equivalence refers to a particular specialization meaning
that an unordered collection shares a common association.
Denote the euivalence sequent as $\sim$.

\begin{definition}{Equivalence}\label{def:equivalence},
  a specialization, $\sim$, of a sequent
\begin{equation*}
\frac{\sim,\,\vdash}{\vdash_\sim}
\end{equation*}  
An equivalence specialization within a context,
such that all antecedents of are symmetrically sequent. 
\begin{equation*}
\frac{\sim,a,b}{a\sim b,\;b\sim a}
\end{equation*}  
\end{definition}

Equivalence indicates mutual dependence within a context
over a nontrivial antecendent collection.

Boolean or Heyting algebeas serve as example contexts for
the deductive construction or analysis of propositions/theorems.
Generally, the design of the context model supports a reperetoire of deductive
approaches for developing arithementic and mathematics. 

\begin{definition}{Proposition}\label{def:proposition},
  an assumption that conforms to the deductive analysis of a context.
  A $0^{th}$ order proposition is atomic and denoted as axiom. 
  A higher order "complex" proposition consists of constituent propositions.
  A \textbf{theorem} is a complex propostion.
\end{definition}

Judgement processes the deductive construction or analysis of an
assumption as either a proposition or an invalid/inconsistent assumption.
Judgement operates within a context.

\begin{definition}{Judgement}\label{def:judgement},
  a process, metaproposition, that invokes a context, $R$, of deduction
  to evaluate a collection of antecedent
  judgements, $J_1,\dots$, as a consequent judgement, $J$.
  \begin{equation*}
  J_1,\dots\,\vdash_R J
\end{equation*}
  A judgement of a collection of axioms, $J_a$,
  results from an trivial antecedent and context.
\begin{equation*}
  \vdash J_a
\end{equation*}  
  A judgement process constructs a proposition from an antecedent collection or
  alternately evaluates an antecedent as a proposition or invalid assumption.
  Construction of a complex consequent judgement folloes from a plurality of
  antecedent judgements.
  Judgement transforms an antecedent collection of judgements to a consequent
  proposition or invalid. Invalid implies non sequitur.
  \begin{equation*}
  \begin{array}{rll}
   Antecedent&-&\text{collection of judgements} \\[3pt]
   Consequent&-&\text{a consequent judgement,} \\[3pt]
     &\quad&\text{a graph of antecedent judgements}
  \end{array}
  \end{equation*}
  A judgement on an antecedent consisting of a non sequitur judgement
  is non seqitur.
\end{definition}

The antecedent refers to a collection of component judgements,
whereas the consequent judgement, a graph of judgements.
A proposition cannot consist of a non sequitur judgement.
A judgement constructs an output proposition based
on antecedents, otherwise the result is invalid, non sequitur.
Combinatorial outcomes of judgement vary based on context.

Judgement on the nothing cotext renders any proposition not limited by
the context.

\begin{definition}{Existential Judgement}\label{def:existentialjudgement},
  Judgement of the nothing domain determines an initial collection.
  \begin{equation*}
  \begin{array}{rll}
   &\exists&\catN{o} \\[3pt]
   &\quad&\text{nothing context} \\[3pt]
   \obk{o}&\quad&\text{arbitrary collection,} \\[3pt]
    &&\text{existentially defined}
  \end{array}
  \end{equation*}
\end{definition}

Existential judgement determins an everything context.
The everything and nothing cotext share a lack of structure.
The for all judgement therefore denotes:


\begin{definition}{For all Judgement}\label{def:foralljudgement},
  For all judgement of everything domain relative to any codomain collection.
  \begin{equation*}
  \begin{array}{rll}
   &\forall& P \\[3pt]
   &\quad&\text{everything collection} \\[3pt]
   P&\quad&\text{arbitrary proposition from everything collection}
  \end{array}
  \end{equation*}
\end{definition}

A proposition of that a plurality of propositions is a nontrivial collection
with members.

\begin{definition}{Plural Judgement}\label{def:pluraljudgement},
  Plural judgement determines that a domain of one or more proposition
  constitutes a codomain collection.
  \begin{equation*}
  \begin{array}{rll}
   P_1,\dots&\vdash& P \\[3pt]
   P_1,\dots&\quad&\text{one or more propositions} \\[3pt]
   P&\quad&\text{a proposition collection}
  \end{array}
  \end{equation*}
\end{definition}

Logical arrows denote constructs relative to plural propositions.
Union, $\vee$, and intersection, $\wedge$, denote join and meet of propositions.
The symbols, $\vee$ and $\wedge$, represent general meet and join
with various applications in different contexts.
Simple union and intersection pertain to set operations.
Logical arrow denote rules for construction of statements in a mathematical
language.

Implication denotes a stricter judgement. Looser judgements such as
existential declaration require little or no logical pretext.
Implication renders judgement on the application of various logical
operations on propositions to hypothesize proposition structure.
As a language example, a domain propositions consiting of clauses implies
a codomain sentence structured with the domain clauses.

\begin{definition}{Hypothetical Judgement}\label{def:hypotheticaljudgement},
  implies a logical association of a domain consisting of one or more
  propositions as a codomain sentence construct.
  \begin{equation*}
  \begin{array}{rll}
   P_1,\dots&\Rightarrow& P \\[3pt]
   P_1,\dots&\quad&\text{one or more propositions} \\[3pt]
   P&\quad&\text{implied proposition, sentence construct}
  \end{array}
  \end{equation*}
\end{definition}

Fractional notation equivalates the denominator justification with
an implied and logically stricter numerator justification.

\begin{definition}{Hypothetical Judgement}\label{def:hypotheticaljudgement},
  Hypothetical judgement implies a logical association of domain of one or more proposition constitutes a codomain sentence construct.
  \begin{equation*}
     \frac{P_1,\dots\vdash P}{\vdash P_1,\dots\Rightarrow P}
  \end{equation*}
  \begin{equation*}
  \begin{array}{rll}
   P_1,\dots&\Rightarrow& P \\[3pt]
   P_1,\dots&\quad&\text{one or more propositions} \\[3pt]
   P&\quad&\text{implied proposition, logical sentence construct}
  \end{array}
  \end{equation*}
\end{definition}

A hypothetical judgement on a collection of propositions renders
a logically structured composit proposition alternately named a complex
proposition.
A membership arrow, $\in$, denotes a domain proposition that is a member
of a plural collection. Subcollection, $\lessdot$, denotes an arrow with
domain collection which consists of only members that are common with the
codomain collection. A member of a complex proposition is refered to as
term.

\begin{definition}{Membership Judgement}\label{def:membershipjudgement},
  membership association of a domain consisting of a proposition
  as amember of as a codomain sentence construct or collection.
  A member proposition is also known as a term.
  \begin{equation*}
     \frac{P_1,\dots\vdash P}{\vdash P_1\in P}
  \end{equation*}
  \begin{equation*}
  \begin{array}{rll}
   P_1,\dots&\Rightarrow& P\quad\text{terms of complex proposition} \\[3pt]
   P_1,\dots&\quad&\text{one or more propositions} \\[3pt]
   P&\quad&\text{implied complex proposition, sentence construct}
  \end{array}
  \end{equation*}
\end{definition}

Subordinate  
A collection with members sharing a similar characteristic is a type.
Members of a type conform to the for all, $\forall$, judgement.

\begin{definition}{Membership Judgement}\label{def:membershipjudgement},
  membership association of a domain consisting of a proposition
  as a member of a codomain collection.
  \begin{equation*}
     \frac{\forall p\in P}{p:P}
  \end{equation*}
\end{definition}

\begin{definition}{Subcollection/Type Judgement}\label{def:subtypejudgement},
  subcollection/subtype association of a domain consisting of a proposition
  as amember of a codomain collection.
  \begin{equation*}
     \frac{a:A\vdash a\in B}{A\lessdot B}
  \end{equation*}
\end{definition}

Logic and types provide a breif introduction to the mechanics to define
concepts and structure. A complete development of category theory requires
a thourgh understanding of mappings and trandformations.
The foundation context of category theory provides a begining.

\begin{definition}{Collection, Foundation Context}\label{def:collection},
  everything is a collection. A collection represents a concept or entity
  associated with a label or identity.
  The trivial collection, $\varnothing$, is uniquely empty,
  consisting of nothing. Non-trival collections consist of other collections.
  Category specializes collection. Objects are constituents of a collection.
  A collection associates with attributes as follows:
\begin{itemize}
\item\textbf{(Arrow)} an order pair consisting of a domain collection
  associated with a codomain collection.
  \begin{itemize}
    \item\textbf{(Domain)} a binding of the arrow
    \item\textbf{(Codomain)} the referent binding of the arrow
 \end{itemize}
\item\textbf{(Property)} a codomain associated with a domain collection.
  A property attributed to a collection or the constituents of a collection
\item\textbf{(Relation)} a codomain associated with a domain arrow.
  Membership, subcollection and equals are specific examples of relations.
  A property associated with the domain to codomain pair of an arrow.
  Denote the general collection of relations as a collection, $\mathcal{C}$
  \begin{itemize}
    \item\textbf{(Membership, $\in$)} an arrow, the domain
      is a constituent collection of the codomain collection.
      The domain of membership is an \textbf{object} of the
      codomain of membership. A domain or codomain must be objects
      as members of the collection for an arrow
    \item\textbf{(Subcollection, $\lessdot$)} an arrow, the constituents of the
      domain collection are constituents of the codomain collection
    \item\textbf{(Equals, $=$)} an arrow, with domain and codomain
      the same collection, an identity arrow of this collection.
      Two objects, $\obk{a}$ and $\obk{b}$, that commute relative
      to the subcollection arrow, $\obk{a}\lessdot\obk{b}$ and
      $\obk{b}\lessdot\obk{a}$, then $\obk{a}=\obk{b}$
 \end{itemize}
\item\textbf{(Equivalence, $\sim$)} an arrow, an unordered pair
  of domain and codomain collections. The domain and codomain
  collections are equivalent relative to common properties.
  Equality implies equivalence but equvalence does not imply equality.
  Equivalence is a partial condition but not an absolute requirement
  for equality.
\end{itemize}
\end{definition}

The arrow is peculiar to category theory. An arrow exists as a relation
with a particular domain/codomain object pair.
In set theory a relation is a concept judged to be defined or undefined
in regard to a given domain/codomain object pair.
For category theory an arrow with domain/codomain pair $\cpair{a}{b}$
is different from an arrow with domain/codomain pair, $\cpair{c}{d}$,
assuming $\obk{a}\neq\obk{c}$ or $\obk{b}\neq\obk{d}$.
Furthermore two arrows with distinct relations are distinct even with an
identical domain/codomain pair.

As a structural tool, arrows compose with an algebra, compositional algebra,
to construct diagrams of directed arrows and objects, essentially a graph.
Diagrams and graphs likewise specify algorithms and
proceedures with the algebra.

MacLane's in fact introduced meta graphs as a preliminary to category theory.
Edemic to category theory is the practice of abstracting objects of
a diagram as meta entities with discriminating identities to focus on the
structure of the arrows interconnecting the objects.

\begin{definition}{Composition Algebra}\label{def:composition},
  each arrow, $\arr{r}$, directs an association from a
  domain object to a codomain object.
  The domain and codomain objects are a referent pair of the  arrow,
  $\arr{r}$; designated by properties,
  $\oper{dom}$ and $\oper{cod}$, respectively:
  \begin{itemize}
    \item\textbf{(Domain)}, $\arrdom{r}$, domain object
    \item\textbf{(Codomain)}, $\arrcod{r}$, codomain object
    \item\textbf{(Relation)}, $\arrrel{r}\in\mathcal{R}$,
      arrow characteristic
    \item\textbf{(Map Notation)},
      $\arr{r}:\,\arrdom{r}\rightarrow\arrcod{r}$ \\
      $\qquad$ or $\quad\arrdom{r}\xrightarrow{\;\obk{r}\;}\arrcod{r}$
    \item\textbf{(Composition Algebra)},\\ given arrows,
      $\arr{r}$,$\arr{s}$,$\arr{t}$, where\\
       $\arrcod{r}=\arrdom{s}$ and $\arrcod{s}=\arrdom{t}$
      \begin{itemize}
      \item\textbf{(Binary operator, $\circ$)}
        $\arr{s}\circ\arr{r}=\arr{s\circ r}$,\\
        $\qquad\arrdom{r}\xrightarrow{\;\obk{s}\circ\obk{r}\;}\arrcod{s}$
      \item\textbf{(Associative)}
        $(\arr{t}\circ\arr{s})\circ\arr{r}=\arr{t}\circ(\arr{s}\circ\arr{r})$
      \item\textbf{(Identity)} 
      \begin{itemize}
        \item\textbf{(Idempotency)} $\arrid{o}\circ\arrid{o}=\arrid{o}$
        \item\textbf{(Composition Identity)}
        $\arr{r}=\arr{r}\circ\arrid{\arrdom{r}}=\arrid{\arrcod{r}}\circ\arr{r}$
      \end{itemize}
      \end{itemize}
  \end{itemize}  
\end{definition}

Collection concepts apply to functional and object oriented programming.
In a practical context the abstraction
of collection identity supports lazy instantiation
of program constructs and computer resource management.
In complement, Mac Lane, \cite{maclane1998}, introduced the
the concept of meta-categories prior to penetrating set
and more specific category definitions. Reflexively, category
theory admits and partitions meta concepts in complement
with specific concomitant definitions.
The prefered symbol here, $\lessdot$, indicates a broader subcollection
relation extending to the specialized collection, category,
hence the traditional set operator symbol, $\subset$ is not used.

Within the order context each arrow is attributed with order.
A zero order arrow is a collection unendowed with domain or codomain. 
Arrows with positive integer order endow a mapping behavior;
an $n^{th}$ order arrow associates an $n-1^{st}$ order domain arrow to an
$n-1^{st}$ order codomain arrow.
The equal arrow, $=$, is a special equivalance arrow, $\sim$.
The equial arrow is the identity of an a collection.
Arrows with negative order
are detailed in following sections but in breif a negative order arrow
associates a domain arrow with a codomain constiting of
lower order arrows. The domain and codomain operators, $\oper{dom}$ and
$\oper{cod}$, are arrows of $-1^{st}$ order.
A higher context consists of a hierarchy based on category with
associated maps of categories such as functor and natural transform.




\section{Appendix}

\bibliography{sample-paper}

\bibliographystyle{prsty}
\begin{thebibliography}{99}
\bibitem{cartan1966}Cartan,\'{E}., ``The Theory of Spinors'', Hermann, [1966]
\bibitem{cartan1913}Cartan,\'{E}., ``Les Groupes Projectifs qui ne laissent
  invariante aucune multiplicit\'{e} plane'', Bull. Soc. Math. France, [1913]
\bibitem{grothendieck1957}Grothendieck,A., ``Sur quelques points d`Alg\`{e}bra Homologique'', T\^{o}hoku Math. J., 9, pp. 119-221, [1957]
\bibitem{brauerweyl1937}Brauer,R. and Weyl,H., ``Spinors in n Dimensions'', Am. J. Math., 57, pp. 425-449, [1937]
\bibitem{russell1902}Russell,B., Letter to Frege,In Heijenoort 1967, pp. 124-125, [1902]
\bibitem{russell1903}Russell,B., ``The Principles of Mathematics'', Cambridge Univ. Press, Cambridge, Vol. I, [1903]
\bibitem{pauli1927}Pauli,W., ``Zur Quantenmechanik des Magnetischen Elektrons'', Z. Phys., 43, pp. 601-623, [1927]
\bibitem{zermelo1908}Zermelo,E., ``Untersuchungen \"{U}ber die Grundlagen der Mengeniehre``, Math. Ann., 65, pp. 261-281, [1908]
\bibitem{schafer2008}Schafer,R.D., ``An Introduction to Nonassociative Algebras'', Project Gutenberg(Public Domain USA). Internet, [2008]
\bibitem{dirac1928}Dirac,P.A.M., ``The Quantum Theory of the Electron'', Proc. R. Soc. \(London\), [1928]
\bibitem{birkoff1935}Birkoff,G.D., ``On the Structure of Abstract Algebras'', Proc. Cambridge P. Soc. 31, pp. 433-454, [1935]
\bibitem{jipsenrose1992}Jipsen,P.,Rose,H.``Varieties of Lattices'', Springer-Verlag,Berlin-Heidelberg, [1992]
\bibitem{hodges1993}Hodges,W., ``Model Theory'', Cambridge Univ. Press, Cambridge, [1993]
\bibitem{lawvere1965}Lawvere,F.W., ``An Elementary Theory of the Category of Sets'', Proc. N. Acad. Sc., PNAS, [1965]
\bibitem{lawvere1997}Lawvere,F.W.,Shanuel,S.H., ``Conceptual Mathematics, A first Introduction to Categories'', Cambridge Univ. Press, Cambridge, [1997]
\bibitem{awodey2010}Awodey,S., ``Category Theory'', Oxford Univ. Press, Oxford, [2010]
\bibitem{barrwells1990}Barr,M.,Wells,.C, ``Category Theory for Computing Science'', Prentice Hall, [1990], pdf version online [1998]
\bibitem{lang2002}Lang,S., ``Algebra'', Springer-Verlag, NY, [2002]
\bibitem{maclane1986}Mac Lane,S., ``Mathematics Form and Function'', Springer-Verlag, NY, [1986]
\bibitem{maclane1998}Mac Lane,S., ``Category Theory for the Working Mathemetician'', Second Edition, Springer-Verlag, NY, [1998]
\bibitem{bourbaki1989}Bourbaki,N., ``Algebra I'', Springer-Verlag, Berlin Heidelberg, [1989]
\bibitem{bourbaki22003}Bourbaki,N., ``Algebra II'', Springer-Verlag, Berlin Heidelberg, [2003]
\bibitem{hungerford1974}Hungerford,T.W., ``Algebra'', Springer-Verlag, NY, [1974]
\bibitem{burris1981}Burris,S.,Sankappanavar,H.P., ``A Course in Universal Algebra'', Springer-Verlag, NY, [1981]
\bibitem{rotman1998}Rotman,J., ``Galois Theory'', Springer-Verlag, NY, [1998]
\bibitem{fultonharris2004}Fulton,W., Harris,J., ``Representation Theory: A First Course'', Springer-Verlag, NY, [2004]
\bibitem{humphreys1972}Humphreys,J., ``Introduction to Lie Algebras and Representation Theory'', Springer-Verlag, NY, [1972]
\bibitem{brouwer1911}Brouwer,L.E.J., ``\"Uber Abbildung von Mannigfaltigkeiten'', Math. Ann., 71, pp. 97-115, [1911]
\bibitem{birkhoff1913}Birkhoff,G.D., ``Proof of Poincar\'e's Geometric Theorem'', Trans. Amer. Math. Soc., 14, pp. 14-22, [1913]
\bibitem{birkhoff1925}Birkhoff,G.D., ``An Extension of Poincar\'e's Last Geometric Theorem'', Acta Math., 47, pp. 297-311, [1925]
\bibitem{poincare1912}Poincar\'e,H., ``Sur un Theor\`eme de G\'eom\'etrie'', Rend. Circ. Mat. Palermo, 33, pp. 375-407, [1912]
\bibitem{brownnwumann977}Brown,H.,NeumannD., ``Proof of the Poincar\'e-Birkhoff Fixed-Point Theorem'', Michigan Math. J., 24, pp. 21-31, [1977]

\end{thebibliography}

\end{document}
