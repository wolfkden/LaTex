%\documentclass[aps,twocolumn,secnumarabic,graphics,flotfix,graphicx,
%url,bm,tightenlines,nobibnotes,nobalancelastpage,amsmath,amssymb,
%nofootinbib]{article}
\documentclass[aps,twocolumn,secnumarabic,nobalancelastpage,amsmath,amssymb,
amsthm,nofootinbib,parskip=full]{revtex4}

% standard graphics specifications
% alternative graphics specifications
% helps with long table options
% for on-line citations
% special 'bold-math' package

\usepackage{utf8}{inputenc}
%\usepackage[bottom]{footmisc}
\usepackage{titlesec}
\usepackage{mathtools}
\usepackage{graphics}      % standard graphics specifications
\usepackage{graphicx}      % alternative graphics specifications
\usepackage{longtable}     % helps with long table options
\usepackage{url}           % for on-line citations
\usepackage{bm}            % special 'bold-math' package
%\usepackage{unicode-math}
\usepackage{stmaryrd}
\usepackage{mathrsfs}
%\usepackage{mathabx}
%\setmathfont{XITS Math}
%\setmathfont[version=setB,StylisticSet=1]{XITS Math}
\usepackage[utf8]{inputenc}
\usepackage[english]{babel}
\usepackage{pgf, tikz, tikz-cd}
\usepackage{chngcntr}
\counterwithin{figure}{section}
\usepackage{calrsfs}
\usepackage{stringstrings}
\usepackage{xstring}
\usepackage{xparse}

\DeclareMathAlphabet{\pazocal}{OMS}{zplm}{m}{n}
%\DeclareMathAlphabet{\eurocal}{OMS}{zplm}{m}{n}
\usepackage{calligra}
\usepackage[T1]{fontenc}
\DeclareFontShape{T1}{calligra}{m}{n}{<->s*[1.2]callig15}{}
\DeclareMathAlphabet{\classscr}{T1}{calligra}{m}{n}
\usepackage[mathscr]{euscript}
\let\euscr\mathscr \let\mathscr\relax% just so we can load this and rsfs
\usepackage[scr]{rsfso}
\usepackage{relsize}
\usepackage{physics}
%\usepackage{wasysym}
\DeclareSymbolFont{extraitalic}      {U}{zavm}{m}{it}
\DeclareMathSymbol{\Qoppa}{\mathord}{extraitalic}{161}
\DeclareMathSymbol{\qoppa}{\mathord}{extraitalic}{162}
\DeclareMathSymbol{\Stigma}{\mathord}{extraitalic}{167}
\DeclareMathSymbol{\Sampi}{\mathord}{extraitalic}{165}
\DeclareMathSymbol{\sampi}{\mathord}{extraitalic}{166}
\DeclareMathSymbol{\stigma}{\mathord}{extraitalic}{168}
\DeclareMathSymbol{\Koppa}{\mathord}{extraitalic}{163}
\DeclareMathSymbol{\koppa}{\mathord}{extraitalic}{164}
\DeclareMathSymbol{\backepsilon}{\mathord}{extraitalic}{169}
\DeclareMathSymbol{\vareth}{\mathord}{extraitalic}{170}
\DeclareMathSymbol{\Digamma}{\mathord}{extraitalic}{159}
\DeclareMathSymbol{\Varkappa}{\mathord}{extraitalic}{160}
\DeclareMathSymbol{\vardigamma}{\mathord}{extraitalic}{154}
\DeclareMathSymbol{\Varrho}{\mathord}{extraitalic}{156}
\DeclareMathSymbol{\snatural}{\mathord}{extraitalic}{157}

\usetikzlibrary{matrix,arrows, decorations,
  positioning, automata, calc, fit, shapes.geometric}
\newtheorem{theorem}{Theorem}[section]
\newtheorem{proposition}{Proposition}[section]
\newtheorem{corollary}{Corollary}[theorem]
\newtheorem{lemma}[theorem]{Lemma}
\newtheorem{proof}{Proof}[theorem]
\newtheorem{axiom}{Axiom}

\numberwithin{equation}{section}

%%\theoremstype{definition}
\newtheorem{definition}{Definition}[section]
%\newtheorem*{remark}{Remark}

%\setlength{\parskip}{\baselineskip}
%\setlength{\parskip}{4em}

\tikzset{every loop/.style={min distance=12mm,looseness=10}}
\tikzset{place/.style={circle,thick,minimum size=8mm}}
\tikzset{
modal/.style={>=stealth’,shorten >=1pt,shorten <=1pt,auto,node distance=1.5cm,
semithick},
world/.style={circle,draw,minimum size=0.5cm,fill=gray!15},
point/.style={circle,draw,inner sep=0.5mm,fill=black},
reflexive above/.style={->,loop,looseness=7,in=120,out=60},
reflexive below/.style={->,loop,looseness=7,in=240,out=300},
reflexive left/.style={->,loop,looseness=7,in=150,out=210},
reflexive right/.style={->,loop,looseness=7,in=30,out=330}
}

\newcommand*{\xslant}[2][70]{%
  \begingroup
    \sbox0{#2}%
    \pgfmathsetlengthmacro\wdslant{\the\wd0 + cos(#1)*\the\wd0}%
    \leavevmode
    \hbox to \wdslant{\hss
      \tikz[
        baseline=(X.base),
        inner sep=0pt,
        transform canvas={xslant=cos(#1)},
      ] \node (X) {\usebox0};%
      \hss
      \vrule width 0pt height\ht0 depth\dp0 %
    }%
  \endgroup
}

%\DeclareRobustCommand{\innerd}[1]{\text{\scalebox{1}{$\mathsf{i}$}}_{#1}}
\DeclareRobustCommand{\xssmall}[2]{\text{\scalebox{#1}{$\mathsf{#2}$}}}

\NewDocumentCommand{\tensorkor}{m >{\SplitList{,}}O{}}
 {\begingroup
%  \newcommand\object{\vphantom{\,}}
%   \def\@temp{\Processlist{#2}{\dotensorpre}\IfEq{#1}{d}{\xssmall{.8}{#1}}{#1}}
   \def\@temp{\IfEq{#1}{d}{\xssmall{.8}{#1}}{#1}}
   \@temp
  \newcommand\object{\vphantom{\@temp}}
  \ProcessList{#2}{\dotensorkor}%
  \endgroup}

\NewDocumentCommand{\dotensorpre}{m}
{%
% {\object}#1
% \IfBeginWith{#1}{!}{\,^a{\object}}{{\object}#1}
 \IfBeginWith{#1}{!}{\object^{\StrBehind{#1}{!}}}{\object}
}
\NewDocumentCommand{\dotensorkor}{m}
{%
% {\object}#1
% \IfBeginWith{#1}{!}{\,^a{\object}}{{\object}#1}
 \IfBeginWith{#1}{!}{\object}{{\object}#1}
}

\makeatletter
\newcommand*{\xslantmath}{}
\def\xslantmath#1#{%
  \@xslantmath{#1}%
}
\newcommand*{\@xslantmath}[2]{%
  % #1: optional argument for \xslant including brackets
  % #2: math symbol
  \ensuremath{%
    \mathpalette{\@@xslantmath{#1}}{#2}%
  }%
}
\newcommand*{\@@xslantmath}[3]{%
  % #1: optional argument for \xslant including brackets
  % #2: math style
  % #3: math symbol
  \xslant#1{$#2#3\m@th$}%
}
\makeatother

\newcommand{\nlessdot}{\mathrlap{\lessdot}\;|\;\,}

\newcommand{\gtrplus}{\mathrlap
  {\textbf{\raisebox{1.15pt}{\hspace{2.6pt}\scalebox{.4}{+}}}}>}
\newcommand{\padtxt}[1]{\hspace*{1em}\parsecat{#1}\hspace*{1em}}

\newcommand{\abk}[1]{\xslantmath{\mathfrak{#1}}}
%\newcommand{\subp}[1]{\xslantmath{\mathscr{#1}}}
\newcommand{\subp}[1]{\mathbf{#1}}
\newcommand{\obk}[1]{\mathpzc{#1}}
\newcommand{\diag}[1]{\mathbf{\mathsf{#1}}}
\newcommand{\diagobj}[1]{\diag{#1}_0}
\newcommand{\diagarr}[1]{\diag{#1}_1}
\newcommand{\diagdom}[1]{\oper{dom}\diag{#1}}
\newcommand{\diagcod}[1]{\oper{cod}\diag{#1}}
%\newcommand{\cat}[1]{\pazocal{#1}}
\newcommand{\cat}[1]{\pazocal{#1}}
\newcommand{\iarr}[1]{{\large \abk{i}}_{\obk{#1}}}
\newcommand{\tarr}[1]{{\large \abk{t}}_{\obk{#1}}}
\newcommand{\earr}[1]{{\large \abk{e}}_{\obk{#1}}}
\newcommand{\earrid}[1]{\oper{id}_{\obk{#1}}}
\newcommand{\earrdom}[1]{\oper{dom}\earr{#1}}
\newcommand{\earrcod}[1]{\oper{cod}\earr{#1}}
\newcommand{\arr}[1]{{\large \abk{a}}_{\obk{#1}}}
\newcommand{\arrid}[1]{\oper{id}_{\obk{#1}}}
\newcommand{\arrdom}[1]{\oper{dom}\arr{#1}}
\newcommand{\arrcod}[1]{\oper{cod}\arr{#1}}
\newcommand{\arrrel}[1]{\oper{rel}\arr{#1}}
\newcommand{\oarr}[2]{{\large \abk{a}}_{{\subp{#1},\obk{#2}}}}
\newcommand{\oarrid}[2]{\oper{id}_{\subp{#1},\obk{#2}}}
\newcommand{\oarrdom}[2]{\oper{dom}\oarr{#1}{#2}}
\newcommand{\oarrcod}[2]{\oper{cod}\oarr{#1}{#2}}
\newcommand{\oarrrel}[2]{\oper{rel}\oarr{#1}{#2}}
\newcommand{\arrn}[2]{{\large \abk{a}}^{#2}_{\obk{#1}}}
\newcommand{\ccid}[1]{\cat{#1}_{\oper{id}}}
\newcommand{\ccobj}[1]{\cat{#1}_{\subp{0}}}
\newcommand{\ccarr}[1]{\cat{#1}_{\subp{1}}}
\newcommand{\occarr}[2]{\cat{#2}_{\subp{#1}}}
\newcommand{\ocatarr}[2]{\parsecat{#2}_{\subp{#1}}}
\newcommand{\catarr}[1]{\parsecat{#1}_{\subp{1}}}
\newcommand{\catobj}[1]{\parsecat{#1}_{\subp{0}}}
\newcommand{\owraparr}[2]{(#2)_{\subp{#1}}}
\newcommand{\wrpcobj}[1]{#1_{,\subp{0}}}
\newcommand{\wrpcarr}[1]{#1_{,\subp{1}}}
\newcommand{\owrparr}[2]{{#2}_{\subp{#1}}}
\newcommand{\wrparr}[1]{#1_{\subp{1}}}
\newcommand{\wrpobj}[1]{#1_{\subp{0}}}
\newcommand{\wraparr}[1]{(#1)_{\subp{1}}}
\newcommand{\wrapobj}[1]{(#1)_{\subp{0}}}
\newcommand{\catfull}[1]{\cat{#1}(\ccobj{#1},\ccarr{#1})}
\newcommand{\oper}[1]{\mathbf{#1}\,}
\newcommand{\catN}[1]{\mathbf{\large #1}}
\newcommand{\initcat}{\catN{0}}
\newcommand{\termcat}{\catN{1}}
\newcommand{\singleton}[1]{\catN{1}_{#1}}
\newcommand{\singletoncat}[1]{\singleton{\parsecat{#1}}}
\newcommand{\fst}[1]{\oper{fst}\,#1}
\newcommand{\snd}[1]{\oper{snd}\,#1}
\newcommand{\adj}[2]{#1\dashv #2}
\newcommand{\adjcat}[2]{\pazocal{#1}\dashv\pazocal{#2}}
\newcommand{\adjfunct}[2]{\funct{#1}\dashv\funct{#2}}
\newcommand{\adjnat}[2]{\nfunct{#1}\dashv\nfunct{#2}}
\newcommand{\adjcom}[2]
{\funct{#2}\circ\funct{#1}\dashv\funct{#1}\circ\funct{#2}}
\newcommand{\adjtpl}[3]{(\parsecat{#1},\,\funct{#2},\,\nfunct{#3})}
\newcommand{\adjdbl}[2]{(\funct{#2},\,\nfunct{#3})}
\newcommand{\montpl}[3]{\mathsf{T}(\funct{#1},\,\nfunct{#2},\,\nfunct{#3})}

\newcommand{\efarr}[3]{\oarr{#1}{\epair{#2}{#3}}}
\newcommand{\ofarr}[3]{\oarr{#1}{\ffpair{#2}{#3}}}
\newcommand{\ofarrid}[3]{\langle\arrid{{#1}^*},\,\arrid{#2}\rangle}
\newcommand{\ofarrdom}[3]{\oper{dom}\ofarr{#1}{#2}{#3}}
\newcommand{\ofarrcod}[3]{\oper{cod}\ofarr{#1}{#2}{#3}}
\newcommand{\farr}[2]{\arr{\fpair{#1}{#2}}}
\newcommand{\farrid}[2]{\langle\arrid{#1},\,\arrid{#2}\rangle}
\newcommand{\farrdom}[2]{\oper{dom}\farr{#1}{#2}}
\newcommand{\farrcod}[2]{\oper{cod}\farr{#1}{#2}}
\newcommand{\ffarrdom}[2]{\oper{dom}_{\obk{1}}\farr{#1}{#2}}
\newcommand{\ffarrcod}[2]{\oper{cod}_{\obk{1}}\farr{#1}{#2}}
\newcommand{\arrpidom}[1]{\nprojdown{1}{\oper{dom}}\arr{#1}}
\newcommand{\arrpicod}[1]{\nprojdown{1}{\oper{cod}}\arr{#1}}
\newcommand{\arrpidomn}[2]{\nprojdown{#1}{\oper{dom}}\arr{#2}}
\newcommand{\farrpicodn}[3]{\nprojdown{#1}{\oper{cod}}\oarr{#2}{#3}}
\newcommand{\farrpidomn}[3]{\nprojdown{#1}{\oper{dom}}\oarr{#2}{#3}}
\newcommand{\arrpicodn}[2]{\nprojdown{#1}{\oper{cod}}\arr{#2}}
\newcommand{\farrpidomn}[2]{\arrpidom{\oarr{#1}{#2}}}
\newcommand{\farrpicodn}[2]{\arrpicod{\oarr{#1}{#2}}}
\newcommand{\farrpidom}[2]{\arrpidom{\fpair{#1}{#2}}}
\newcommand{\farrpicod}[2]{\arrpicod{\fpair{#1}{#2}}}
\newcommand{\arrpifst}[2]{\nprojfst{1}{\farr{#1}{#2}}}
\newcommand{\arrpisnd}[2]{\nprojsnd{1}{\farr{#1}{#2}}}
\newcommand{\oarrpiop}[3]{\nprojdown{#1}{\oper{#2}}\,#3}
\newcommand{\arrpiop}[2]{\oarrpiop{1}{#1}{#2}}

\newcommand{\largenat}{\mathpzc{NAT}}
\newcommand{\wrpcat}[1]{#1_{\functscr{O}}}
\newcommand{\wrpfunct}[1]{#1_{\functscr{F}}}
\newcommand{\wrpnat}[1]{#1_{\functscr{N}}}
\newcommand{\largenatcat}{\wrpcat{\largenat}}
\newcommand{\largenatfunct}{\wrpfunct{\largenat}}
\newcommand{\largenatnat}{\wrpnat{\largenat}}
\newcommand{\cnat}[2]{\largenat\cpr{#1}{#2}}
\newcommand{\cfnat}[2]{\largenat\fpair{#1}{#2}}
\newcommand{\natfunct}[2]{\wrpfunct{\cnat{#1}{#2}}}
\newcommand{\natnat}[2]{\wrpnat{\cnat{#1}{#2}}}
\newcommand{\nfunctscr}{\functscr{N}}
\newcommand{\nfunct}[1]{\nfunctscr_{\obk{#1}}}
\newcommand{\nfunctdom}[1]{\oper{dom}\,\nfunct{\obk{#1}}}
\newcommand{\nfunctcod}[1]{\oper{cod}\,\nfunct{\obk{#1}}}
\newcommand{\nfunctobj}[1]{\nfunctscr_{obk{#1},\subp{2}}}
\newcommand{\nfunctarr}[1]{\nfunctscr_{\obk{#1},\subp{3}}}

\newcommand{\cone}[2]{\mathpzc{CONE}\cpr{#1}{#2}}
\newcommand{\conefunct}[2]{{\mathpzc{CONE}\wrpfunct{\cpr{#1}{#2}}}}
\newcommand{\conenat}[2]{{\mathpzc{CONE}\wrpnat{\cpr{#1}{#2}}}}
\newcommand{\cocone}[2]{\mathpzc{COCO}\cpr{#1}{#2}}
\newcommand{\coconefunct}[2]{{\mathpzc{COCO}\wrpfunct{\cpr{#1}{#2}}}}
\newcommand{\coconenat}[2]{{\mathpzc{COCO}\wrpnat{\cpr{#1}{#2}}}}

\newcommand{\functbag}[1]{\llbracket #1\rrbracket}

%{{\raisebox{.1\baselineskip}{\ensuremath{\digamma}}}}
%{\mathrlap{\lessdot}\;|\;\,}
\newcommand{\oprojection}[2]{\xslantmath{\mathsf{P}}_
  {\!\!\overset{#2}{#1}}}
\newcommand{\nprojection}[1]{\xslantmath{\mathsf{P}}_
  {\!\!\shortdownarrow\!\raisebox{.3\baselineskip}{\scalebox{.5}{#1}}}}
\newcommand{\nprojdown}[2]{\nprojection{#1}_{,\obk{#2}}}
\newcommand{\nprojfst}[2]{\oprojection{\shortleftarrow}{#1}\,#2}
\newcommand{\nprojsnd}[2]{\oprojection{\shortrightarrow}{#1}\,#2}
\newcommand{\simplex}[1]{\scalebox{1}[1.4]{\ensuremath{\rhd}}_{\catN{#1}}}
\newcommand{\simplexfunct}[1]{\wrpfunct{\simplex{#1}_{,}}}
\newcommand{\simplexcat}[1]{\wrpcat{\simplex{#1}_{,}}}
\newcommand{\simplexcat}[1]{\wrpcat{\simplex{#1}_{,}}}
\newcommand{\simplexnn}{\scalebox{1}[1.4]{\ensuremath{\rhd}}_{\mathbb{N}}}

\newcommand{\snat}{\mathpzc{Nat}}
\newcommand{\smallnat}[2]{\snat\cpr{#1}{#2}}
\newcommand{\smallnatobj}{\wrpcat{\smallnat}}
\newcommand{\smallnatnat}{\wrpnat{\smallnat}}
\newcommand{\smallnatfunct}{\wrpfunct{\smallnat}}
\newcommand{\snatfunct}[2]{\wrpfunct{\smallnat{#1}{#2}}}
\newcommand{\snatnat}[2]{\wrpnat{\smallnat{#1}{#2}}}
\newcommand{\snfunct}[1]{\nfunctscr_{\obk{#1}}}
\newcommand{\snfunctdom}[1]{\oper{dom}\,\nfunctscr_{\obk{#1}}}
\newcommand{\snfunctcod}[1]{\oper{cod}\,\nfunctscr_{\obk{#1}}}
\newcommand{\snfunctobj}[1]{\nfunctscr_{\obk{#1},\subp{2}}}
\newcommand{\snfunctarr}[1]{\nfunctscr_{\obk{#1},\subp{3}}}

\newcommand{\homfunct}[2]{\left[\parsecat{#1},\,\parsecat{#2}\right]}
\newcommand{\chom}[1]{\left|\carr{#1}\right|}
\newcommand{\homobj}[1]{\wrpobj{\chom{#1}}}
\newcommand{\homarr}[1]{\wrparr{\chom{#1}}}
\newcommand{\projg}[3]{\xslantmath{\mathbf{#1}}^{#3}(\mathsf{#2})}
\newcommand{\ccpairn}[2]{\nprojection{#1}(\parsecat{#2})}
\newcommand{\ccpair}[1]{\nprojection{1}(\parsecat{#1})}
\newcommand{\ccpairg}[1]{\nprojection{1}(\mathsf{#1})}
\newcommand{\apair}[2]{\langle\arr{#1},\,\arr{#2}\rangle}
\newcommand{\epair}[2]{\langle|\obk{#1},\,\obk{#2}|\rangle}
\newcommand{\ffpair}[2]{\langle\!\langle\obk{#1},\,\obk{#2}\rangle\!\rangle}
\newcommand{\fpair}[2]{\langle\obk{#1},\,\obk{#2}\rangle}
\newcommand{\fopair}[2]{\langle #1,\,#2\rangle}
\newcommand{\pair}[2]{(#1,\,#2)}
\newcommand{\cpair}[2]{(\obk{#1},\,\obk{#2})}
\newcommand{\cpr}[2]{(\pazocal{#1},\,\pazocal{#2})}
\newcommand{\ctpair}[4]{(\pair{#1}{#2},\,\pair{#3}{#4})}
\newcommand{\ccrel}[2]{\parsecat{#1}_{\obk{#2}}}
\newcommand{\csub}[2]{#1\lessdot #2}
\newcommand{\carrn}[2]{\parsecat{#1}_{\overset{#2}{\rightarrow}}}
\newcommand{\carrnobj}[2]{\wrapobj{\carrn{#1}{#2}}}
\newcommand{\carrnarr}[2]{\wraparr{\carrn{#1}{#2}}}
\newcommand{\carr}[1]{\parsecat{#1}_{\rightarrow}}
\newcommand{\carrobj}[1]{\carr{#1}_{\subp{,\,0}}}
\newcommand{\carrarr}[1]{\carr{#1}_{\subp{,\,1}}}
\newcommand{\functscr}[1]{\small\xslantmath{\pazocal{#1}}}

\newcommand{\bfunct}[1]{\functscr{F}_{\parsecat{#1}}} % behavior
\newcommand{\bfunctobj}[1]{\functscr{F}_{\parsecat{#1},\subp{0}}} % behavior
\newcommand{\bfunctarr}[1]{\functscr{F}_{\parsecat{#1},\subp{1}}} % behavior

\newcommand{\funct}[1]{\functscr{F}_{\obk{#1}}}
\newcommand{\functobj}[1]{\funct{#1}_{,\subp{1}}}
\newcommand{\functarr}[1]{\funct{#1}_{,\subp{2}}}
\newcommand{\functV}[2]{\funct{#1}(#2)}
\newcommand{\functid}[1]{\oper{id}_{\parsecat{#1}}}
\newcommand{\functdom}[1]{\oper{dom}\funct{#1}}
\newcommand{\functcod}[1]{\oper{cod}\funct{#1}}

\newcommand{\atlas}[1]{\functscr{A}_{\pazocal{#1}}}
\newcommand{\atlasobj}[1]{\atlas{#1}_{,\subp{1}}}
\newcommand{\atlasarr}[1]{\atlas{#1}_{,\subp{2}}}
\newcommand{\atlasV}[2]{\atlas{#1}(#2)}
\newcommand{\atlasid}[1]{\oper{id}_{\parsecat{#1}}}
\newcommand{\atlasdom}[1]{\oper{dom}\atlas{#1}}
\newcommand{\atlascod}[1]{\oper{cod}\atlas{#1}}

\newcommand{\smetric}[2]{\mathsf{E}\pa{g,\,#1,\,#2}}
\newcommand{\ssmetric}[1]{\mathsf{E}\pa{g,\,#1}}
\newcommand{\lmetric}[2]{\ppv{g}{\mathsf{L}}\pa{#1,\,#2}}

\newcommand{\ofunct}[1]{\functscr{F}_{#1}}
\newcommand{\ofunctobj}[1]{\functscr{F}_{#1,\subp{1}}}
\newcommand{\ofunctarr}[1]{\functscr{F}_{#1,\subp{2}}}
\newcommand{\ofunctV}[2]{\ofunct{#1}(#2)}
\newcommand{\ofunctid}[1]{\oper{id}_{#1}}
\newcommand{\ofunctdom}[1]{\oper{dom}\ofunct{#1}}
\newcommand{\ofunctcod}[1]{\oper{cod}\ofunct{#1}}

\newcommand{\umpclass}{\mathcal{F}}
\newcommand{\umpclassp}[1]{{\umpclass}_{\parsecat{#1}}}
\newcommand{\umpclasspc}[2]{\umpclassp{#1}(\parsecat{#2})}
\newcommand{\umpall}{\umpclassp{\forall}}
\newcommand{\umpallc}[1]{\umpall(\parsecat{#1})}
\newcommand{\umpconst}[1]{\umpall_{\parsecat{#1}}}
\newcommand{\umpconstc}[2]{\umpconst{#1}(\parsecat{#2})}
\newcommand{\umpcat}{\mathcal{C}}
\newcommand{\umpcatp}[1]{\umpcat_{\parsecat{#1}}}
\newcommand{\umpdomp}[1]{\oper{dom}\,\umpclassp{#1}}
\newcommand{\umpcodp}[1]{\oper{cod}\,\umpclassp{#1}}
\newcommand{\expab}[2]{\obk{#1}^{\obk{#2}}}
\newcommand{\expxab}[2]{\expab{#2}{#1}\times\obk{#1}}
\newcommand{\mapfab}[3]{{#1}:\,{#2}\longrightarrow {#3}}
\newcommand{\mapaob}[3]{{#1}\xrightarrow{\;{#2}\;}{#3}}

\newcommand{\homcat}{\parsecat{Hom}}
\newcommand{\quivercat}[1]{\parsecat{Quiv}(\parsecat{#1})}
\newcommand{\largepo}{\mathbf{\mathpzc{PO}}}
\newcommand{\largeeq}{\mathbf{\mathpzc{EQUIV}}}
\newcommand{\smalleq}{\mathbf{\mathpzc{Equiv}}}
\newcommand{\poset}{\mathbf{\mathpzc{Poset}}}
\newcommand{\largegrf}{\mathbf{\mathpzc{GRF}}}
\newcommand{\smallgrf}{\mathbf{\mathpzc{Grf}}}
\newcommand{\functcat}{\mathbf{\mathpzc{FUN}}}
\newcommand{\functcatcat}{\mathbf{\mathpzc{FUN}}_{\functscr{C}}}
\newcommand{\functcatfunct}{\mathbf{\mathpzc{FUN}}_{\functscr{F}}}
\newcommand{\smallfunctcat}{\mathbf{\mathpzc{Fun}}}
\newcommand{\smallfunctcatcat}{\smallfunctcat_{\functscr{C}}}
\newcommand{\smallfunctcatfunct}{\smallfunctcat_{\functscr{F}}}
\newcommand{\functcatp}[1]{\functcat_{\parsecat{#1}}}
\newcommand{\functcatcatp}[1]{\functcat_{\parsecat{#1},\functscr{C}}}
\newcommand{\functcatfunctp}[1]{\functcat_{\parsecat{#1},\functscr{F}}}
\newcommand{\wrpcat}[1]{{#1}_{\functscr{C}}}
\newcommand{\wrpfunct}[1]{{#1}_{\functscr{F}}}
\newcommand{\wrpnat}[1]{{#1}_{\functscr{N}}}
\newcommand{\largecat}{\mathbf{\mathpzc{CAT}}}
\newcommand{\largecatobj}{\wrpobj{\largecat}}
\newcommand{\largecatarr}{\wrparr{\largecat}}
\newcommand{\smallcat}{\parsecat{Cat}}
\newcommand{\smallcatobj}{\wrpobj{\smallcat}}
\newcommand{\smallcatarr}{\wrparr{\smallcat}}
\newcommand{\expcat}{\mathpzc{Exp}}
\newcommand{\setcat}{\mathpzc{Set}}
\newcommand{\smallsetcat}{\parsecat{set}}
\newcommand{\fincat}{\parsecat{Fin}}
\newcommand{\forgetful}[1]{|#1|}
\newcommand{\forgetgrf}[1]{\functscr{U}(#1)}
\newcommand{\rffunct}{\functscr{U}^{\Circlearrowright}}
\newcommand{\rforgetgrf}[1]{\rffunct(#1)}
\newcommand{\ffunct}{\funct{U}}
\newcommand{\subcat}[1]{\mathpzc{Sub}\parsecat{#1}}

\newcommand{\rquiver}[1]{\parsecat{Q}^{\Circlearrowright}(\mathsf{#1})}
\newcommand{\wquiver}[1]{\parsecat{Q}(\mathsf{#1})}
\newcommand{\nwquiver}[2]{\parsecat{Q}^{#1}(\mathsf{#2})}
\newcommand{\quiver}[1]{\Gamma(\mathsf{#1})}
\newcommand{\graph}[1]{\mathsf{#1}}
\newcommand{\graphv}[1]{\mathsf{#1}_{\mathbf{v}}}
\newcommand{\graphe}[1]{\mathsf{#1}_{\mathbf{e}}}
\newcommand{\graphfree}[1]{\parsecat{C}(\graph{#1})}
\newcommand{\graphclass}[1]{{\mathcal{F}}\,(\graph{#1})}

\newcommand{\evalxab}[2]{\mapaob{\expxab{#1}{#2}}{\oper{eval}}{\obk{#2}}}

\newcommand{\di}[1]{\boldsymbol #1}
\newcommand{\dI}{\boldsymbol \imath}

\newcommand{\freefunct}{{\raisebox{.1\baselineskip}{\ensuremath{\digamma}}}}
\newcommand{\freeset}[1]{\freefunct(#1)}
\newcommand{\pset}[1]{{\raisebox{.15\baselineskip}{\Large\ensuremath{\wp}}}(#1)}
\newcommand{\scard}[1]{\kappa(#1)}

\newcommand{\numchars}[1]{\noindent The string #1 has \StrLen{#1} characters. }

%\newcommand{\hmult}{\scalebox{.4}{\bigtriangledown}}
\newcommand{\cohopfmult}{\scalebox{.6}[1]{\ensuremath{\Delta}}}
\newcommand{\hopfmult}{\raisebox{\depth}{\scalebox{.6}[-1]{\ensuremath{\Delta}}}}

\newcommand{\abs}[1]{\left|#1\right|}
\newcommand{\spin}[1]{\mathbf{#1}}
\newcommand{\adjoint}[1]{\widetilde{#1}}
\newcommand{\uvector}[1]{\hat{\mathbf{#1}}}
\newcommand{\svector}[2]{\left[\begin{array}{r} #1 \\[5pt]
                                                #2 \end{array}\right]}
\newcommand{\tsvector}[2]{\left[\begin{array}{cc} #1 & #2 \end{array}\right]}
\newcommand{\spinor}[1]{\svector{#1_1}{#1_2}}
\newcommand{\lspinor}[2]{e^{#1}\spinor{#2}}
\newcommand{\aspinor}[1]{\svector{-#1^*_2}{#1^*_1}}
\newcommand{\alspinor}[2]{e^{-#1}\aspinor{#2}}

\newcommand{\pspinor}[2]
           {\svector{e^{-\iu#2/2}\cos(\frac{#1}{2})}
             {e^{\iu#2/2}\sin(\frac{#1}{2})}}
\newcommand{\qspinor}[3]{e^{\iu#3/2}\pspinor{#1}{#2}}
\newcommand{\espinor}[3]{\qspinor{#1}{#2}{i#3}}
\newcommand{\hspinor}[3]{e^{-#3/2}\pspinor{#1}{#2}}
\newcommand{\ahspinor}[3]{e^{#3/2}\apspinor{#1}{#2}}
\newcommand{\apspinor}[2]
           {\svector{-e^{-\iu#2/2}\sin(\frac{#1}{2})}
                    {e^{\iu#2/2}\cos(\frac{#1}{2})}}
\newcommand{\aqspinor}[3]{e^{-\iu#3/2}\apspinor{#1}{#2}}
\newcommand{\aespinor}[3]{\aqspinor{#1}{#2}{i#3}}
\newcommand{\iu}{\iota\mkern1mu}
\newcommand{\rpauli}[1]{\varsigma_#1\mkern2mu}
\newcommand{\pauli}[1]{\sigma_#1\mkern2mu}
\newcommand{\levicivita}[1]{\varepsilon_{#1}\mkern2mu}

\newcommand{\smatrix}[4]{\left[\begin{array}{rr}
                          #1 & #2 \\[3pt] #3 & #4\end{array}\right]}
\newcommand{\psu}{\smatrix{1}{0}{0}{1}}
\newcommand{\psx}{\smatrix{0}{1}{1}{0}}
\newcommand{\psy}{\smatrix{0}{-\iu}{\iu}{0}}
\newcommand{\psz}{\smatrix{1}{0}{0}{-1}}

\newcommand{\qmatrix}[2]
{\smatrix{1+\cos#2}{e^{\iu#1}\sin#2}{e^{-\iu#1}\sin#2}{1-\cos#2}}
\newcommand{\aqmatrix}[2]
{\smatrix{1-\cos#2}{-e^{\iu#1}\sin#2}{-e^{-\iu#1}\sin#2}{1+\cos#2}}
\newcommand{\dualform}{\raisebox{1pt}{\scalebox{.8}
  {\rotatebox[origin=c]{60}{$\leftrightharpoons$}}}}
\newcommand{\bigdual}{\raisebox{-3pt}{\scalebox{2.2}{$\diamond$}}}
\newcommand{\qs}[1]{\mathsf{#1}}
\newcommand{\qv}[1]{\mathbf{\mathsf{#1}}}
\newcommand{\ppv}[2]{{\,\raisebox{2pt}{\scalebox{.5}{(#1)\,}}}#2}
\newcommand{\qvl}[2]{\ppv{#1}{\mathbf{\mathsf{#2}}}}
\newcommand{\sv}[1]{\mathbf{#1}}
\newcommand{\uvl}[2]{\overset{#1}{\sv{#2}}}
\newcommand{\svl}[2]{\ppv{#1}{\sv{#2}}}
\newcommand{\ssl}[2]{\ppv{#1}{#2}}
\newcommand{\fv}[2]{#1_{\mathsf{#2}}}
\newcommand{\fvl}[3]{\fv{#1}{#2}_{,#3}}
\newcommand{\lint}[1]{\mathlarger{\int}_{#1}}
\newcommand{\olint}[1]{\mathlarger{\oint}_{\partial#1}}
\newcommand{\pa}[1]{\left(#1\right)}
\newcommand{\fa}[2]{#1\pa{#2}}
\newcommand{\MAN}[1]{\cat{#1}}
\newcommand{\MANobj}[1]{\ccobj{#1}}
\newcommand{\MANarr}[1]{\ccarr{#1}}
\newcommand{\MET}[1]{\cat{#1}}
\newcommand{\METobj}[1]{\ccobj{#1}}
\newcommand{\METarr}[1]{\ccarr{#1}}
\newcommand{\MU}[1]{\mathcal{#1}}
\newcommand{\mU}[1]{\mathsf{#1}}
\newcommand{\morph}[1]{\arr{#1}}
\newcommand{\morphdom}[1]{\arrdom{#1}}
\newcommand{\morphcod}[1]{\arrcod{#1}}

\makeatletter
\def\instring#1#2{TT\fi\begingroup
  \edef\x{\endgroup\noexpand\in@{#1}{#2}}\x\ifin@}
%
\def\isuppercase#1{%
  \ensuremath{%
  \instring{#1}{AÂBCÇDEFGĞHIİÎJKLMNOÖÔPRSŞTUÜÛVYZ}%
  }
}%
\makeatother

\newcommand{\checkcat}[1]{
\StrBefore[1]{#1}{/}[\Topcat]
\StrBehind[1]{#1}{/}[\Botcat]
\StrLen{\Topcat}[\toplen]
\StrLen{\Botcat}[\botlen]
\IfEq{\toplen}{0}{\mathbf{\mathpzc{#1}}}
{(\parseonecat{\Topcat}/\parseonecat{\Botcat})}
}

\newcommand{\UpperCats}{ABCDEFGHIJKLMNOPQRSTUVWXYZ}

\newcommand{\parsecat}[1]{%
    \StrLen{#1}[\slen]
    \ifcase\slen
      #1
    \or\parseonecat{#1}
    \else
      \checkcat{#1}
    \fi
}

\newcommand{\parseonecat}[1]{
  \IfInteger{#1}{\catN{#1}}
    { \IfSubStr{\UpperCats}{#1}{\pazocal{#1}}{\obk{#1}} }
}

%\newcommand{\catcaseset}[1]{
%\if\isuppercase{#1}\cat{#1}\else\obk{#1}\fi
%}

\def\CircleArrowleft{\ensuremath{%
  \reflectbox{\rotatebox[origin=c]{180}{$\circlearrowleft$}}}}
\def\CircleArrowright{\ensuremath{%
  \reflectbox{\rotatebox[origin=c]{180}{$\circlearrowright$}}}}
\def\Circlearrowleft{\ensuremath{%
  \rotatebox[origin=c]{100}{$\circlearrowleft$}}}
\def\Circlearrowright{\ensuremath{%
  \rotatebox[origin=c]{150}{$\circlearrowright$}}}

\setcounter{secnumdepth}{4}

\DeclareRobustCommand{\atled}{\text{\reflectbox{$\delta$}}}

\titleformat{\paragraph}{\normalfont\normalsize\bfseries}{\theparagraph}{1em}{}
\titlespacing*{\paragraph}{0pt}{3.25ex plus 1ex minus .2ex}{1.5ex plus .2ex}

\tikzset{
commutative diagrams/.cd,
%row sep=3cm,
%column sep=3cm,
%matrix scale/.style={/tikz/matrix xscale=3,/tikz/matrix yscale=3},
%matrix scale=2,
arrow style=tikz,
diagrams={>=latex}
}
\tikzset{res/.style={ellipse,draw,minimum height=0.1cm,minimum width=0.1cm}}

\begin{document}
\title{Quantum Spin Formulation}
\author         {Wolfgang Kraske}
\email        {wolfkden@gmail.com}
\homepage     {http://www.oviumzone.com}
\affiliation  {OVium Studies in Physics}
\date{\today}

\begin{abstract}
$\vardigamma\Varrho\varrho\snatural\natural\backepsilon\vareth\eth\Digamma\digamma\Varkappa\varkappa$
Spin manifests from the motion of a spherical coordinate system. Sans fixed
points, motion evinces a the continuous automorphisms of the surface of a
sphere consistent with a open maping of a three sphere. Anomolous
fixed points curiously render quantum observables denoted as spin.
A spinor coordinate system engenders a mathematical conceptualization of
spin derived from the Dirac equation to gauge theories familiar to
electrodynamics and Yang-Mills theory. 
\end{abstract}

\maketitle

\setlength{\parindent}{0em}
\setlength{\parskip}{0.5em}
\renewcommand{\baselinestretch}{1.0}

\section{Introduction}

% \textsuperscript

\section{Basic Theory}\label{sec:basictheory}

\begin{definition}{Sequent}\label{def:sequent},
  denoted with symbol, $\vdash$, relates an
  antecedent, $\obk{a}$, to the consequent, $\obk{c}$.
\begin{equation*}
\obk{a}\vdash\obk{c}
\end{equation*}
A cedent refers either an antecedent or consequent mutually
associated by a sequent
  \begin{equation*}
  \begin{array}{rll}
   Antecedent&-&\text{an assumption} \\[3pt]
   Consequent&-&\text{an assumption dependent on antecedent}
  \end{array}
  \end{equation*}
\end{definition}

\begin{equation*}
\spinor{z}\qquad z_1,z_2\in\mathbb{C},\qquad 1=z_1z_1^*+z_2z_2^*
\end{equation*}

\begin{equation*}
\begin{array}{lclrrrr}
z_1&=&\abs{z_1}e^{\iu\mu/2}e^{-\iu\phi/2}\,,&\qquad&
\abs{z_1}&=&\cos\left(\frac{\theta}{2}\right) \\[3pt]
z_2&=&\abs{z_2}e^{\iu\mu/2}e^{\;\iu\phi/2}\,,&\qquad&
\abs{z_2}&=&\sin\left(\frac{\theta}{2}\right)
\end{equation*}

\begin{equation*}
\iu=\sqrt{-1}\qquad zz^*=z^*z=\abs{z}\qquad z\in\mathbb{C}
\end{equation*}

\begin{equation*}
\spin{\Phi}=\spinor{z}\qquad\adjoint{\spin{\Phi}}=\aspinor{z}\qquad
-\spin{\Phi}=\adjoint{\adjoint{\Phi}}
\end{equation*}
\begin{equation*}
\spin{\Phi}^\dagger=\tsvector{z_1^*}{z_2^*}\qquad
\adjoint{\spin{\Phi}}^\dagger=\tsvector{-z_2}{z_1}
\end{equation*}

\begin{equation*}
\begin{array}{ccccc}
1&=&\spin{\Phi}^\dagger\spin{\Phi}&=&
\adjoint{\spin{\Phi}}^\dagger\adjoint{\spin{\Phi}} \\[5pt]
0&=&\adjoint{\spin{\Phi}}^\dagger\spin{\Phi}&=&
\spin{\Phi}^\dagger\adjoint{\spin{\Phi}}
\end{array}
\end{equation*}

\begin{equation*}
\begin{array}{ccccccc}
\rpauli{0}&=&\;\:\spin{\Phi}\spin{\Phi}^\dagger+
\adjoint{\spin{\Phi}}\adjoint{\spin{\Phi}}^\dagger&\qquad&
\rpauli{1}&=&\spin{\Phi}\adjoint{\spin{\Phi}}^\dagger+
\adjoint{\spin{\Phi}}\spin{\Phi}^\dagger \\[5pt]
\rpauli{2}&=&\iu\left(\spin{\Phi}\adjoint{\spin{\Phi}}^\dagger-
\adjoint{\spin{\Phi}}\spin{\Phi}^\dagger\right)&\qquad&
\rpauli{3}&=&\spin{\Phi}\spin{\Phi}^\dagger-
\adjoint{\spin{\Phi}}\adjoint{\spin{\Phi}}^\dagger
\end{array}
\end{equation*}

\begin{equation*}
\rpauli{0}=\rpauli{i}^2\,,\qquad\rpauli{i}=\rpauli{i}^\dagger\qquad
i\in\left\{ 0\cdots 3\right\}
\end{equation*}
\begin{equation*}
\rpauli{i}\rpauli{j}\rpauli{k}=\iu\levicivita{ijk}\rpauli{0},\qquad
i,j,k\in\left\{ 1\cdots 3\right\}
\end{equation*}

\begin{equation*}
0=\phi=\theta
\end{equation*}
\begin{equation*}
\levicivita{ijk}=\left\{\begin{array}{rrlll}
                           1&\text{even permutation}&
                                   \left(ijk\right)&=&\left(123\right)
\\[3pt]
                           -1&\text{odd  permutation}&
                                   \left(ijk\right)&=&\left(123\right)
                         \end{array}
\end{equation*}
\begin{equation*}
\rpauli{0}=\adjoint{\rpauli{0}}\,,\qquad
-\rpauli{i}=\adjoint{\rpauli{i}}\qquad i\in\left\{ 1\cdots 3\right\}
\end{equation*}

\begin{equation*}
0=\mu=\phi=\theta
\end{equation*}

\begin{equation*}
\spin{e}=\svector{1}{0}\qquad\adjoint{\spin{e}}=\svector{0}{1}
\end{equation*}

\begin{equation*}
\begin{array}{ccccccc}
\pauli{0}&=&\;\:\spin{e}\spin{e}^\dagger+
\adjoint{\spin{e}}\adjoint{\spin{e}}^\dagger&\qquad&
\pauli{1}&=&\spin{e}\adjoint{\spin{e}}^\dagger+
\adjoint{\spin{e}}\spin{e}^\dagger \\[5pt]
\pauli{2}&=&\iu\left(\spin{e}\adjoint{\spin{e}}^\dagger-
\adjoint{\spin{e}}\spin{e}^\dagger\right)&\qquad&
\pauli{3}&=&\spin{e}\spin{e}^\dagger-
\adjoint{\spin{e}}\adjoint{\spin{e}}^\dagger
\end{array}
\end{equation*}

\begin{equation*}
\begin{array}{ccccccc}
\pauli{0}&=&\psu&\qquad&\pauli{1}&=&\psx \\[9pt]
\pauli{2}&=&\psy&\qquad&\pauli{3}&=&\psz
\end{array}
\end{equation*}

\begin{equation*}
1=\pauli{0}
\end{equation*}
\begin{equation*}
\mathbf{i}=-\iu\pauli{1}\qquad\mathbf{j}=-\iu\pauli{2}\qquad
\mathbf{k}=-\iu\pauli{3}
\end{equation*}

W.L.O.G. eigenvectors
\begin{equation*}
\spin{\Phi}=\pspinor{\phi}{\theta}\qquad
\adjoint{\spin{\Phi}}=\apspinor{\phi}{\theta}
\end{equation*}

singular matrices
\begin{equation*}
\begin{array}{rcccl}
\spin{\Phi}\spin{\Phi}^\dagger&=&\qmatrix{\phi}{\theta}&=&
  \pauli{0}+\sigma\cdot\uvector{r} \\[11pt]
\adjoint{\spin{\Phi}}\adjoint{\spin{\Phi}}^\dagger&=&\aqmatrix{\phi}{\theta}
  &=&\pauli{0}-\sigma\cdot\uvector{r}
\end{array}
\end{equation*}

eigenvalues
\begin{equation*}
e^{-\frac{\mu}{2}}\qquad e^{\frac{\mu}{2}}\qquad\mu\in\mathbb{C}
\end{equation*}

Minkowski vector
\begin{equation*}
\qv{M}=\pauli{0}ct-\left(\sigma\cdot\uvector{r}\right)r\qquad
\mathbf{r}=\uvector{r}r
\end{equation*}

\begin{equation*}
\begin{array}{ccccccc}
\sigma\cdot\uvector{r}&=&
\pauli{1}\uvector{r}_1+\pauli{2}\uvector{r}_2+\pauli{3}\uvector{r}_3&\qquad&
\uvector{r}_1&=&\cos\phi_r\sin\theta_r \\[9pt]
\uvector{r}_2&=&\sin\phi_r\sin\theta_r&\qquad&
\uvector{r}_3&=&\cos\theta_r
\end{array}
\end{equation*}

Boost
\begin{equation*}
\qv{U}\qv{M}\qv{U}^\dagger
\end{equation*}

\begin{equation*}
\begin{array}{l}
\qv{U}=e^{-\frac{\mu}{2}}\spin{\Phi}\spin{\Phi}^\dagger+
e^{\frac{\mu}{2}}\adjoint{\spin{\Phi}}\adjoint{\spin{\Phi}}^\dagger \\[3pt]
\qquad =\pauli{0}\cosh\frac{\mu}{2}- \\[3pt]
\qquad\qquad \left(
\left(\pauli{1}\cos\phi+\pauli{2}\sin\phi\right)\sin\theta+
\pauli{3}\cos\theta\right)\sinh\frac{\mu}{2} \\[3pt]
\qquad =\pauli{0}\cosh\frac{\mu}{2}-
\left(\sigma\cdot\uvector{v}\right)\sinh\frac{\mu}{2} \\[3pt]
\qquad=\exp\left(-\left(\sigma.\uvector{v}\right)\frac{\mu}{2}\right)
\end{array}
\end{equation*}

\begin{equation*}
\qv{U}\adjoint{\qv{U}}=
\adjoint{\qv{U}}\qv{U}=\det{\qv{U}}\pauli{0}=\pauli{0}
\end{equation*}

\begin{equation*}
0=\mu-\mu*,\qquad\qv{U}=\qv{U}^\dagger
\end{equation*}

\begin{equation*}
\begin{array}{ccccccc}
\cosh\mu&=&\gamma&\qquad&\sinh\mu&=&\gamma\tanh\mu \\[9pt]
\tanh\mu&=&\frac{v}{c}&\qquad&\mathbf{v}&=&\uvector{v}v
\end{array}
\end{equation*}

\begin{equation*}
\begin{array}{ccccccc}
\sigma\cdot\uvector{v}&=&
\pauli{1}\uvector{v}_1+\pauli{2}\uvector{v}_2+\pauli{3}\uvector{v}_3&\qquad&
\uvector{v}_1&=&\cos\phi\sin\theta \\[5pt]
\uvector{v}_2&=&\sin\phi\sin\theta&\qquad&
\uvector{v}_3&=&\cos\theta
\end{array}
\end{equation*}

Minkowski vector
\begin{equation*}
\begin{array}{rcl}
\qv{M}&=&\pauli{0}ct-
r\sigma\cdot\left(\uvector{r}_\parallel\cos\delta+
  \uvector{r}_\perp\sin\delta\right) \\[3pt]
&=&\left(\pauli{0}ct-
\left(\sigma\cdot\uvector{r}_\parallel\right)r\cos\delta\right)-
  \left(\sigma\cdot\uvector{r}_\perp\right)r\sin\delta \\[3pt]
\uvector{r}&=&\uvector{r}_\parallel\cos\delta+\uvector{r}_\perp\sin\delta \\[3pt]
\cos\delta&=&\uvector{v}\cdot\uvector{r} \\[3pt]
\uvector{r}_\parallel\cos\delta&=&
             \uvector{v}\left(\uvector{v}\cdot\uvector{r}\right),\quad
                                   \uvector{r}_\parallel=\uvector{v} \\[3pt]
\uvector{r}_\perp\sin\delta&=&\uvector{r}-\uvector{v}\cos\delta
\end{array}
\end{equation*}

\begin{equation*}
\qv{M}^\prime=\qv{U}\qv{M}\qv{U}^\dagger,
\quad\qv{U}^2=\exp\left(-\left(\sigma.\uvector{v}\right)\mu\right)
\end{equation*}

\begin{equation*}
\begin{array}{rcl}
\qv{M}^\prime&=&
\qv{U}^2\,\left(\pauli{0}ct-
\left(\sigma\cdot\uvector{r}_\parallel\right)r\cos\delta\right) \\[3pt]
&&\qquad-\left(\sigma\cdot\uvector{r}_\perp\right)r\sin\delta \\[3pt]
&=&\pauli{0}ct^\prime-\left(\sigma.\uvector{r}^\prime\right)r^\prime \\[3pt]
\qv{U}\left(\sigma\cdot\uvector{r}_\perp\right)\qv{U}^\dagger&=&
\left(\sigma\cdot\uvector{r}_\perp\right)\qv{U}^{-1}\qv{U}=
\left(\sigma\cdot\uvector{r}_\perp\right) \\[3pt]
\uvector{r}^\prime&=&
    \uvector{r}^\prime_\parallel\cos\delta^\prime+
    \uvector{r}^\prime_\perp\sin\delta^\prime, 
    \quad\uvector{r}^\prime_\parallel=\uvector{v},
    \quad\uvector{r}^\prime_\perp=\uvector{r}_\perp \\[3pt]
ct^\prime&=&ct\cosh\mu+r\cos\delta\sinh\mu \\[3pt]
r^\prime\cos\delta^\prime&=&ct\sinh\mu+r\cos\delta\cosh\mu \\[3pt]
r^\prime\sin\delta^\prime&=&r\sin\delta \\[3pt]
\left(\frac{r^\prime}{r}\right)^2&=&
              \left(\frac{ct}{r}\sinh\mu+\cos\delta\cosh\mu\right)^2+
              \sin^2\delta
\end{array}
\end{equation*}

Quaternion Rotation
\begin{equation*}
0=\mu-\mu*
\end{equation*}

\begin{equation*}
\qv{U}\qv{M}\qv{U}^\dagger
\end{equation*}

\begin{equation*}
\begin{array}{l}
\qv{U}=e^{-i\frac{\mu}{2}}\spin{\Phi}\spin{\Phi}^\dagger+
e^{i\frac{\mu}{2}}\adjoint{\spin{\Phi}}\adjoint{\spin{\Phi}}^\dagger \\[3pt]
\qquad =\pauli{0}\cos\frac{\mu}{2}- \\[3pt]
\qquad\qquad i\left(
\left(\pauli{1}\cos\phi+\pauli{2}\sin\phi\right)\sin\theta+
\pauli{3}\cos\theta\right)\sin\frac{\mu}{2} \\[3pt]
\qquad =\pauli{0}\cos\frac{\mu}{2}-
i\left(\sigma\cdot\uvector{v}\right)\sin\frac{\mu}{2} \\[3pt]
\qquad =\exp\left(-i\left(\sigma.\uvector{v}\right)\frac{\mu}{2}\right)
\end{array}
\end{equation*}

\begin{equation*}
\qv{U}\adjoint{\qv{U}}=
\adjoint{\qv{U}}\qv{U}=
\qv{U}\qv{U}^\dagger=\qv{U}^\dagger\qv{U}=
\det{\qv{U}}\pauli{0}=\pauli{0}
\end{equation*}

\begin{equation*}
\qv{M}^\prime=\qv{U}\qv{M}\qv{U}^\dagger,
\quad\qv{U}^\dagger^2=\qv{U}^{-2}=
\exp\left(i\left(\sigma.\uvector{v}\right)\mu\right)
\end{equation*}

\begin{equation*}
\begin{array}{rcl}
\qv{M}^\prime&=&
\left(\pauli{0}ct-
\left(\sigma\cdot\uvector{r}_\parallel\right)r\cos\delta\right) \\[3pt]
&&\qquad-\,\qv{U}^2
   \left(\sigma\cdot\uvector{r}_\perp\right)r\sin\delta \\[3pt]
&=&\pauli{0}ct-\left(\sigma.\uvector{r}^\prime\right)r \\[3pt]
\uvector{r}^\prime&=&
    \uvector{r}^\prime_\parallel\cos\delta^\prime+
    \uvector{r}^\prime_\perp\sin\delta^\prime,\quad r^\prime=r,
    \quad\uvector{r}^\prime_\parallel=\uvector{v} \\[3pt]
\left(\sigma\cdot\uvector{r}^\prime_\perp\right)&=&
               \left(\sigma\cdot\uvector{r}_\perp\right)\qv{U}^{-2}=
               \qv{U}^2\left(\sigma\cdot\uvector{r}_\perp\right) \\[3pt]
ct^\prime&=&ct,\quad\cos\delta^\prime=\cos\delta,\quad
                                       \sin\delta^\prime=\sin\delta
\end{array}
\end{equation*}

Construction and Analysis

Mathematical dialectic interposes construct versus reductive analysis
in the study of dynamical systems.
Construction of spaces from elementary properties
provides a bottom up approach whereas reductive analysis
of dynmical expression provides a top down approach.
Intentional construction of spaces from elementary principles
establishes explanatory models for dynamical systems
whereas analysis of symmetries reduce general dynamical
systems to regular expression.
Manifold denotes a space beholden to the literal behavior
of general dynamical systems, in complement phase space
denotes a regular space for idealized dynamical behavior.
Quantum mechanics exibits a mainifold of wave behavior
shaped by dynamics in a classical phase space.
General dynamical systems yield representation to
manifolds regularized by local tangent space behavior.
Phase space embraces a field
with Hausdorff seperability, local convexity,
and norm, all properties crucial for the
formation of vector spaces and differential local behavior.
Banach spaces meet these requisite criterion with additional
metric properties that support the agregate requirements of integration.
The real field conforms to Banach space requirements.
A vector field with metric develops in
construction from a base field, $\mathbb{K}$,
with  the coproduct operation, $\oplus$:

\begin{equation*}
\begin{array}{c}
\mathbb{K}^n=\bigoplus^n \mathbb{K} \\[7pt]
\text{where $n$ is cardinality of dimension}
\end{array}
\end{equation*}

Classical physical representations require finite cardinality, $n$,
whereas quantum mechanics requires the countable cardinality, $n=\aleph_0$
of Hilbert space.
Metric spaces with infinite cardinality define, 
$\mathsf{L}^q_r$ spaces, where $q+r=qr$, integers for normalization;
Hilbert space dictates, $q=r=2$, to imply square normed function spaces,
over a a Banach space.

Other than integration properties, phase space dynamics
admit expression with finite dimensional
constructions via coordinate bases having a metric, $g^{\mu\nu}$.
Obvious examples hail from familiar
Euclidean spatial and Minkowski pseudo-Euclidiean
temporal-spatial representation.

Fundamental forms in Euclidian and Pseudo-Euclidean spaces

Construction begins with a metric space, $\mathsf{E}$,
characterized by a metric, $g$, with an implied
quadratic fundamental form, $\Phi$.
The metric, $g$, relates dual vector spaces, $\lmetric{V}{V^*}$,
where, $V$ and $V^*$, denote
covariant and contravariant spaces respectively.
Covariant and contravariant spaces may represent vector
spaces of conceptually different applicatiopns however
sharing the same dimensionality and Banach properties, i.e. $\mathbb{K}^n$.
For instance vector space, $Q$, may be accompanied by a
symplectic dual, $P$, for to join
temporal-spatial with dual energy-momentum spaces, $\smetric{Q}{P}$.
The metric, $g$, is a special invertable linear map,
hence the dual metric maps the inverse form,
$\lmetric{V^*}{V}$.
The dual spaces are each $n$-dimensional vector spaces, $\mathbb{K}^n$,
over a separable abelian field,
$\mathbb{K}$, admiting finite measure over compact intervals.
The familiar real line, $\mathbb{R}$, resonates a convenient
example of $\mathbb{K}$.

Arbitrary covariant vectors of, $V$, map to distinct contravariant
vectors of $V^*$, via metric transform implied by the
Einstein summation convention over dimension indicies,
$\mu,\,\nu\in [n]=\{1\cdots n\}$.
The contravariant representation of $x$ is $x^\mu$
and the covariant representation is, $x_\mu$, denote
the metric transform from covariant to contravariant,
$\lmetric{V}{V^*}$:, as, $g^{\mu\nu}$:

\begin{equation*}
\begin{array}{rcl}
x^\mu$=$g^{\mu\nu} x_\nu,\quad x^\nu\in V^*\;\text{and}\;x_\mu\in V
\end{array}
\end{equation*}

Alternately contravariant vectors map distinct covariant vectors,
denote metric of $\lmetric{V^*}{V}$:, as, $g_{\mu\nu}$:

\begin{equation*}
\begin{array}{rcl}
x_\mu$=$g_{\mu\nu} x^\nu,\quad x^\mu\in V^*\;\text{and}\;x_\nu\in V
\end{array}
\end{equation*}

In accordance, metric behavior associates dual vector spaces.
The dimension $n$ designates the integral range of coordinates in the
dual vector spaces. The space $\mathbb{K}^n$ represents
each dual vector space designated as $V$ and $V^*$. Covariant and
contravariant label the behavior of dual space vectors
relative to the metric, $g$.

Abelian addition and component multiplication of coordinates introduce the
vector concept in the corresponding $n$-dimensional affine space.

The fundamental form defines a measure over aggregate vector components,
$x_\mu$, in a field, $\mathbb{K}$, as:

\begin{equation*}
\begin{array}{rcl}
\Phi\pa{x}$=$g^{\mu\nu}x_\mu x_\nu=x^\nu x_\nu=x_\mu x^\mu \\[7pt]
\Phi\pa{x}\in\mathbb{K}\quad\text{for all}\;x_\mu\in\mathbb{K}^n \\[7pt] 
\end{array}
\end{equation*}

Employing a traditional dot operator, $\cdot$, the product shows:
$x_\mu\cdot x^\mu=x_\mu x^\mu$, Operator notation is dispensed with
unless necessary for detail.

Sylvester's law of inertia assures the invariance of the fundamental form
and the metric trace through any non-singular coordinate transformation.
A reduced expression of the fundamental form with a symmetric metric:

\begin{equation*}
\begin{array}{rcl}
\Phi\pa{x}$=$g^{\mu\mu} x_\mu x_\mu&=&g_{\mu\mu} x^\mu x^\mu
\end{array}
\end{equation*}

The trace of the squared metric tensor reveals the dimension of vectors.
\begin{equation*}
\begin{array}{rcl}
n&=&\tr\pa{g^{\mu\nu}g_{\mu\nu}}=\tr\pa{g_{\mu\nu}g^{\mu\nu}}
\end{array}
\end{equation*}

A weighted summation of fundamental forms renders a
scalar product between vectors in the
corresponding $n$-dimensional affine space:

\begin{equation*}
\begin{array}{rcl}
x_\mu y^\mu = x^\mu y_\mu $=$ \frac{1}{4}\pa{\Phi\pa{x+y}-\Phi\pa{x-y}}
\end{array}
\end{equation*}

The trace of the metric tensor exposes the hyperbolic nature of a
pseudo-Euclidean space:

\begin{equation*}
\begin{array}{rcl}
\frac{1}{2}\pa{n+\tr\pa{g_{\mu\nu}}}&=&\frac{1}{2}\pa{n+\tr\pa{g^{\mu\nu}}}
\end{array}
\end{equation*}

A Euclidean metric space shows no hyperbolic trace.
The familiar Minkowski space demonstrates a hyperbolic trace of $1$.

The fundamental form measures the
length of a vector. Isotropic vectors have length zero.
An anisotropic form dictates an Euclidian space with
zero as a unique isotropic vector.
An isotropic form dictates a pseudo-euclidean space with a non-trivial
collection of non-zero isotropic vectors.
Time-like vectors have a positive length and space-like vectors have a
negative length; light-like vectors constitute isotropic manifolds
that partition a pseudo-Euclidean space. 

Covariant and contravariant vectors form dual spaces. In particular the
scalar product with a contravariant vector space, $V^*$, linearly maps
a covariant vector space, $V$ into an abelian field $\mathbb{K}$. 

\begin{equation*}
\begin{array}{rcl}
x^\mu:V\rightarrow\mathbb{K}\quad\text{for all}\;x^\mu\in V^*
\end{array}
\end{equation*}

Alternately,
\begin{equation*}
\begin{array}{rcl}
x_\mu:V^*\rightarrow\mathbb{K}\quad\text{for all}\;x_\mu\in V
\end{array}
\end{equation*}

Dual is a binary operation on a vector field, $V$.

\begin{equation*}
V=V^{**}
\end{equation*}

A general basis, $\{\svl{i}{a}\}_{i\in[n]}$, of the space, $V$,
complements the dual basis, $\svl{j}{\alpha}_{j\in[n]}$, of $V^*$.

\begin{equation*}
\svl{j}{\alpha}^\mu \svl{i}{a}_\mu=\delta^j_i
\end{equation*}

A contravariant transform, $\gamma^\nu_\mu$, from basis,
$\alpha^\mu$, to basis, $\alpha^{\prime\nu}$, expresses as follows:

\begin{equation*}
\alpha^{\prime\nu}=\gamma^\nu_\mu\alpha^\mu
\end{equation*}

The inverse transform, $\pa{\gamma^{-1}}_\nu^\mu$, similarly transforms
the covariant basis, $a_\mu$, to covariant basis,
$a^\prime_\nu$, expresses as follows:

\begin{equation*}
a^\prime_\nu=\pa{\gamma^{-1}}_\nu^\mu a_\mu
\end{equation*}

The idea of basis extends to multilinear forms.
A multilinear basis constructs from a specialized product
of basis vectors spanning a space, $V=\mathbb{K}^n$.
The multivector employs wedge product, a specialization of
the product operation, $\otimes$.
Given a basis for, $V$, of $n$ elements, $\{e_i\}_{i\in[n]}$,
the order of basis indices, ${i\in[n]}$, imparts a sign
on multilinear basis elements where basis elements
partition into $p$-forms.
A $p$-form basis element, $\svl{p}{T}_\gamma$,
represents $p$ distinct basis elements
ordered by acending indices as:

\begin{equation*}
\svl{p}{T}_\gamma=
\langle e_{i_1}\cdots e_{i_p}\rangle,\quad i_1<\cdots<i_p\in[n]
\end{equation*}

Note that $0\leq p\leq n$ and transposition of any two basis
elements proves the $p$-form antilinear:

\begin{equation*}
\begin{array}{c}
\langle e_{i_1}\cdots e_{i_k}\cdots e_{i_l}\cdots e_{i_p}\rangle= \\[5pt]
\qquad\qquad-\langle e_{i_1}\cdots e_{i_l} \cdots e_{i_k}\cdots e_{i_p}\rangle
\end{equation*}

The $0$-form consists of one basis element, $1\in\mathbb{K}$.
By induction each $p$-form basis consists of $\pa{\mqty{n \\ p}}$ elements
due to the distinct sets of $p$ basis vectors.

The $p$-form basis extends to represent general p-linear forms,
$\svl{p}{\varphi}\in{\bigwedge}^p V$:

\begin{equation*}
\begin{array}{rcl}
\svl{p}{\varphi}&\in&\bigwedge^p V,\quad 0\leq p\leq n \\[5pt]
\svl{p}{\varphi}&=&\svl{p}{\varphi}_\gamma\,
           \svl{p}{T}_\gamma,\quad\svl{p}{\varphi}_\gamma\in\mathbb{K} \\[5pt]
\gamma&\in&\left[\pa{\mqty{n \\ p}}\right]
\end{array}
\end{equation*}

Multiforms in a wider multilinear space show as:

\begin{equation*}
\begin{array}{rcl}
\varphi&\in&\bigwedge V,\quad \\[5pt]
\varphi&=&\sum_{p\in[n]} \svl{p}{\varphi},\quad
           \svl{p}{\varphi}\in\bigwedge^p V\\[5pt]
\end{array}
\end{equation*}

Multilinear alternating forms principally represent
tangent space topology on manifolds.

A dual space expresses as a multilinear algebra with \\
$p=1$ demonstrates an intuitive example:

\begin{equation*}
V = {\bigwedge}^{\!\! 1} V
\end{equation*}

Trivially the $0$-form, is a scalar field: $\mathbb{K}=\bigwedge^0 V$.

Combinatorially a $p$-form $\varphi\in\bigwedge^p V$ parametrizes
$\pa{\mqty{n \\ p}}$ components of the $p$ independent vectors.
The ordering of multilinear vectors
for each component comforms to to a sign standard, $\mathfrak{g}$,
familiar to transpositions of the levi-civa matrix, $\varepsilon_{[n]}$.

A monoidal wedge product acts as a binary operation on
$p$-linear forms to construct larger forms:

\begin{equation*}
\begin{array}{rcl}
  \bigwedge^{p+q} V&=&\pa{\bigwedge^p V} \wedge \pa{\bigwedge^q V} \\[7pt]
  \varphi\wedge\psi&=&\pa{-1}^{p+q}\psi\wedge\varphi,
               \quad\varphi\in\bigwedge^p V\;\text{and}\;\psi\in\bigwedge^q V
\end{array}
\end{equation*}

If $p+q$ is odd then $\varphi\wedge\psi$ anticommutes.

Again a convention, $\mathfrak{g}$, for multilinear vector ordering for
each component of a form $\varphi\in\bigwedge^p V$ establishes
a regularity for the application of a wedge product. The convention,
$\mathfrak{g}$, denotes the index set of the algebra.

Vectors from, $V$, combine to construct a non-trivial $p$-form basis:

\begin{equation*}
\dim {\bigwedge}^p V = \pa{\mqty{n \\ p}}
\end{equation*}

Multilinear components distinctly express a $p$-form of
$\varphi\in\bigwedge^p V$:

\begin{equation*}
\varphi= \varphi_\mu \svl{p}{T}^\mu\;\;\text{where}\;\varphi_\mu\in\mathbb{K}
\;\text{and}\;\svl{p}{T}^\mu\in\mathfrak{g}^p\subset{\bigwedge}^p V
\end{equation*}

Each basis element, $\svl{p}{T}^\mu$, is an ordered
multilinear set of $p$ vectors
from the basis of $V$. For general application, the commutator of $p$ and
$q$ forms establish a wedge product characteristic for multilinear forms.

\begin{equation*}
\begin{array}{rcl}
  \comm{\varphi}{\psi}&=&\varphi_\mu\psi_\nu
                          \comm{\svl{p}{T}^\mu}{\svl{q}{T}^\nu} \\[7pt]
  \comm{\svl{p}{T}^\mu}{\svl{q}{T}^\nu}&=&
        \ssl{pq}{f}}_\gamma\svl{pq}{T}^\gamma,\quad
  \ssl{pq}{f}_\gamma\;\;\text{structure constant}\\[7pt]
  \comm{A}{B}&=&AB-BA,\quad\text{commutator operation}
\end{array}
\end{equation*}


The complete multilinear vector dimensionality formed over the basis vectors
of $V$:

\begin{equation*}
2^n=\dim \bigoplus_{p=0}^n {\bigwedge}^p\, V
\end{equation*}



The complement relation of binary coefficients,

\begin{equation*}
\pa{\mqty{n \\ \pa{n-p}}}=\pa{\mqty{n \\ p}}
\end{equation*},

upholds the Hodge star transform, $\pa{}^\Ydown$, between
complimentary multilinear spaces of $$.

\begin{equation*}
\begin{array}{rcl}
  \varphi_\mu\wedge\psi_\mu&\in&\bigwedge^n V \\[7pt]
  \varphi_\mu\in\bigwedge^p V\;&\text{and}&\;
                \psi_\mu\in\bigwedge^{\pa{n-p}} V =\pa{\bigwedge^p V}^\Yup
\end{array}
\end{equation*}

Hodge dual acts on mulilinear basis elements in a straightforward method:

\begin{equation*}
\begin{array}{rcl}
\svl{p}{T}^\gamma^\Yup=
\langle e^{i_1}\cdots e^{i_p}\rangle^\Yup&=&
\langle e^{i_{p+1}}\cdots e^{i_n}\rangle=\svl{n-p}{T}^\gamma \\[7pt]
\varepsilon_{i_1\cdots i_n}&=&\varepsilon_{[n]}
\end{array}
\end{equation*}

The wedge product implies an algebra for an outer derivative,
a contravariant multi-form basis implies the inner derivative.
Denote a multiform basis for the inner derivative as:

\begin{equation*}
\svl{p}{T}^\gamma=
\langle e^{i_1}\cdots e^{i_p}\rangle,\quad i^1<\cdots<i_p\in[n]
\end{equation*}

Note that $0\leq p\leq n$ and transposition of any two basis
elements proves the $p$-form antilinear:

\begin{equation*}
\begin{array}{c}
\langle e^{i_1}\cdots e^{i_k}\cdots e^{i_l}\cdots e^{i_p}\rangle= \\[5pt]
\qquad\qquad-\langle e^{i_1}\cdots e^{i_l} \cdots e^{i_k}\cdots e^{i_p}\rangle
\end{equation*}

The symbol, $\diamond$, now denotes a product for dual multi-linear
forms. The contravariant multilinear forms perform homogeneously as
the did the covariant multilinear form. Heterogeneous products act
to reduce the order of forms in product.

\begin{equation*}
\begin{array}{rcl}
\langle e^{i}\rangle\diamond\langle e_{j}\rangle&=&\hspace{8pt}
            \langle e_{j}\rangle\diamond\langle e^{i}\rangle=\delta^i_j \\[5pt]
\langle e^{i}\rangle\diamond\langle e^{j}\rangle&=&
                   -\langle e^{j}\rangle\diamond\langle e^{i}\rangle \\[5pt]
\langle e_{i}\rangle\diamond\langle e_{j}\rangle&=&
                   -\langle e_{j}\rangle\diamond\langle e_{i}\rangle \\[5pt]
\langle e^{i}\rangle\diamond\langle e^{i}\rangle&=&\hspace{8pt}
                    \langle e_{i}\rangle\diamond\langle e_{i}\rangle=0
\end{array}
\end{equation*}

As shorthand vectors are represented in covariant, $x_{\mu}$, and
contravariant notation, $x^\mu$, with underlying dual
multi-linear basis elements understood.

The symbol, $\dualform$, denotes binary transformation
acting as a multilinear metric transform of
covariant to contravariant basis vectors and vis versa.

A quaternion product of two vectors renders as follows:

\begin{equation*}
\begin{array}{rcl}
x_\mu y^\mu&=&x\pa{1+\dualform}\diamond y \\[5pt]
          &=&x\cdot y+x\wedge y
\end{array}
\end{equation*}

A two dimensional space with complex metric renders the product
for complex numbers.

The Pauli spin form is similar with an imaginary factor associated
with the spatial wedge product. The Pauli product is fully detailed
in the previous section.

\begin{equation*}
\begin{array}{rcl}
x_\mu y^\mu&=&x_\mu\pa{1+\iu\dualform}\diamond x^\mu
\end{array}
\end{equation*}

Dual product, $\diamond$, acts as an interior transform of
covariant $p$-forms to
$\pa{p-1}$-forms or contravariant $p$-forms to $\pa{p-1}$-forms.

\begin{equation*}
\begin{array}{c}
\mapfab{x^\mu}{{\bigwedge}^p V}{{\bigwedge}^{p-1} V},
  \quad x^\mu\in{\bigwedge}^1 V^* \\[7pt]
\mapfab{x_\mu}{{\bigwedge}^p V^*}{{\bigwedge}^{p-1} V^*},
  \quad x_\mu\in{\bigwedge}^1 V \\[7pt]
\end{array}
\end{equation*}

Likewise, dual product, $\diamond$, acts as an exterior transform of
covariant $p$-forms to
$\pa{p+1}$-forms or contravariant $p$-forms to $\pa{p-1}$-forms.

\begin{equation*}
\begin{array}{c}
\mapfab{x_\mu}{{\bigwedge}^p V}{{\bigwedge}^{p+1} V},
  \quad x_\mu\in{\bigwedge}^1 V \\[7pt]
\mapfab{x^\mu}{{\bigwedge}^p V^*}{{\bigwedge}^{p+1} V^*},
  \quad x^\mu\in{\bigwedge}^1 V^* \\[7pt]
\end{array}
\end{equation*}

More details develop in the next sections with the
dual operations of esexterior derivative
and partial derivative. The eventual Lie algebra develops later.

Manifolds: Topological Category Theory

The concept of a mathematical manifold engenders the study of the
current flow of charges in a general space.
The dynamics of discrete particles assumes a special case.
The manifold concept embraces the idea of connected smoothly varying surfaces
embedded immersed within a metric space.
Manifolds embody the qualitative nature of dynamics whereas
metric spaces dictate quantitative measure, differentiation and
the linear coordinate specification of tangent spaces.
Category theory proffers a mathematical
framework to comparatively analyze and construct manifolds from metric spaces.
Riemann surfaces and hyperbolic surfaces in Minkowski space
epitomize historical images of a manifold.
Riemann surfaces idealized as two dimensional sheets
that form closed spheres and tori with one or more loops
demonstrate sufaces that map only with a plurality of local patches into a
two dimensional Euclidean or complex space.
Mapping constraints between manifolds and linear metric spaces
impose homotopy characteristics.

The intention herin concerns manifolds consistent with
Noether's theorem of abstract charges and flows.
Manifolds characterized by locally convex open tangent spaces
dictate the topology of interest.
Continuity of differentiation on the manifold qualify local convexity
and requirements of differentiable smoothness.
The interstitial dynamics of charge currents on smooth manifolds
translates locally with the conformity of a metric phase space.
Such mappings imply the duality of smooth manifolds with
the conformity of Euclidean and pseudo-Euclidean metric spaces.
The mappings are bijective or isomorphic in category theory,
hence the dynamic current of charges throughout a manifold
behave interstitially with piecewise dynamics in a metric space.
The mappings are diffeomorphisms, hence continuous and countably differentiable.
Diffeomorphisms interrelate dynamical frames between a local tangent space
with the regular behavior of Euclidean and pseudo-Euclidean metric spaces.
The local dynamical behavior in a tangent space of a smooth manifold requires
a countable number of locally convex neighbourhoods and diffeomorphisms
to a metric space. A collection of diffeomorphisms with associated convex
domains and codomains constitute an atlas.
The simplest manifolds require a finite atlas.
The surface of a $2$-sphere requires an minimal atlas of 2 diffeomorphisms.
The union of domains of the atlas defines the smooth manifold
whereas the union of codomains embeds within the metric space.
Metric space properties of
convexity, continuity and accessibiliy from a singular region
dictate dynamical behavior in the manifold.

The semantics of category theory establishes a foundation to construct
and analyze the tangent spaces of manifolds.
A collections of objects and morphisms between
objects define a category.
Special categories, called functors, consist exclusively
of morphisms that map a domain category to a codomain category.
Category objects generally exist as distinguishable entities.
The category approach begins with of definition of
general topological sets, not necessarily manifolds.
Topological categories consist of objects that are sets
and morphisms decribing the topological relation between objects.
Topological morhisms act as functions that map set objects.
The aggregate of objects in a topological category cover a 
maximal set, $X$, henceforth denote such a category with
a script font, $\cat{X}$.
A topological category morphism
maps an association between an underlying object pair to define
homological characteristics.
Detailed homology extends beyond introduction to the rhelm
of algebraic topology.
Development progresses further with the comparative study of
target and metric topological categories.
A target category, $\cat{M}$, denotes a topological structure
to analyze or construct whereas a metric category, $\cat{E}$, defines
a metric space or a patchwork of metric spaces
with a comparatively regular structure and behavior.
Categories and functors aggregate into classes
distingued by shared properties.
A topological class, $\umpclass_{C^p}$, consist
of special functors named atlases, $\atlas{F}\in\umpclass_{C^p}$,
where $p$ indicates an order of differentiability.
The atlas transforms a domain target category into a codomain metric category.
The morphisms of an atlas are diffeomorhisms,
uniformly characterized as continuous $p^{th}$ order differentiable functions,
where $p\in\mathbb{N}$, the natural numbers;
$p=0$ denotes a class of continuous morhphisms that are not differentiable.
Each atlas, $\atlas{F}$, maps the target category, $\MAN{M}$,
to the metric category, $\MET{E}$.

\begin{equation*}
\begin{array}{c}
\mapaob{\MAN{M}}{\atlas{F}}{\MET{E}} \\[7pt]
\MAN{M}=\atlasdom{F},\quad\cat{E}=\atlascod{F},
  \quad\atlas{F}\in\umpclass_{C^p} \\[5pt]
\morphdom{m}\in\MANobj{M},\quad\morphcod{m}\in\METobj{E},
  \quad\morph{m}\in\atlasobj{F} \\[5pt]
\end{array}
\end{equation*}

The properties of category and functor classes transform through
universal mapping properties. Special identity and dual operations
instantiate the distinguishablity and reversability properties of
classes.
The idempotent identity operation implies the distinguishability of category
entities, extant from proper classes.
The binary dual operation implies that morphisms are reversable.
The dual of a diffeomorphisms is an exact inverse.
Denote the dual UMP operation with the $*$ superscript.

\begin{equation*}
\begin{array}{c}
\mapaob{\MET{E}^*}{\atlas{F}^*}{\MAN{M}^*} \\[7pt]
\MET{E}^*=\atlasdom{F}^*,\quad\MAN{M}^*=\atlascod{F}^*,
  \quad\atlas{F}^*\in\umpclass_{C^p} \\[5pt]
\morphdom{m}^{-1}\in\METobj{E},\quad\morphcod{m}^{-1}\in\MANobj{M},
  \quad\arr{m}^{-1}\in\atlasobj{F}^* \\[5pt]
\MANobj{M}^*=\MANobj{M}\quad\text{and}\quad\METobj{E}^*=\METobj{E}
\end{array}
\end{equation*}

Topological categories and atlases embraces the binary operation of
composition, $\circ$, on morphisms as well as the union $\cup$ and
intersection $\cap$ operations on category objects.
An atlas, $\atlas{F}$, supports the following properties:
Composition of a diffeomorphism and dual or vis versa
renders the identity morphism for the respective domain or codomain object.
The dual operation extends to any atlas which compose to an identity atlas.

\begin{itemize}
\item\textbf{Target Space Covering}
\begin{equation*}
\begin{array}{c}
M=\bigcup_{\MU{U}\in\MANobj{M}}\MU{U}=\bigcup_{\morph{m}\in\atlasobj{F}}\morphdom{m}
\end{array}
\end{equation*}
\item\textbf{Metric Intersection}
\begin{equation*}
\begin{array}{c}
\MU{U}_i,\,\MU{U}_j\in\MANobj{M},\quad \MU{U}_i=\morphdom{i}
  \;\text{and}\; \MU{U}_j=\morphdom{j} \\[5pt]
\morph{i},\,\morph{j}\in\atlasobj{F} \\[5pt]
\morph{i}\pa{\MU{U}_i\cap \MU{U}_j}\subset\morphcod{i}=\mU{E}_i \\[5pt]
  \;\text{and}\; \\[5pt]
\morph{j}\pa{\MU{U}_i\cap \MU{U}_j}\subset\morphcod{j}=\mU{E}_j \\[5pt]
\mU{E}_i,\,\mU{E}_j\in\METobj{E}\quad\text{metric open sets}
\end{array}
\end{equation*}
\item\textbf{Category Morphisms}
\begin{equation*}
\begin{array}{c}
\mapfab{\morph{j}\circ\morph{i}^{-1}}{\morph{i}\pa{\MU{U}_i\cap \MU{U}_j}}
  {\morph{j}\pa{\MU{U}_i\cap \MU{U}_j}} \\[5pt]
  \;\text{and}\; \\[5pt]
\mapfab{\morph{i}\circ\morph{j}^{-1}}{\morph{j}\pa{\MU{U}_i\cap \MU{U}_j}}
  {\morph{i}\pa{\MU{U}_i\cap \MU{U}_j}} \\[5pt]
\morph{i}\circ\morph{j}^{-1},\,\morph{j}\circ\morph{i}^{-1}\in\METarr{E} \\[5pt]
\morph{i}^{-1}\circ\morph{j},\,\morph{j}^{-1}\circ\morph{i}\in\MANarr{M}
\end{array}
\end{equation*}
\end{itemize}

Overall an atlas relates the tangent space of a
manifold with a metric category.
The pairwise morphisms of an atlas, $\atlas{F}$, implies
the interrelation of tangent space
coordinate frames.

A linear transform interposes tangent space coordinates associated with
metric sets, $\mU{E}_i,\,\mU{E}_j\in\METobj{E}$
relative to the intersection of cover sets, $\MU{U}_i\cap\MU{U}_j$,
where, $\MU{U}_i,\,\MU{U}_j\in\MANobj{M}$.

A stricter confomity of atlas structure defines the behavior of a manifold.
In particular, the morphisms from each tangent space and cover set must
map diffeomorphically to the metric space.
Restriction of metric sets to a uniform metric space, $E$,
implies a universal covering manifold of, $M$, with homology groups.
For instance, the atlas of the $2$-sphere associates two morphisms with
a the same complex space, $\mathbb{C}$.
The dimension of the tangent space
imposes a minimum differentiabliity condition on the associated
diffeomorphisms of the atlas.
Fundamental theorems for inverse mapping, convexity and Taylor
expansion intercalate all aspects of atlas definition.

Tangent Space

Tangent space confers the dividends earned from the travails of
atlas construction.
The manifold provides a deformable evironment for a current of charges
whereas the metric space transforms to a tangent vector space
describing the instantaneous dynamics of a flow within the manifold.

Ideally a manifold with complicated surface structure
yields to an algebraic decoupling and reduction into simpler manifolds.
Quantum particle models surrender to approximate solutions with reduction
techniques.
The application of diffeomorphic functor mappings
between manifolds lends utility accordingly.

\begin{figure}[section]
     \begin{tikzpicture}[
            > = stealth, % arrow head style
            shorten > = 1pt, % don't touch arrow head to node
            auto,
            node distance = 3cm, % distance between nodes
            semithick % line style
        ]

        \node[] (A) {$\cat{M}$};
        \node[] (B) [right of=A] {$\cat{E}$};
        \node[] (C) [below of=B,right of=D] {$\cat{F}$};
        \node[] (D) [below of=A] {$\cat{N}$};
        
        \path[->] (A) edge node {$\atlas{H}$} (D);
        \path[->] (A) edge node {$\atlas{F}$} (B);
        \path[->] (B) edge node {$\atlas{g}$} (C);
        \path[->] (D) edge node {$\atlas{G}$} (C);
      \end{tikzpicture}
      \caption {Arrow identity, composition and assosiative law
        }\label{fig:arrowalgebra}
\end{figure}

Functorially related 

Cosider a special functor, $\funct{\cat{F}\digamma\stigma\Stigma\varsigma}$,
that transforms a domain category,
$\cat{M}=\functdom{A}$, to a codomain,  $\cat{E}=\functcod{A}$.
The dual
Each arrow, $a\in\ofunctobj{A}$, is a bijection with inverse, $a^{-1}$.
Objects of the codomain category, $\ccobj{E}$, are open sets in
a Banach space, typically Euclidean or pseudo-Euclidean.
Objects of the domain category, $\cat{M}$,
are an open cover collection,
$\{\ppv{a}{\mathscr{U}}\}=\ccobj{M}$ of a manifold,
$\cat{M}$. The index, $\alpha$, designates the
first order arrows, $\ofunctobj{A}$.
Let $\mathcal{A}=\ofunctobj{A}$ The following unions :

The open cover of the manifold, $\pazocal{M}$, specifies
objects of a category
A collection $L$ $\mathcal{L}$ $\mathscr{L}$ $\pazocal{L}$
$\euscr{L}$ $\mathfrak{L}$ manifold, $\mathscr{M}$, is covered by 
A local open set, $\ppv{\alpha}{\mathscr{U}}$, of a manifold embodies a tangent space, $T_x$.
A parametrization, $\xi$, characterizes the local behavior of a tangent space,
a convenient basis for the coordinate frame.
Consider a point $x\in\mathscr{U}$, the local tangent space of $x$,
follows:

\begin{equation*}
\begin{array}{rcl}
\mathscr{U}&\in&\mathscr{O}\pa{V} \\[7pt]
T_x\mathscr{U}&=&\{\pa{x,\xi}\mid x\in\mathscr{U},\;
                  \xi\in \mathbb{K}^n\} \\[7pt]
T\mathscr{U}&=&\bigcup_{x\in\mathscr{U}} T_x\mathscr{U}
  \subset\mathscr{U}\times\mathbb{K}^n \\[7pt]
\end{array}
\end{equation*}

The parametrization, $\xi$, manifests as a basis for a local tangent space
or differential form characterizing the coordinate, $x_\mu$. The basis, $\xi$,
generates a local multilinear algebra. The set of orthonomal unit vectors,
$e_\mu=\frac{\partial\;}{\partial x^\mu}x$, represent a simple basis for a
flat tangent space.

Simple tangent spaces characterize free particles and independent fields.
Interaction of particles and fields require mapping into an interleaving
tangent space. Hence consider a mapping, $F$, from a tangent
space $T_x$ into a space $T_{F\pa{x}}$.

\begin{equation*}
\begin{array}{rcl}
&&\hspace{-8pt}\mapaob{\mathscr{U}}{F}{\mathscr{V}}\subset\mathbb{K}^m \\[11pt]
T_{F\pa{x}}\mathscr{V}&=&\{\pa{F\pa{x},F\circ\xi}\mid \\[7pt]
                          &&\quad F\pa{x}\in\mathscr{V},\;
                  F\circ\xi\in \mathbb{K}^m\} \\[7pt]
T\mathscr{V}&=&\bigcup_{F\pa{x}\in\mathscr{V}} T_{F\pa{x}}\mathscr{V}
  \subset\mathscr{V}\times\mathbb{K}^m \\[11pt]
&&\hspace{-20pt}\mapaob{T_x\mathscr{U}}{T_xF}{T_{F\pa{x}}\mathscr{V}}
\end{array}
\end{equation*}

Detail of the interleaving interaction of particles and fields
progresses via the sequential mapping from free tangent spaces to
ones with greater involvement. Consider a sequence of two mappings,
$F$ and $G$:

\begin{equation*}
\begin{array}{c}
\mapaob{\mathscr{U}}{F}{\mapaob{\mathscr{V}}{G}{\mathscr{W}}} \\[11pt]
\mapaob{T_x\mathscr{U}}{\;\;T_xF\;\;}{
  \mapaob{T_{F\pa{x}}\mathscr{V}}
  {T_{F\pa{x}}G}{T_{G\circ F\pa{x}}\mathscr{W}}} \\[11pt]
\end{array}
\end{equation*}

The mechanism of differential forms applies directly to the vector space
$\mathscr{U}\subset\mathbb{K}^n$ as ${\bigwedge}^p\mathscr{U}$.
A complete multilinear space in a tangent space is similarly.

\begin{equation*}
\begin{array}{rcl}
\bigwedge\mathscr{U}=\bigoplus^n_{p=0}{\bigwedge}^p\mathscr{U}
\end{array}
\end{equation*}

Differential Forms

The mathematical construct of $p$-forms and tangent space provision tools
to elaborate on conformal coordinates, surfaces and higher dimensional
structures familiar to manifolds.
These topological features generally vary nonlinearly
over manifolds. A simplifying assumptions constrains manifolds to vary
smoothly with countable continuous derivatives.
Hence consider a simple curve, $u^\mu$, in tangent space, $T_x$, coordinates
parameterized by a variable, $\tau$. furthermore a differential vector,
$dx_\mu$, measures alignment with, $u^\mu$, via scalar product:

\begin{equation*}
\begin{array}{c}
u^\mu dx_\mu\quad\text{or}\quad u_\mu dx^\mu
\end{array}
\end{equation*}

Assiming, $u^\mu$, to propagate smoothly as a function of, $\tau$,
a tangent vector develops accordingly from the chain rule:

\begin{equation*}
\begin{array}{c}
dx_\mu=\frac{d\xi_\mu}{d\tau}d\tau
\end{array}
\end{equation*}

The differential form evinces above as an infinitestimal vector.
Alternately a divergence of a vector direction denotes as $\partial_\mu$,
in a particular coordinate systems, $\xi$, with indices, $\mu$.

Using differential forms each tangent space reduces to an infinitestimal
patch. Linear coordinate transformation relate tangent spaces,
$T_x$ and $T_{x^\prime}$, from $\pa{x,\,\xi}$ to
$\pa{x^\prime,\,\xi^{\prime}}$. A one form transformation
shows as follows:

\begin{equation*}
\begin{array}{rcl}
\frac{\partial\;}{\partial x^{\prime\mu}}&=&
                \frac{\partial x^\nu}{\partial x^{\prime\mu}}
                      \frac{\partial\;}{\partial x^\nu}
\end{array}
\end{equation*}

The matrix, $\frac{\partial x^\nu}{\partial x^{\prime\mu}}$,
corresponds to the inverse of the gauge transform, $\gamma^\mu_\nu$.
Simplifying notation for a partial operator as,
$\partial^{\prime}_\mu=\frac{\partial\;}{\partial x^{\prime\mu}}$,
reduces the above to:

\begin{equation*}
\begin{array}{rcl}
\partial^{\prime}_\mu&=&\pa{\gamma^{-1}}^\nu_\mu\partial_\nu
\end{array}
\end{equation*}

Tensors support gauge transforms for higher order forms.
Smoothly continuous transforms mediate the analysis of cooridinate
behavior throughout a tangent space.

\begin{equation*}
\begin{array}{rcll}
\partial^{\prime\gamma}_\mu&=&
          \pa{\gamma^{-1}}^{\gamma\nu}_{\lambda\mu}\partial^\lambda_\nu & \\[11pt]
\lambda&&\text{contravariant index list }&\lambda_1\cdots\lambda_i \\[5pt]
\gamma&&\text{contravariant index list }&\gamma_1\cdots\gamma_j \\[5pt]
\mu&&\text{covariant index list }&\mu_1\cdots\mu_k \\[5pt]
\nu&&\text{covariant index list }&\mu_1\cdots\mu_l
\end{array}
\end{equation*}

Differential forms conform to the $\wedge$-algebra of multilinear forms.
As infinitestimals, differential forms maintain tensor qualities
with the manifest of aggregate measurable quantites under integration.
In expression, a one form is a vector such as $dx_\mu$ and a
two form is a wedge product of two one forms that defines infinitestiman area,
$dx_\mu\wedge dx_\nu=-dx_\nu\wedge dx_\mu$.
covatiant and contravariant products conjoin as a Clifford product,
consisting of a binary sum of a scalar product and a wedge product,
$dx_\mu dx^\mu$. Scalar product also cojoins forms of different order.
A $0$-form, $\phi_\mu$, with a $1$-form, $dx^\mu$, enjoin as $\phi_\mu\dx^\mu$,
whereas a $1$-form and diveregence, $dx^\mu\partial_\mu$,
generalizes to a tensor:

\begin{equation*}
\begin{array}{rcl}
dx^\nu\partial_\mu&=&\delta^\nu_\mu
\end{array}
\end{equation*}

A zero differential form corresponds with a zero multilinear form having
representation as the scalar field, $\mathbb{K}$.
Higher order differential forms
demonstrate greater description of local tangent space behavior,
in particular, the distinction as a differential tensor.

\begin{equation*}
\begin{array}{rcl}
\partial^{\prime}_\mu&=&\pa{\gamma^{-1}}^\nu_\mu\partial_\nu
\end{array}
\end{equation*}

Transformation of a differential form from one tangent coordinate
systems to another follows the same rules a differentiation:

\begin{equation*}
\begin{array}{rcl}
dx^{\prime\mu}&=&\gamma^\mu_\nu dx^\nu
\end{array}
\end{equation*}


A bijective mapping from a a domain tangent, $T_x$, space to a
codomain tangent space, $T_{x^\prime}$,

Coordinates of an open domain, $\mathscr{U}$, index a tangent space, $T_x$.
A basis, $\xi$, mapped over the domain, $\mathscr{U}$, completes
definition of the tangent space, $T_x$.

Stoke's Theorem
\begin{equation*}
\begin{array}{rcl}
\varphi\in\bigwedge^{p-1}\mathscr{U},&\quad&K\subset\mathscr{U} \\[11pt]
\lint{K} d\varphi &=& \olint{K}\varphi \\[11pt]
1\xslantmath{\delta}2\delta 3\atled\imath\xssmall{.8}{d}
  \tensorkor{i}[_\mu,^\nu]\tensorkor{d}[_\mu,^\nu,_\gamma]
  d\iota\dd
\end{array}
\end{equation*}

Closed Stokes Theorem
\begin{equation*}
\begin{array}{rcl}
\lint{V}\nabla\times\sv{f}\,dv&=&\olint{V}\uvector{n}\times\sv{f}\,da
\end{array}
\end{equation*}

Stokes Theorem-Flux Integral
\begin{equation*}
\begin{array}{rcl}
  \lint{S}
  \nabla\times\sv{f}\,ds&=&\olint{S}\sv{f}\cdot d\sv{l}
\end{array}
\end{equation*}

Divergence Form, $\sv{f}$ is a $p$ form
\begin{equation*}
\begin{array}{rcl}
\lint{V}\left(\omega\wedge d\sv{f}+\left(-1\right)^pd\omega\wedge\sv{f}\right)
      &=&\olint{V}\omega\wedge\sv{f} \\[7pt]
\text{where}\quad\lint{V}da\wedge d\sv{f}
      &=&\olint{V}da\wedge\sv{f} \\[9pt]
\text{when}\qquad dv\wedge\nabla\sv{f}&=&
            da\wedge d\sv{f} \\[7pt]
\text{and}\:\qquad\qquad d^2 a &=& 0
\end{array}
\end{equation*}

Delta Operator
\begin{equation*}
\begin{array}{rcl}
\delta_\mu&=&dx^\mu\partial_\mu,\quad\left[\partial_\mu,\,\partial_\nu\right]=0
\end{array}
\end{equation*}

Lagrangian Field Theory, Lagrangian field density
\begin{equation*}
\pazocal{L}\equiv\fa{\pazocal{L}}{\fv{\phi}{p},\,\fvl{\phi}{p}{\mu};\,x_\mu}
\end{equation*}

\begin{equation*}
\begin{array}{rcl}
x_\mu&&\text{coordinate component with index }\mu \\[5pt]
\fv{\phi}{p}&&\text{field component with index by }p  \\[5pt]
\fvl{\phi}{p}{\mu}&=&\partial_\mu\fv{\phi}{\mu}
                      \text{ field components with partial} \\[5pt]
              &&\text{derivative of component }x_\mu
\end{array}
\end{equation*}

Action Integral
\begin{equation*}
S=\lint{\pazocal{M}}d^4\mathpzc{m}\,\pazocal{L}
\end{equation*}

Least Action Principle
\begin{equation*}
\begin{array}{rcl}
0=d_\mu S
      &=&\lint{\pazocal{M}}d^4\mathpzc{m}\wedge d_\mu\pazocal{L}
\end{array}
\end{equation*}

First Order Lagrangian Form
\begin{equation*}
\begin{array}{rcl}
d_\mu\pazocal{L}&=&\delta_\mu\fvl{\phi}{p}{\nu}
\partial_{\fvl{\phi}{p}{\nu}}\pazocal{L}+
             \delta_\mu\fv{\phi}{p}\,\partial_{\fv{\phi}{p}}\pazocal{L} \\[9pt]
&=&
\partial_\nu\pa{\delta_\mu\fv{\phi}{p}\,\partial_{\fvl{\phi}{p}{\nu}}\pazocal{L}}
\\[9pt]
   &&-\delta_\mu\fv{\phi}{p}\,
      \pa{\partial_\nu\partial_{\fvl{\phi}{p}{\nu}}-\partial_{\fv{\phi}{p}}}
      \pazocal{L} \\[9pt]
\end{array}
\end{equation*}

Divergence of Lagrangian Form
\begin{equation*}
\begin{array}{rcl}
0&=&\lint{\pazocal{M}}d^4\mathpzc{m}\wedge\partial_\nu
\pa{\delta_\mu\fv{\phi}{p}\,
     \partial_{\fvl{\phi}{p}{\nu}}\pazocal{L}} \\[9pt]
&=&\olint{\pazocal{M}}d^3\mathpzc{m}\wedge\uvector{n}^\nu
\pa{\delta_\mu\fv{\phi}{p}\,\partial_{\fvl{\phi}{p}{\nu}}\pazocal{L}} \\[9pt]
0&=&\uvector{n}^\nu\pa{\delta_\mu\fv{\phi}{p}\,
     \partial_{\fvl{\phi}{p}{\nu}}\pazocal{L}}
     \mathlarger{\arrowvert}_{\partial\pazocal{M}} \\[9pt]
d^4\mathpzc{m}&=&d^3\mathpzc{m}\wedge d\mathpzc{m}^\nu \\[9pt]
0&=&\pa{\partial_\nu\partial_{\fvl{\phi}{p}{\nu}}
     -\partial_{\fv{\phi}{p}}}\pazocal{L} \\[9pt]
\end{array}
\end{equation*}

Second Order Euler-Lagrange Form for Fields
\begin{equation*}
\begin{array}{rcl}
d_\mu\pazocal{L}&=&\delta\fvl{\phi}{p}{\nu\mu}
                    \partial_{\fvl{\phi}{p}{\nu}}\pazocal{L}
              +\delta\fvl{\phi}{p}{\mu}\partial_{\fv{\phi}{p}}\pazocal{L}
              +\delta_\mu\pazocal{L}
                \\[9pt]
            &=&g_{\mu\nu}
              \delta\fvl{\phi}{p}{\mu\nu}\partial_{\fvl{\phi}{p}{\nu}}\pazocal{L}
              +g_{\mu\nu}
                \fvl{\phi}{p}{\mu}
                \delta_\nu\partial_{\fvl{\phi}{p}{\nu}}\pazocal{L}
              +\delta_\mu\pazocal{L}
                \\[9pt]
            &=&g_{\mu\nu}\delta_\nu\pa{\fvl{\phi}{p}{\mu}
                \partial_{\fvl{\phi}{p}{\nu}}\pazocal{L}}
              +\delta_\mu\pazocal{L}
                \\[9pt]
            &=&g_{\mu\nu}
               d_\nu\pa{\fvl{\phi}{p}{\mu}\partial_{\fvl{\phi}{p}{\nu}}\pazocal{L}}
              +\delta_\mu\pazocal{L}
                \\[9pt]
\end{array}
\end{equation*}

Euler-Lagrange Current Form
\begin{equation*}
\begin{array}{rcl}
-\delta_\mu\pazocal{L}&=&d_\nu\pa{\fvl{\phi}{p}{\mu}
               \partial_{\fvl{\phi}{p}{\nu}}\pazocal{L}
               -g^{\mu\nu}\pazocal{L}}
\end{array}
\end{equation*}

4-Momentum
\begin{equation*}
\begin{array}{rcl}
\fv{\pi}{p}^\mu&=&\partial_{\fvl{\phi}{p}{\mu}}\pazocal{L}
\end{array}
\end{equation*}

Stress-Energy-Momentum Tensor
\begin{equation*}
\begin{array}{rcl}
\qv{T}^\nu_\mu&=&\fvl{\phi}{p}{\mu}\fv{\pi}{p}^\nu-g^{\mu\nu}\pazocal{L}
\end{array}
\end{equation*}

Canonical Transform % with $g^{\mu\nu}$ an identity transform
\begin{equation*}
\begin{array}{rcl}
\pazocal{T}^\mu_\mu&=&
            \fv{\pi}{p}^\mu\fvl{\phi}{p}{\mu}-g^{\mu\mu}\pazocal{L} \\[5pt]
2\pazocal{L}&=&
            \fv{\pi}{p}^\mu\fvl{\phi}{p}{\mu}-\pazocal{T}^\mu_\mu \\[5pt]
d_\mu\pazocal{T}^\mu_\mu&=&\fv{\pi}{p}^\mu\delta_\mu\fvl{\phi}{p}{\mu}
          +\delta_\mu\fv{\pi}{p}^\mu\fvl{\phi}{p}{\mu}-
                            g^{\mu\mu}d_\mu\pazocal{L} \\[5pt]
&=&\delta_\mu\fv{\pi}{p}^\mu\fvl{\phi}{p}{\mu}
    -g^{\mu\mu}\delta_\mu\fv{\phi}{p}\pa{\partial_{\fv{\phi}{p}}\pazocal{L}}^\mu
    -g^{\mu\mu}d_\mu\pazocal{L} \\[5pt]
&=&\delta_\mu\fv{\pi}{p}^\mu\fvl{\phi}{p}{\mu}
    -g^{\mu\mu}\delta_\mu\fv{\phi}{p}\fvl{\pi}{p}{\mu}
    -g^{\mu\mu}\delta_\mu\pazocal{L} \\[5pt]
&=&\delta_\mu\fv{\pi}{p}\partial_{\fv{\pi}{p}}\pazocal{T}^\mu_\mu
    +\delta_\mu\fv{\phi}{p}\partial_{\fv{\phi}{p}}\pazocal{T}^\mu_\mu
    +\delta_\mu\pazocal{T}^\mu_\mu \\[5pt]
\end{array}
\end{equation*}

Hamiltonian Field Density, Minkowski form $g^{00}=1$
\begin{equation*}
\pazocal{H}\equiv\fa{\pazocal{H}}{\fv{\phi}{p},\,\fv{\pi}{p};\,x_\mu},\quad
\pazocal{H}=\pazocal{T}^0_0
\end{equation*}

Symplectic Field Components
\begin{equation*}
\fvl{\phi}{p}{\mu}=\partial_{\fv{\pi}{p}^\mu}\pazocal{T}^\mu_\mu,\quad
-\fvl{\pi}{p}{\mu}^\mu=\partial_{\fv{\phi}{p}}\pazocal{T}^\mu_\mu,\quad
-\delta_\mu\pazocal{L}=\delta_\mu\pazocal{T}^\mu_\mu
\end{equation*}

\begin{equation*}
d_\mu\pazocal{T}^\mu_\mu=\delta_\mu\pazocal{D}^\mu_\mu=-\delta_\mu\pazocal{L}
\end{equation*}

Alternate Functions
\begin{equation*}
\pazocal{F}\equiv\fa{\pazocal{F}}{\fv{\phi}{p},\,\fv{\pi}{p};\,x_\mu},\quad
\pazocal{G}\equiv\fa{\pazocal{G}}{\fv{\phi}{p},\,\fv{\pi}{p};\,x_\mu}
\end{equation*}

Poisson Bracket
\begin{equation*}
\pb{\pazocal{F}}{\pazocal{G}}_\mu=
          \partial_{\fv{\phi}{p}}\qv{\pazocal{F}}
          \partial_{\fv{\pi}{p}^\mu}\qv{\pazocal{G}}
         -\partial_{\fv{\pi}{p}^\mu}\qv{\pazocal{F}}
          \partial_{\fv{\phi}{p}}\qv{\pazocal{G}}
\end{equation*}

External Current Dependence 
\begin{equation*}
\dv{x_\mu}\pazocal{F}=\pb{F}{T^\mu_\mu}_\mu+\partial_\mu\pazocal{F}
\end{equation*}

Free Field: Closed system relative to $x_\mu$ coordinate system
\begin{equation*}
0=\delta_\mu\pazocal{L},\quad
\pazocal{H}\equiv\fa{\pazocal{H}}{\fv{\phi}{p},\,\fv{\pi}{p}},\quad
\pazocal{L}\equiv\fa{\pazocal{L}}{\fv{\phi}{p},\,\fvl{\phi}{p}{\mu}}
\end{equation*}

Conservation of Current for Free Field Lagrangian:
\begin{equation*}
\begin{array}{rcl}
0&=&d_\nu\,\qv{T}^\nu_\mu,\quad 0=\delta_\mu\pazocal{L}
\end{array}
\end{equation*}

Stress-Energy-Angular Momentum Tensor
\begin{equation*}
\begin{array}{rcl}
\pazocal{M}^{\mu\nu\gamma}&=&
           \pa{\qv{T}^{\mu\nu}x^\gamma-\qv{T}^{\mu\gamma}x^\nu} \\[5pt]
0&=&\partial_\mu\pazocal{M}^{\mu\nu\gamma}
\end{array}
\end{equation*}

Field Propagator
\begin{equation*}
\begin{array}{rcl}
\fv{\phi}{p}^\prime\pa{x_\mu^\prime}&=&\euscr{N}\pa{\Delta x_\mu}
                  \lint{\pazocal{P}_\mu}\hspace*{-4pt}
  \euscr{D}x_\mu\:\fv{\phi}{p}\pa{x_\mu}
                           e^{\frac{\iu}{\hbar}\delta_\mu\tau\pazocal{L}} \\[13pt]
\fv{\phi}{p}^\prime\pa{x_\mu^\prime}&=&
           e^{delta_\mu\tau\frac{\iu}{\hbar}\pazocal{L}}\:
                             \fv{\phi}{p}\pa{x_\mu}
\end{array}
\end{equation*}

Coordinate Geodesic Operator
\begin{equation*}
\begin{array}{rcl}
\delta_\mu&=&\mathfrank{L}_{\xi^\mu}
\end{array}
\end{equation*}

\begin{equation*}
\begin{array}{rcl}
\delta_\mu\phi&=&dx_\mu\xi^\mu\partial_\mu\phi \\[5pt]
\delta_\mu A^\nu&=&\pa{\partial_\nu\partial^\mu\xi}A^\nu-
                    \pa{\partial^\nu\xi}\partial_\nu A^\mu \\[5pt]
\delta_\nu A_\mu&=&\pa{\partial_\mu\partial^\nu\xi}A_\nu+
                    \pa{\partial^\nu\xi}\partial_\nu A_\mu \\[5pt]
\end{array}
\end{equation*}

Lagrangian Field Theory, Lagrangian field density
\begin{equation*}
\pazocal{L}\equiv\fa{\pazocal{L}}{\fv{\phi}{p},\,\fvl{\phi}{p}{\mu};\,x_\mu}
\end{equation*}

\begin{equation*}
\begin{array}{rcl}
x_\mu&&\text{coordinate component with index }\mu \\[5pt]
\fv{\phi}{p}&&\text{field component with index by }p  \\[5pt]
\fvl{\phi}{p}{\mu}&=&\partial_\mu\fv{\phi}{\mu}
                      \text{ field components with partial} \\[5pt]
              &&\text{derivative of component }x_\mu
\end{array}
\end{equation*}

Electromagnetic Potential
\begin{equation*}
\begin{array}{rcl}
\qvl{e}{A}^\mu&=&\pauli{0}\ssl{e}{a}+\sigma\cdot\svl{e}{a} \\[3pt]
\qvl{m}{A}^\mu&=&\pauli{0}\ssl{m}{a}+\sigma\cdot\svl{m}{a} \\[3pt]
\qv{A}^\mu&=&\qvl{e}{A}^\mu+\iu\qvl{m}{A}^\mu \\[3pt]
\adjoint{\qv{A}^\mu}=\qv{A}_\mu&=&\qvl{e}{A}_\mu+\iu\qvl{m}{A}_\mu
\end{array}
\end{equation*}

4 Vector Divergence
\begin{equation*}
\begin{array}{rcl}
\partial^\mu&=&\pauli{0}\partial_{ct}-\sigma\cdot\nabla \\[3pt]
\adjoint{\partial^\mu}=\partial_\mu&=&
               \pauli{0}\partial^{ct}+\sigma\cdot\nabla \\[3pt]
\Box&=&\partial_\mu\partial^\mu=\partial^\mu\partial_\mu=
               \partial_{ct}^2-\nabla^2
\end{array}
\end{equation*}

Field Strength Tensor
\begin{equation*}
\begin{array}{rcl}
\partial_\mu\qv{A}^\mu&=&
    \pauli{0}\left(\partial_{ct}\ssl{e}{a}^0+\nabla\cdot\svl{e}{a}\right)
    +\iu\pauli{0}\left(\partial_{ct}\ssl{m}{a}^0+\nabla\cdot\svl{m}{a}\right)
              \\[3pt]
 &&+\sigma\cdot\left(\nabla\ssl{e}{a}^0+\partial_{ct}\svl{e}{a}
                  -\nabla\times\svl{m}{a}\right) \\[3pt]
 &&+\iu\sigma\cdot\left(\nabla\ssl{m}{a}^0+\partial_{ct}\svl{m}{a}
                  +\nabla\times\svl{e}{a}\right) \\[3pt]
               &=&\sigma\cdot\left(-\qv{E}+i\qv{B}\right) \\[3pt]
\partial^\mu\qv{A}_\mu&=&
                   \sigma\cdot\left(\qv{E}+\iu\qv{B}\right) \\[3pt]
0&=&\partial_{ct}a^0+\nabla\cdot\mathbf{a}\quad\text{Lorentz Gauge}
\end{array}
\end{equation*}

\begin{equation*}
\begin{array}{rclll}
\mathbf{u}&=&\frac{1}{2}\left(\abs{\qv{E}}^2+\abs{\qv{B}}^2\right)
&&\text{Energy Density} \\[3pt]
\mathbf{g}&=&\qv{E}\times\qv{B} &&\text{Poynting Vector}
\end{array}
\end{equation*}

Gauge Invariant Forms, $\Lambda$
\begin{equation*}
\begin{array}{rcl}
  0&=&\Box\Lambda\quad\text{Condition} \\[3pt]
  \qv{A}_\mu^\prime&=&\qv{A}_\mu-\partial_\mu\Lambda \\[3pt]
  \qv{A}^\mu^\prime&=&\qv{A}^\mu-\partial^\mu\Lambda \\[3pt]
  \partial^\mu\qv{A}_\mu^\prime&=&\partial^\mu\qv{A}_\mu-\Box\Lambda
        =\partial^\mu\qv{A}_\mu\\[3pt]
  \partial_\mu\qv{A}^\mu^\prime&=&\partial_\mu\qv{A}^\mu-\Box\Lambda
        =\partial_\mu\qv{A}^\mu
\end{array}
\end{equation*}

Power Equations
\begin{equation*}
\begin{array}{rcl}
\frac{1}{2}\left(\partial^\mu\qv{A}_\mu\right)
  \left(\partial_\mu\qv{A}^\mu\right)&=&
              -\pauli{0}\mathbf{u}-\sigma\cdot\mathbf{g} \\[3pt]
\frac{1}{2}\left(\partial_\mu\qv{A}^\mu\right)
  \left(\partial^\mu\qv{A}_\mu\right)&=&
              -\pauli{0}\mathbf{u}+\sigma\cdot\mathbf{g}
\end{array}
\end{equation*}

Relativistic Conservation
\begin{equation*}
\begin{array}{rcl}
\left(\partial_\mu\qv{A}^\mu\right)^2&=&
                  \pauli{0}\left(\abs{\qv{E}}^2-\abs{\qv{B}}^2
                -\iu 2\left(\qv{E}\cdot\qv{B}\right)\right) \\[3pt]
\left(\partial^\mu\qv{A}_\mu\right)^2&=&
                  \pauli{0}\left(\abs{\qv{E}}^2-\abs{\qv{B}}^2
                 +\iu 2\left(\qv{E}\cdot\qv{B}\right)\right) \\[3pt]
\end{array}
\end{equation*}

Charge Density
\begin{equation*}
\begin{array}{rclll}
4\pi\ssl{e}{\rho}&=&\nabla\cdot\qv{E}&&\text{Electric} \\[3pt]
4\pi\ssl{m}{\rho}&=&\nabla\cdot\qv{B}&&\text{Magnetic}
\end{array}
\end{equation*}

Current Density
\begin{equation*}
\begin{array}{rclll}
\frac{4\pi}{c}\svl{e}{j}&=&
              -\partial_{ct}\qv{E}+\nabla\times\qv{B}&&\text{Electric} \\[3pt]
\frac{4\pi}{c}\svl{m}{j}&=&
               -\partial_{ct}\qv{B}-\nabla\times\qv{E}&&\text{Magnetic}
\end{array}
\end{equation*}

Electro-Magnetic Current
\begin{equation*}
\begin{array}{rclll}
\frac{c}{4\pi}\qvl{e}{J}_\mu&=&\pauli{0}\,c\ssl{e}{\rho}
                       -\sigma\cdot\svl{e}{j}&&\text{Electric} \\[3pt]
\frac{c}{4\pi}\qvl{m}{J}_\mu&=&\pauli{0}\,c\ssl{m}{\rho}
                       -\sigma\cdot\svl{m}{j}&&\text{Magnetic}
\end{array}
\end{equation*}

Current Tensor
\begin{equation*}
\begin{array}{rcl}
\partial^\mu\partial_\mu\qv{A}^\mu&=&
         \pauli{0}\nabla\cdot\left(\qv{E}-\iu\qv{B}\right)
        -\sigma\cdot\left(\partial_{ct}\qv{E}-
                                   \iu\partial_{ct}\qv{B}\right) \\[3pt]
      &&+\iu\sigma\cdot\nabla\times\left(\qv{E}-\iu\qv{B}\right) \\[3pt]
   &=&\qvl{e}{J}^\mu-\iu\qvl{m}{J}^\mu \\[3pt]
\partial_\mu\partial^\mu\qv{A}_\mu&=&
        \pauli{0}\nabla\cdot\left(\qv{E}+\iu\qv{B}\right)
        +\sigma\cdot\left(\partial_{ct}\qv{E}+
                                    \iu\partial_{ct}\qv{B}\right) \\[3pt]     &&+\iu\sigma\cdot\nabla\times\left(\qv{E}+\iu\qv{B}\right) \\[3pt]
   &=&\qvl{e}{J}_\mu+\iu\qvl{m}{J}_\mu \\[3pt]
\end{array}
\end{equation*}

Continuity Current
\begin{equation*}
0=\partial_t\,\ssl{e}{\rho}+\nabla\cdot\svl{e}{j}=
  \partial_t\,\ssl{m}{\rho}+\nabla\cdot\svl{m}{j}
\end{equation*}

\begin{equation*}
\begin{array}{rcl}
\partial_\mu\left(\qvl{e}{J}^\mu-\iu\qvl{m}{J}^\mu\right)&=&
\Box\,\sigma\cdot\left(-\qv{E}+\iu\qv{B}\right) \\[3pt]
\partial^\mu\left(\qvl{e}{J}_\mu+\iu\qvl{m}{J}_\mu\right)&=&
\Box\,\sigma\cdot\left(\qv{E}+\iu\qv{B}\right) \\[3pt]
\frac{c}{4\pi}\Box\sigma\cdot\qv{E}
       &=&-\sigma\cdot\left(\nabla c\ssl{e}{\rho}+\partial_{ct}\,\svl{e}{j}
          +\nabla\times\svl{m}{j}\right) \\[3pt]
\frac{c}{4\pi}\Box\sigma\cdot\qv{B}
       &=&-\sigma\cdot\left(\nabla c\ssl{m}{\rho}+\partial_{ct}\,\svl{m}{j}
          -\nabla\times\svl{e}{j}\right)
\end{array}
\end{equation*}

Work Density
\begin{equation*}
\begin{array}{rcl}
  \partial_\mu\partial^\mu\qv{A}_\mu\partial_\mu\qv{A}^\mu&=&
              \left(\qvl{e}{J}_\mu+\iu\qvl{m}{J}_\mu\right)
              \sigma\cdot\left(-\qv{E}+\iu\qv{B}\right) \\[3pt]
  &=&-2\partial_\mu
      \left(\pauli{0}\mathbf{u}+\sigma\cdot\mathbf{g}\right)
\end{array}
\end{equation*}

Electromagnetic Power Density
\begin{equation*}
\begin{array}{rcl}
  \partial_\mu\left(\pauli{0}\mathbf{u}+\sigma\cdot\mathbf{g}\right)
  &=&\pauli{0}
       \left(\partial_{ct}\mathbf{u}+\nabla\cdot\mathbf{g}\right)\\[3pt]
   &&\sigma\cdot\left(\nabla\mathbf{u}+\partial_{ct}\mathbf{g}+
                       i\nabla\times\mathbf{g}\right) \\[3pt]
\end{array}
\end{equation*}

Electromagnetic Force Density
\begin{equation*}
\begin{array}{rcl}
             && \sigma\cdot\left(-\qv{E}+\iu\qv{B}\right)
                \frac{4\pi}{c}\left(\qvl{e}{J}_\mu+\iu\qvl{m}{J}_\mu\right)
               \\[3pt]
&&\quad =\pauli{0}\left(
            \qv{E}\cdot\left(-\partial_{ct}\qv{E}+\nabla\times\qv{B}\right)
            +\qv{B}\cdot\left(-\partial_{ct}\qv{B}-\nabla\times\qv{E}\right)
                                                             \\[3pt]
   &&\qquad -\iu\qv{B}\cdot\left(-\partial_{ct}\qv{E}+\nabla\times\qv{B}\right)
            +\iu\qv{E}\cdot\left(-\partial_{ct}\qv{B}-\nabla\times\qv{E}\right)
            \right) \\[3pt]
   &&\qquad +c\sigma\cdot\left(
                  -\qv{E}\left(\nabla\cdot\qv{E}\right)
                  -\qv{B}\left(\nabla\cdot\qv{B}\right)
                  +\iu\qv{B}\left(\nabla\cdot\qv{E}\right)
                  -\iu\qv{E}\left(\nabla\cdot\qv{B}\right)
             \right) \\[3pt]
   &&\qquad +\sigma\cdot\left(
     +\iu\qv{E}\times\left(-\partial_{ct}\qv{E}+\nabla\times\qv{B}\right)\\[3pt]
   &&\qquad\qquad
     +\iu\qv{B}\times\left(-\partial_{ct}\qv{B}-\nabla\times\qv{E}\right)\\[3pt]
   &&\qqud\qquad
        +\qv{B}\times\left(-\partial_{ct}\qv{E}+\nabla\times\qv{B}\right)\\[3pt]
   &&\qqud\qquad
        -\qv{E}\times\left(-\partial_{ct}\qv{B}-\nabla\times\qv{E}\right)
             \right) \\[3pt]
&&\quad =-2\pauli{0}\left(\partial_{ct}\mathbf{u}+\nabla\cdot\mathbf{g}
                    \right) \\[3pt]
&&\qquad -2\sigma\cdot\left(\nabla\mathbf{u}+\partial_{ct}\mathbf{g}
                  +\iu\nabla\times\mathsf{g}\right)
\end{array}
\end{equation*}

Tensor Transform
\begin{equation*}
\begin{array}{rcl}
&&\qv{C}\left(\nabla\cdot\qv{C}\right)-\qv{C}\times\nabla\times\qv{C} \\[3pt]
&&\qquad
   =\qv{C}\left(\nabla\cdot\qv{C}\right)+\left(\qv{C}\cdot\nabla\right)\qv{C}
    -\frac{1}{2}\nabla\left(\qc{C}\cdot\qv{C}\right) \\[3pt]
&&\qquad
   =\nabla\cdot\left(
   \qv{C}\qv{C}^\dagger-\frac{1}{2}\qv{I}\left(\qv{C}^\dagger\qv{C}\right)\right)
\end{array}
\end{equation*}

Maxwell Stress Tensor
\begin{equation*}
\begin{array}{rcl}
  \qv{T}&=&\qv{E}\qv{E}^\dagger+\qv{B}\qv{B}^\dagger-\qv{I}\mathsf{u}
\end{array}
\end{equation*}

Poynting Flux
\begin{equation*}
\begin{array}{rcl}
\frac{1}{2}\nabla\times\mathsf{g}&=&
           \qv{E}\left(\nabla\cdot\qv{B}\right)
           -\qv{B}\left(\nabla\cdot\qv{E}\right) \\[3pt]
         &&+\left(\qv{B}\cdot\nabla\right)\qv{E}
           -\left(\qv{E}\cdot\nabla\right)\qv{B}
\end{array}
\end{equation*}

\begin{definition}{Collection, Foundation Context}\label{def:collection},
  everything is a collection. A collection represents a concept or entity
  associated with a label or identity.
  The trivial collection, $\varnothing$, is uniquely empty,
  consisting of nothing. Non-trival collections consist of other collections.
  Category specializes collection. Objects are constituents of a collection.
  A collection associates with attributes as follows:
\begin{itemize}
\item\textbf{(Arrow)} an order pair consisting of a domain collection
  associated with a codomain collection.
  \begin{itemize}
    \item\textbf{(Domain)} a binding of the arrow
    \item\textbf{(Codomain)} the referent binding of the arrow
 \end{itemize}
\item\textbf{(Property)} a codomain associated with a domain collection.
  A property attributed to a collection or the constituents of a collection
\item\textbf{(Relation)} a codomain associated with a domain arrow.
  Membership, subcollection and equals are specific examples of relations.
  A property associated with the domain to codomain pair of an arrow.
  Denote the general collection of relations as a collection, $\mathcal{C}$
  \begin{itemize}
    \item\textbf{(Membership, $\in$)} an arrow, the domain
      is a constituent collection of the codomain collection.
      The domain of membership is an \textbf{object} of the
      codomain of membership. A domain or codomain must be objects
      as members of the collection for an arrow
    \item\textbf{(Subcollection, $\lessdot$)} an arrow, the constituents of the
      domain collection are constituents of the codomain collection
    \item\textbf{(Equals, $=$)} an arrow, with domain and codomain
      the same collection, an identity arrow of this collection.
      Two objects, $\obk{a}$ and $\obk{b}$, that commute relative
      to the subcollection arrow, $\obk{a}\lessdot\obk{b}$ and
      $\obk{b}\lessdot\obk{a}$, then $\obk{a}=\obk{b}$
 \end{itemize}
\item\textbf{(Equivalence, $\sim$)} an arrow, an unordered pair
  of domain and codomain collections. The domain and codomain
  collections are equivalent relative to common properties.
  Equality implies equivalence but equvalence does not imply equality.
  Equivalence is a partial condition but not an absolute requirement
  for equality.
\end{itemize}
\end{definition}



\section{Appendix}

\bibliography{sample-paper}

\bibliographystyle{prsty}
\begin{thebibliography}{99}
\bibitem{cartan1966}Cartan,\'{E}., ``The Theory of Spinors'', Hermann, [1966]
\bibitem{cartan1913}Cartan,\'{E}., ``Les Groupes Projectifs qui ne laissent
  invariante aucune multiplicit\'{e} plane'', Bull. Soc. Math. France, [1913]
\bibitem{grothendieck1957}Grothendieck,A., ``Sur quelques points d`Alg\`{e}bra Homologique'', T\^{o}hoku Math. J., 9, pp. 119-221, [1957]
\bibitem{brauerweyl1937}Brauer,R. and Weyl,H., ``Spinors in n Dimensions'', Am. J. Math., 57, pp. 425-449, [1937]
\bibitem{russell1902}Russell,B., Letter to Frege,In Heijenoort 1967, pp. 124-125, [1902]
\bibitem{russell1903}Russell,B., ``The Principles of Mathematics'', Cambridge Univ. Press, Cambridge, Vol. I, [1903]
\bibitem{pauli1927}Pauli,W., ``Zur Quantenmechanik des Magnetischen Elektrons'', Z. Phys., 43, pp. 601-623, [1927]
\bibitem{zermelo1908}Zermelo,E., ``Untersuchungen \"{U}ber die Grundlagen der Mengeniehre``, Math. Ann., 65, pp. 261-281, [1908]
\bibitem{schafer2008}Schafer,R.D., ``An Introduction to Nonassociative Algebras'', Project Gutenberg(Public Domain USA). Internet, [2008]
\bibitem{dirac1928}Dirac,P.A.M., ``The Quantum Theory of the Electron'', Proc. R. Soc. \(London\), [1928]
\bibitem{birkoff1935}Birkoff,G.D., ``On the Structure of Abstract Algebras'', Proc. Cambridge P. Soc. 31, pp. 433-454, [1935]
\bibitem{jipsenrose1992}Jipsen,P.,Rose,H.``Varieties of Lattices'', Springer-Verlag,Berlin-Heidelberg, [1992]
\bibitem{hodges1993}Hodges,W., ``Model Theory'', Cambridge Univ. Press, Cambridge, [1993]
\bibitem{lawvere1965}Lawvere,F.W., ``An Elementary Theory of the Category of Sets'', Proc. N. Acad. Sc., PNAS, [1965]
\bibitem{lawvere1997}Lawvere,F.W.,Shanuel,S.H., ``Conceptual Mathematics, A first Introduction to Categories'', Cambridge Univ. Press, Cambridge, [1997]
\bibitem{awodey2010}Awodey,S., ``Category Theory'', Oxford Univ. Press, Oxford, [2010]
\bibitem{barrwells1990}Barr,M.,Wells,.C, ``Category Theory for Computing Science'', Prentice Hall, [1990], pdf version online [1998]
\bibitem{lang2002}Lang,S., ``Algebra'', Springer-Verlag, NY, [2002]
\bibitem{maclane1986}Mac Lane,S., ``Mathematics Form and Function'', Springer-Verlag, NY, [1986]
\bibitem{maclane1998}Mac Lane,S., ``Category Theory for the Working Mathemetician'', Second Edition, Springer-Verlag, NY, [1998]
\bibitem{bourbaki1989}Bourbaki,N., ``Algebra I'', Springer-Verlag, Berlin Heidelberg, [1989]
\bibitem{bourbaki22003}Bourbaki,N., ``Algebra II'', Springer-Verlag, Berlin Heidelberg, [2003]
\bibitem{hungerford1974}Hungerford,T.W., ``Algebra'', Springer-Verlag, NY, [1974]
\bibitem{burris1981}Burris,S.,Sankappanavar,H.P., ``A Course in Universal Algebra'', Springer-Verlag, NY, [1981]
\bibitem{rotman1998}Rotman,J., ``Galois Theory'', Springer-Verlag, NY, [1998]
\bibitem{fultonharris2004}Fulton,W., Harris,J., ``Representation Theory: A First Course'', Springer-Verlag, NY, [2004]
\bibitem{humphreys1972}Humphreys,J., ``Introduction to Lie Algebras and Representation Theory'', Springer-Verlag, NY, [1972]
\bibitem{brouwer1911}Brouwer,L.E.J., ``\"Uber Abbildung von Mannigfaltigkeiten'', Math. Ann., 71, pp. 97-115, [1911]
\bibitem{birkhoff1913}Birkhoff,G.D., ``Proof of Poincar\'e's Geometric Theorem'', Trans. Amer. Math. Soc., 14, pp. 14-22, [1913]
\bibitem{birkhoff1925}Birkhoff,G.D., ``An Extension of Poincar\'e's Last Geometric Theorem'', Acta Math., 47, pp. 297-311, [1925]
\bibitem{poincare1912}Poincar\'e,H., ``Sur un Theor\`eme de G\'eom\'etrie'', Rend. Circ. Mat. Palermo, 33, pp. 375-407, [1912]
\bibitem{brownnwumann977}Brown,H.,NeumannD., ``Proof of the Poincar\'e-Birkhoff Fixed-Point Theorem'', Michigan Math. J., 24, pp. 21-31, [1977]

\end{thebibliography}

\end{document}
