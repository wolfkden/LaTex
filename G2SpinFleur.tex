%\documentclass[aps,twocolumn,secnumarabic,graphics,flotfix,graphicx,
%url,bm,tightenlines,nobibnotes,nobalancelastpage,amsmath,amssymb,
%nofootinbib]{article}
\documentclass[aps,twocolumn,secnumarabic,nobalancelastpage,amsmath,amssymb,
amsthm,nofootinbib,parskip=full]{revtex4}

% standard graphics specifications
% alternative graphics specifications
% helps with long table options
% for on-line citations
% special 'bold-math' package

\usepackage{utf8}{inputenc}
%\usepackage[bottom]{footmisc}
\usepackage{graphics}      % standard graphics specifications
\usepackage{graphicx}      % alternative graphics specifications
\usepackage{longtable}     % helps with long table options
\usepackage{url}           % for on-line citations
\usepackage{bm}            % special 'bold-math' package
\usepackage[utf8]{inputenc}
\usepackage[english]{babel}

\newtheorem{theorem}{Theorem}[section]
\newtheorem{corollary}{Corollary}[theorem]
\newtheorem{lemma}[theorem]{Lemma}

%%\theoremstype{definition}
\newtheorem{definition}{Definition}[section]

\newtheorem*{remark}{Remark}

%\setlength{\parskip}{\baselineskip}
%\setlength{\parskip}{4em}

\newcommand{\di}[1]{\boldsymbol #1}

\newcommand{\dI}{\boldsymbol \imath}

\begin{document}
\title{Nonassociative Structures in Physics}
\author         {Wolfgang Kraske}
\email        {wolfkden@gmail.com}
\homepage     {http://www.oviumzone.com}
\affiliation  {OVium Studies in Physics}
\date{\today}

\begin{abstract}
Associative algebra is a cornerstone of particle
physics and mathematics. A sustained flourish of
Lie algebra research for more than century maintains
strict associative rules to establish the fields of
quantum mechanics and gravitation. The new
tensor formulation of physics demanded the surrender of
classical commutative structure to model
paradoxical phenomenon, such as electron spin,
while maintaining acquiescence to associative rules.
The mathematics of Lie algebra supported the journey
with Lie brackets and grading of algebraic structures.
An emerging perogtive to develop non-associative algebra 
harkens from the relentless pursuit of new frontiers in
fields such as particle physics, gravity, quantum computation,
functional programming and artificial intelligence.
For instance in particle physics the exceptional algebra, $G2$,
has been explored primarily with the formalism of Lie algebra
despite obvious nonassociative behaviors of the module
structure. The algebra of $G2$ provides a curious
complex of intertwining spin structures that interact
nonassociatively with cartan basis elements. Interestingly
the non-associative nature of $G2$ is limited to an ambiguity
of polarity. Hence the spin submanifolds of $G2$ subordinate to a
larger double cover structure of the compact space, $Sp(8)$.
A great expanse of algebraic structure is expanded in
this manuscript as embraced by
Bourbaki's magma structure, \cite{bourbaki1989}, and
more general composition algebras.
\end{abstract}

\maketitle

\setlength{\parindent}{0em}
\setlength{\parskip}{0.5em}
\renewcommand{\baselinestretch}{1.0}


\section{Appendix}

Notation and peripherial concepts not detailed herein are
found in Hungerford's algebra text, \cite{hungerford1974}.
For instance topics relied on in the text are,
\cite{hungerford1974}, cartesian product,
$A^{n+1}=A^n\times A$, $n\in\mathbb{N}$, tensor product,
direct sum and set theory.
Axioms of choice and order, \cite{hungerford1974},
that intuit a basis of the real number set,
$\mathbb{R}$, are also assumed. Other areas
of adherance are model theory, \cite{hodges1993},
and category theory, \cite{awodey2010}.
Certainly the study of mathematics embraces several
language theories, \cite{hodges1993},
or disaplines. In particular
the theory of algebra associates well with
category theory, \cite{awodey2010}. A relationship that
is exploited in the text.

The manuscript details the composition and extension of
mathematical strucures the require a formalism and appliance
of algebra beyond the scope of traditional text, \cite{hungerford1974}.
To achieve this objective concepts and definitions are
adopted from Birkoff, \cite{birkoff1935}, Hodges, \cite{hodges1993},
and Burris, \cite{burris1981}:

\begin{definition}{Algebra, $\mathcal{A}$}\label{algebradef},
  a language theory shaped set of functions 
\end{definition}

\begin{definition}{Algebra, $\mathcal{A}$}\label{algebradef},
  a language theory consisting of models with
  a signature $(A,F)\models\mathcal{A}$,
  consisting of a nonempty set, $A$, of elements and a
  nonempty set of composable functions, $F$.
  $f\in F$. such that $f:\,A^{n}\rightarrow A$.
  Nullary functions have an arity value of zero,
  A nullary function of a signature is map of a
  constant element in the set, $A$.
\begin{itemize}
\item\textbf{\small (Arity)} arity
  function, $ar:\,F\rightarrow\mathbb{N}$,
  partitions the function
  set, $F$, by mapping signature
  $f:\,A^{n}\rightarrow A\,\forall\,f\in F$.
\item \textbf{\small (Closure)} codomains
  of the functions in the set $F$ cover
  the element set $A$: \\
  $A\subset\,\cup_{f\in F}f(A^{ar(f)})$
\item \textbf{\small (Composition)}
  Each function, $f\in F$, of the signature
  is characterised by
  an integral arity, $n=ar(f)\in\mathbb{N}$,
  $\cup_{f\in F}f(A^{ar(f)})\subset A$
\item \textbf{\small (Constant)}
  $\cup_{f\in F}f(A^{ar(f)})\subset A$
\end{itemize}
\end{definition}

A property of any algebra, $(A,F)$, is containment:

\begin{itemize}
\item \textbf{\small (Containment)}
  $\cup_{f\in F}f(A^{ar(f)})\subset A$
\end{itemize}

If the union of the codomains of the function set, $F$, cover $A$ then the algebra
has a closure property:

\begin{itemize}
\item \textbf{\small (Compositional Closure)}
  $\cup_{f\in F}f(A^{ar(f)})=A$
\end{itemize}

An algebra with closure has a basis:

\begin{itemize}
\item \textbf{\small (Basis)}
  $X\subset A$, $X$ is a basis of $A$ $\Rightarrow$ $A=\langle X\rangle=\cup_{f\in F}f(X^{ar(f)})$.
\end{itemize}

The composition of functions, $f\in F$, dictate a structure on elements of the set $A$.
An algebra, $(A,f)$, with
a binary function, $f$, dictates a tree structure on members of the set,
$A$, relative to a basis $A=\langle X\rangle$.

\begin{itemize}
\item \textbf{\small (Binary Forest)}
  $(A,f)$ is an algebra, $f$ is a binary function and $A=\langle X\rangle$:
  $c\in A\Rightarrow c=f(a,b)\,a,b\in A$ or $c\in X$.
\end{itemize}



If an algebra, $(A,f)$, has an identity element, $1_A$, relative to the binary
function, $f$ then the condition for a basis is relaxed:

\begin{itemize}
\item \textbf{\small (Binary Identity)}
  $(A,f)$ is an algebra. $f$ is a binary function with identity element, $1_A\in A$: \\
  $a=f(a,1_A)=f(1_A,a)\forall\,a\in A$ then: \\
  $c\in A\Rightarrow c=f(a,b)\,a,b\in A$.
\end{itemize}

Algebras with a binary function over a set
are classified as magmas. Borbaki, \cite{bourbaki1989},
employed magmas to define a wide vatiety of signatures
including the classical signatures in algebra.

The composalble functions of an algebra also dictate invariant
subalgebras.

\begin{definition}{Sublgebra}\label{subalgebradef},
  denotes an algebra subset of another algebra:
  $B\subset A$ where both $(B,F)$ and $(A,F)$ are algebras.
  A proper subalgebra requires $B\neq A$.
\end{definition}

As a corollary to, definition \ref{subalgebradef}, a proper subalgebra
cannot be a basis for the containing algebra.

As an example the power set $\mathcal{P}(A)$ of a set $A$,
with the binary function of union or intersetion
is an example of an algebra, $(\mathcal{P}(A),\cup)$
or $(\mathcal{P}(A),\cap)$. Binary functions may also
be used with operator notation, $f(A,B)=A\,{}^\backprime f{}^\backprime \,B$ where
tick accents $\backprime$ are omitted for familiar operators such as
$\cup,\,\cap,\,+,\,\cdot,\,\bullet$. Since the
binary set operators $\cup$ and $\cap$, are associative the algebras,
$(\mathcal{P}(A),\cup)$ and $(\mathcal{P}(A),\cap)$, provide
examples of semigroups. Algebras with
associative properties, Su, define a semigroup:

\begin{definition}{Semigroup, $(G,\bullet)$}\label{semigroup},
  an algebra with an associative binary operator, $\bullet$, \\
  $a\bullet (b\bullet c) = (a\bullet b)\bullet c\in G$ $\forall$ $a,b,c\in G$
\end{definition}

A semigroup with an identity element is a monoid:



The various function signature of an algebra, $F$, $(A,F)$,
permits the analysis of the structure and behavior
of the set, $A$, relative to the functions of $F$.
Closure is an obvious property of algebra class:



Various structures of algebra are defined with different
signatures. Signatures with a binary function define
a large class of structures. Operator notation
is convenient to denote binary functions for a
structure, $(A,\bullet)$. A semigroup is
such a structure with the additional property
of associativity 



\begin{definition}{Composition Algebra },
  
\end{definition}

Omega algebra provides the generalization

Magmas are introduced, \cite{borbaki1989}, to fully embrace
algebraic developments. A Magma is a
set with a binary composition operation. A magma




Algebraic developments are detailed in Serge Lang's text, 
\cite{lang2002}, and Joseph Rotman's text, \cite{rotman1998}. 
Development begins with sets and the action of a binary operator on a set.
Greater development will be dedicated to compositional algebras that
combine multiple structures and operators.

Consider a set $G$ with a binary function, a binary operator:
\begin{center}
$\cdot\, {:}\, G \times G \to G$:
\end{center}

\begin{itemize}
\item \textbf{\small (Closure)} $a\cdot b\in G$ $\forall$ $a,b\in G$
\end{itemize}

A simplifying notation for binary operator, $\cdot$ , is used:

\begin{center}
 $\cdot(a,b)\equiv a\cdot b$
\end{center}

A binary operator that is invariant to the order of the input argument pair
is commutative:

\begin{itemize}
\item \textbf{\small (Commutative)} $a\cdot b=b\cdot a\in G$, $\forall$ $a,b\in G$
\end{itemize}

Algebraic structures with commutative algebraic operations are abelian.

A semigroup is a set, $G$, with an associative binary operator, $\cdot$ :

\begin{itemize}
\item \textbf{\small (Associative)}
  $a\cdot (b\cdot c) = (a\cdot b)\cdot c\in G$ $\forall$ $a,b,c\in G$
\end{itemize}

A semigroup, $G$, with an identity element, $e\in G$, is a monoid $(G,\cdot,e)$:

\begin{itemize}
\item \textbf{\small (Identity)} $a\cdot e=e\cdot a=a$, $\forall$ $a \in G$
\end{itemize}

Uniqueness of the identity element derives as a corollary to the monoid definition.
The trivial monoid consists of a single identity element, $G=\{e\}$.

A monoid, $(G,\cdot,e)$, is a group if for every element, $a\in G$, 
there exists an inverse element $a^{-1}\in G$ such that:

\begin{itemize}
\item \textbf{\small (Inverse)} $\forall$ $a\in G$ $\exists$ $a^{-1}\in G$ \\
  such that $e=a\cdot a^{-1}=a^{-1}\cdot a$
\end{itemize}

As a corollary, the inverse of any element of a group is unique.

To support the development of a class of composite algebras a
free group provides a construct.
A free group is constructed from a set, $X$, and a unit, $e$.
A vocabulary $W$ of words $w$ of length $\ell(w)\in\mathbb{N}$ are constructed
by listing all combinations elements of $X$ such that $W\in X^{\ell(w)}$.
The identity element, $e$, is the zero length word.
The product of words is fulfiled by the non-abelian operation of 
appending of words.
Assume the product identity is $e$.
The vocabulary is the set $W={\cup}_{i\in\mathbb{N}}X^i$.
A collection of words that reduce to $e$ of length 0 determine a rule set
to reduce the length of words in W. 
This group structure is familiar to Coxeter groups and
useful in developing bases for algebras

A module is developed as a free group over a field. The center of the module
is the field.
If the generating set of the free group is abelian then the module is a polynomial ring. 

Semigroups, monoids and groups are algebraic structures 
with a single binary operator.
A compositional algebra is a structure with more than one 
binary operator. 
The simplest compositional algebra is a ring.
A ring, $(R,+,\cdot,0)$, is a composition of an 
additive commutative group $(R,+,0)$
and a multiplicative semigroup, $(R,\cdot)$.
A ring with identity is a ring with a 
multiplicative monoid $(R,\cdot,1)$.
All rings are assumed to have a multiplicative identity 
unless stated otherwise.
The composition of ring operators, $+$ and $\cdot$, 
demonstrate distributive ring, $R$, properties:

\begin{itemize}
\item \textbf{\small (Left Distributive)}
  $a\cdot(b + c)=a\cdot b+a\cdot c\in R,$
\item \textbf{\small (Right Distributive)}
  $(a+b)\cdot c=a\cdot c+b\cdot c\in R,$
\end{itemize}

\begin{center}
  $\forall$ $a,b,c\in R$.
\end{center}

The trivial ring consists of one element with one property:

\begin{itemize}
\item \textbf{(\small Trivial Ring)} $\mathfrank{Z}=\{0\}$ where $1=0$.
\end{itemize}

Unless otherwise stated all rings are assumed to be not trivial.

The distributive properties of a ring assert the following corollaries:

\begin{itemize}
\item \textbf{\small (Zero)} $0\cdot a=a\cdot 0=0$, $\forall$ $a\in R$
\item \textbf{\small (Unit Reflection)} $(-1)+1=1+(-1)=1-1=0$
\item \textbf{\small (Element Reflection)} $(-1)\cdot a=a\cdot (-1)=-a$, 
$(-a)+a=a+(-a)=a-a=0$ $\forall$ $a\in R$
\end{itemize}

A ring, $R$, has at least one ideal subset. All ideals, 
$\mathfrak{I}\subset R$, have properties:

\begin{itemize}
\item \textbf{(Coset)} $a-b\in \mathfrak{I}\,\,\forall\, a,b\in\mathfrak{I}$
\item \textbf{(Closure)} $r\cdot a\in \mathfrak{I}$ and
  $a\cdot r\in\mathfrak{I}$ $\forall$ $a\in\mathfrak{I}$ and $r\in R$
\end{itemize}

All rings have the ideals, $\{0\}$, and $R$. If $R$ has an identity 
then $R=\{1\}$. In general denote a set of elements, $P$,
that generate and ideal $\mathfrak{I}$ as $\mathfrak{I}=\{P\}$ 
In general a prime ideal $\mathfrak{P}$ is not a proper subset 
of any ideal other than $R$. A maximal ideal is the largest 
proper ideal subset of the ring $R$.

If the multiplicative operator of a ring, $R$, is commutative then $R$ is commutative.

The zero divisor subset, $ZD$, of a ring, $R$,
consists of elements with the following property:

\begin{itemize}
\item \textbf{\small (Zero)} $0\in ZD$,
\item \textbf{\small (Zero Divisor)} $ZD\subset R$
  such that $\forall$ $a\in ZD$ there exists $b\in ZD$ where
  $a\cdot b=0$ or $b\cdot a=0$
\end{itemize}

A ring with a trivial zero divisor set, $ZD=\{0\}$, is an integral domain.

A nontrivial ring, $(R,+,\cdot,0,1)$, also has a nontrivial subset,
$R^*\subset R\backslash\{0\}$, of units.

\begin{itemize}
\item \textbf{\small (Unit)} $1\in R^*$, 
\item \textbf{\small (Closure)} $\forall$ $u\in R^*$ $\exists$ $u^{-1}\in R^*$
  such that $u\cdot u^{-1}=u^{-1}\cdot u=1\in R^*$
\end{itemize}

A ring consisting wholely of units and the additive identity is a division ring,
$R=\{0\}\cup R^*$. A commutative division ring is a field.

A more complex compositional structure is a module.
A left $R-module$ is a compositional algebra
of an additive group, $(A,+,0)$ and a ring $(R,+,\cdot,0)$
with a compositional function, $f{:}R\times A -> A$,
denote $f(r,a)=ra\in A$ $\forall$ $r\in R$ and $a\in A$.
Composition of ring and function operations demonstrate 
distibutive properties:

\begin{itemize}
\item $r(a+b)=ra + rb\in A$ $\forall$ $r\in R$ and $a,b\in A$
\item $(r+s)A=ra+sa\in A$ $\forall$ $r,s\in R$ and $a\in A$
\item $(r\cdot s)a=r\cdot (sa)\in A$ $\forall$ $r,s\in R$ and $a\in A$
\end{itemize}

A unitary $R-module$ incorporates a ring with identity for the multiplicative unit
of the ring with identity $R$ of the module $A$.

\begin{itemize}
\item \textbf{\small (Unit)} $1a=a$ for $1\in R$ and $\forall$ $a\in A$
\end{itemize}

A left vector space is a left $R-module$ over a division ring.

A right $R-module$ transposes the operational properties of the ring $R$
relative to the left $R-module$, $M$.

A module may have different structure to discriminate 
operations from the left and right, e.g. covarant/contravariant matrix algebra.
If the ring of the module is commutative the left and right module
operations are identical. A module with a commutative base ring centralizer
is a bimodule.

Modules may be constructed, extended or derived from the dependent ring. For instance
a module may be a basis of ideals, multivectors, derivations, matrices or tensors.

\begin{center}
$\alpha\cdot\prod_{x\in X}x^{i_x},\,i_x\in\mathbb{N},\,\alpha\in R$
\end{center}

A free group over a variable set may defines a module over
a division ring or a field. An abelian free group over a filed
is a polynomial.

Polynomial modules are rings. Hence, if the base ring is a domain
then the polynomial ring is also a domain.

polynomial domains have irreducuble elements.

\begin{itemize}
\item \textbf{\small (Irreducible)} $p\in M$ is irreducuble
  if $p=q\cdot r\,\Rightarrow$ either $q$ or $r$ is a unit of $M$.
\end{itemize}

A further journey in compositional structure is an algebra over a
commutative ring, $K$, a $K-algebra$, $\mathbb{A}$.

\begin{itemize}
\item \textbf{\small (Unitary)} $(\mathbb{A},+)$ is a unitary left $K$ module
\item \textbf{\small (Closure)} $k\cdot(ab)=(k\cdot a)b=a(k\cdot b)$
  $\forall$ $k\in K$ and $a,b\in\mathbb{A}$
\end{itemize}

\bibliography{sample-paper}

\bibliographystyle{prsty}
\begin{thebibliography}{99}
\bibitem{cartan1966}Cartan,\'{E}., ``The Theory of Spinors'', Hermann, [1966]
\bibitem{cartan1913}Cartan,\'{E}., ``Les Groupes Projectifs qui ne laissent
  invariante aucune multiplicit\'{e} plane'', Bull. Soc. Math. France, [1913]
\bibitem{brauerweyl1937}Brauer,R. and Weyl,H., ``Spinors in n Dimensions'', Am. J. Math., 57, pp. 425-449, [1937]
\bibitem{pauli1927}Pauli,W., ``Zur Quantenmechanik des Magnetischen Elektrons'', Z. Phys., 43, pp. 601-623, [1927]
\bibitem{schafer2008}Schafer,R.D., ``An Introduction to Nonassociative Algebras'', Project Gutenberg(Public Domain USA). Internet, [2008]
\bibitem{dirac1928}Dirac,P.A.M., ``The Quantum Theory of the Electron'', Proc. R. Soc. \(London\), [1928]
\bibitem{birkoff1935}Birkoff,G.D., ``On the Structure of Abstract Algebras'', Proc. Cambridge P. Soc. 31, pp. 433-454, [1935]
\bibitem{jipsenrose1992}Jipsen,P.,Rose,H.``Varieties of Lattices'', Springer-Verlag,Berlin-Heidelberg, [1992]
\bibitem{hodges1993}Hodges,W., ``Model Theory'', Cambridge Univ. Press, Cambridge, [1993]
\bibitem{awodey2010}Awodey,S., ``Category Theory'', Oxford Univ. Press, Oxford, [2010]
\bibitem{lang2002}Lang,S., ``Algebra'', Springer-Verlag, NY, [2002]
\bibitem{bourbaki1989}Bourbaki,N., ``Algebra I'', Springer-Verlag, Berlin Heidelberg, [1989]
\bibitem{bourbaki22003}Bourbaki,N., ``Algebra II'', Springer-Verlag, Berlin Heidelberg, [2003]
\bibitem{hungerford1974}Hungerford,T.W., ``Algebra'', Springer-Verlag, NY, [1974]
\bibitem{burris1981}Burris,S.,Sankappanavar,H.P., ``A Course in Universal Algebra'', Springer-Verlag, NY, [1981]
\bibitem{rotman1998}Rotman,J., ``Galois Theory'', Springer-Verlag, NY, [1998]
\bibitem{fultonharris2004}Fulton,W., Harris,J., ``Representation Theory: A First Course'', Springer-Verlag, NY, [2004]
\bibitem{humphreys1972}Humphreys,J., ``Introduction to Lie Algebras and Representation Theory'', Springer-Verlag, NY, [1972]
\bibitem{brouwer1911}Brouwer,L.E.J., ``\"Uber Abbildung von Mannigfaltigkeiten'', Math. Ann., 71, pp. 97-115, [1911]
\bibitem{birkhoff1913}Birkhoff,G.D., ``Proof of Poincar\'e's Geometric Theorem'', Trans. Amer. Math. Soc., 14, pp. 14-22, [1913]
\bibitem{birkhoff1925}Birkhoff,G.D., ``An Extension of Poincar\'e's Last Geometric Theorem'', Acta Math., 47, pp. 297-311, [1925]
\bibitem{poincare1912}Poincar\'e,H., ``Sur un Theor\`eme de G\'eom\'etrie'', Rend. Circ. Mat. Palermo, 33, pp. 375-407, [1912]
\bibitem{brownnwumann977}Brown,H.,NeumannD., ``Proof of the Poincar\'e-Birkhoff Fixed-Point Theorem'', Michigan Math. J., 24, pp. 21-31, [1977]

\end{thebibliography}

\end{document}
