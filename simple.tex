%%%%%%%%%%%%%%%%%%%%%%%%%%%%%%%%%%%%%%%%%%%%%%%%%%%%%%%%%%%%%%%%%%%%%%%
% Sample template for MIT Junior Lab Student Written Summaries
% Available from http://web.mit.edu/8.13/samplepaper/sample-paper.tex
%
% Last Updated June 20, 2004
%
% Adapted from the American Physical Societies REVTeK-4 Pages
% at http://publish.aps.org
%
% ADVICE TO STUDENTS: Each time you write a paper, start with this
%    template and save under a new filename.  If convenient, don't
%    erase unneeded lines, just comment them out.  Often, they
%    will be useful containers for information.
%%%%%%%%%%%%%%%%%%%%%%%%%%%%%%%%%%%%%%%%%%%%%%%%%%%%%%%%%%%%%%%%%%%%%%%

%\documentclass[aps,twocolumn,secnumarabic,graphics,flotfix,graphicx,
%url,bm,tightenlines,nobibnotes,nobalancelastpage,amsmath,amssymb,
%nofootinbib]{article}
\documentclass[aps,twocolumn,secnumarabic,nobalancelastpage,amsmath,amssymb,
nofootinbib,parskip=full]{revtex4}

% standard graphics specifications
% alternative graphics specifications
% helps with long table options
% for on-line citations
% special 'bold-math' package

\usepackage{utf8}{inputenc}
%\usepackage[bottom]{footmisc}
\usepackage{graphics}      % standard graphics specifications
\usepackage{graphicx}      % alternative graphics specifications
\usepackage{longtable}     % helps with long table options
\usepackage{url}           % for on-line citations
\usepackage{bm}            % special 'bold-math' package

%\setlength{\parskip}{\baselineskip}
\setlength{\parskip}{4em}

\newcommand{\di}[1]{\boldsymbol #1}

\newcommand{\dI}{\boldsymbol \imath}

\begin{document}
\title{Emerging Insight into Spinor Concepts}
\author         {Wolfgang Kraske}
\email        {wolfkden@gmail.com}
\homepage     {http://www.oviumzone.com}
\affiliation  {OVium Studies in Mathematical Physics}
\date{\today}

\begin{abstract}
Spinors resonate deep mysteries in quantum physics while
galvanizing fields of mathematical research.
The manuscript herein reviews spinor concepts and
unifies mathematical principles for a comprehensive
application in information theory.
\end{abstract}

\maketitle

\section{Introduction}

The namesake of spinor harkens from the observable fine structure of
hydrogen energy spectra. The concept of the spinor defies representation
by the classical geometries of spatial coordinates and momenta.
Mathematically, particle spin is a manifestation of the discontinuities
in the rotation group $SO(3)$ of a sphere $S^2$. The spin group, $Sp(3)$,
corresponds directly with a binary, $\mathbb{Z}_2$, cover of the
rotational group, $SO(3)$.

\begin{equation}
Sp(3) / \mathbb{Z}_2 \cong SO(3)
\end{equation}

The seminal contribution of spin structure of electron energy states
was introduced by Pauli in 1927, \cite{pauli1927}, and Dirac in 1928,
\cite{dirac1928}. Although an epic discovery in physics,
the spin concept, sans the moniker, spinor, was originated
in 1913 by the encompassing mathematical studies by \'{E}lie Cartan,
\cite{cartan1913}. Mathematically, the invariance of the spinor
in complex multilinear spaces was discovered by \'{E}lie Cartan
as an extension of the manifold
representations familiar to tensor calculus and exterior algebra.
The particle spin ansance of 1920's physics and Cartan's
prior accomplishment were
comprehensively recognized and further formalized in 1937 within several
mathematically rigorous treatise of spin
extending to general 2n and 2n+1 dimensional manifolds. Specifically,
spinors extend the multilinear concept of tensors to include
double sheeted coverings of manifolds.
Henceforth, the lecture notes of \'{E}lie Cartan summarized and
elaborated the accomplishments of physicists, Pauli \cite{pauli1927}
and Dirac \cite{dirac1928} as well as the mathematicians,
Brauer and Weyl, \cite{brauerweyl1937}.
In 1966, the published lecture notes were posthumously
translated into English \cite{cartan1966}.

Peripatetically, this manuscript approaches spinor analysis
with various algebraic structures, most generally with modules and lie algebra.
Algebraic structures of module, field and lie algebra are variously used
in the following mathematical developments although generalizations are
introduced to adjust to specific interpretations in physics and
computational application.

Fundamental algebraic structures used are monoid, ring and module,
\cite{rotman1998},\cite{lang2002}.


\section{Composite Algebra}

Composite algebras provides an accessible approach to engage the reader
with Lie theory and spinor concepts. Division algebras
embraces the mathematical development of modular spaces
from the beginings of the 19\textsuperscript{th} and
even 18\textsuperscript{th} century. Basic properties of
a division algebra, $\mathbb{A}$, are the following:

\begin{itemize}
\item \textbf{\small (Initial Base Field)} Real Numbers $\mathbb{R}$
\item \textbf{\small (Unit)} the division algebras, $\mathbb{A}$ has
\item \textbf{\small (Fundamental Form)} there exists a fundamental form $\Phi$
  acting on elements $x\in\mathbb{A}$ such that:
  \begin{itemize}
  \item $\Phi(x)$ is a measure of the square distance of
    x from the origin of $\mathbb{A}$
  \item  provides a partial ordering on elements of $\mathbb{A}$
  \end{itemize}
\end{itemize}

Division algebra construction embraces an interative process begining
with the real field, $\mathbb{R}$ and subsequently extending
to more complex modular structures.

Exposition begins with the real number line, $\mathbb{R}$,
as a base field. The real line, $\mathbb{R}$, is a commutative field
as well as a simple manifold intuitively familiar to most readers.
As a division algebra, the fundamental form of $\mathbb{R}$ is the
square of any element, $\Phi(x)=x^2,\,\forall\, x\in\mathbb{R}$,
$\Phi:\mathbb{R}\rightarrow\mathbb{R}_+$. $\Phi$ is a polynomial
in $\mathbb{R}[x]$. $\Phi$ is a nonegative mapping easily
extended to $\Phi(x)+1\in\mathbb{R}[x]$, an irreducible polynomial
relative to $\mathbb{R}$.
As an irreducible, the polynomial $1+x^2$ is a generator of
the ideal $\{1+x^2\}$ over the domain $\mathbb{R}(x)$.
The extension, eq. \ref{eq:complexfield}, sequencially yields the next
division algebra:

\begin{equation} \label{eq:complexfield}
\mathbb{R}[x]/\{x^2+1\}
\end{equation}

The division algebra, eq. \ref{eq:complexfield}, reduces each polynomial
of $\mathbb{R}[x]$ to coset polynomial of order 1, equivalently
a vector in $\mathbb{R}^2$:

\begin{equation} \label{eq:complexreduce}
  p(x)\equiv a+bx \in \mathbb{R}[x]/\{x^2+1\}\, ,
\end{equation}
\begin{center}
where $p(x)\in\mathbb{R}[x]$ and $a,b\in\mathbb{R}$.
\end{center}

The ring, $\mathbb{R}[x]$, splits over the irreducible polynomial,
$x^2+1$, eq. \ref{eq:complexreduce}.

Further inspection of the polynomial $x^2+1$ proves to have an symbolic
solution historically designated as $\di{i}$, hence the moniker, imaginary
number. A little analysis further shows the division ring, eq.
\ref{eq:complexfield}, to be isomorphic to the complex number field, $\mathbb{C}$.

\begin{equation} \label{eq:complexclosure}
\mathbb{C}\cong\mathbb{R}(x)\equiv\mathbb{R}[x]/(x^2+1)
\end{equation}

Equation \ref{eq:complexclosure} is a well known result in algebra,
the closure of the polynomial domain over $/mathbb{R}$,
$\mathbb{R}(x)$, is the complex field, $\mathbb{C}$, or otherwise stated all
roots of the polynomial domain are contained within the complex field, $/mathbb{C}$.

Continuing the process with the next interation yields the quaternion non-abelian
field, $\mathbb{Q}$, then the octonian composition algebra, $\mathbb{O}$.

To develop the next level of division algebra polynomial domain, $\mathbb{C}[x]$ extended from the complex
numbers $\mathbb{C}$. To define a fundamental form for $\mathbb{C}$ over the extension field $\mathbb{C}[x]$
is necessary. Once again an extension element is introduced, $\di{j}$, as a solution to the polynomial $X^2+1$.
The two imaginary extension units, $\di{i}$ and $\di{j}$ anticommute to establish a reflection operation over
$\mathbb{C}[x]$. The reflection operator $\di{j}$ defines a partial order over $\mathbb{C}$ as
$(xj)^2=xjxj=xx^*jj=-xx^*=-|x|^2$, the reflection operation, $*$, is complex conjugation. For
$|x|^2=1$ with $x\in\mathbb{C}$ shows a new solution to $(xj)^2+1$. 

The quaternion field derives from the extension of $\mathbb{C}[x]$ with unit operators $\di{j}$. Unit operators
$\di{i}$ and $\di{j}$ combined to form a third unit operator $\di{k}.

\begin{align} 
  \di{k}=\di{i}\cdot\di{j}=-\di{j}\cdot\di{i} \label{eq:qunit1} \\
  \di{i}\cdot\di{k}=\di{i}\cdot\di{i}\cdot\di{j}=-\di{j} \label{eq:qunit2} \\
  \di{k}\cdot\di{j}=\di{i}\cdot\di{j}\cdot\di{j}=-\di{i} \label{eq:qunit3} \\
  \di{k}^2=\di{i}\cdot\di{j}\cdot\di{i}\cdot\di{j}=-\di{i}\cdot\di{i}\cdot\di{j}\cdot\di{j}=-1 \label{eq:qunit4} \\
  -1=\di{i}^2=\di{j}^2=\di{k}^2 \label{eq:qunit5}
\end{align}

The properties of the quaternion unit operators affords a quaternion modular form with real coefficients:

\begin{equation} \label{eq:quaternion}
  q\in\mathbb{Q}\,\,,q=r+x\di{i}+x\di{j}+x\di{k}\,\,,r,x,y,z\in\mathbb{R}
\end{equation}

The anti-commuting properties, eqs. \ref{eq:qunit1} \ref{eq:qunit2} \ref{eq:qunit3},
of the extension units prove the quaternion field to have anti-commutative properties as well.

Again a new conjugation operator for the quaternion field and a subsequently more complex division algebra is sought.

This manuscript expands the formalisms of multilinear algebra and tensors
derived from a fundamental form, $\Phi$, over a pseudo-Euclidean vector
space, $V$. Vector spaces are defined over a field designated, $K$,
generally assumed a closed complex field, $\mathbb{C}$, unless confined to
the real field, $\mathbb{R}$. A general vector space, $V$,
has a dimension, $n\in\mathbb{N}$,
consisting of a set of vectors defined over the field, $K$,

\begin{equation}
V = \left\{v\mid\ v=\left(x_1\dots x_n\right),\,x_i\in K\right\}.
\end{equation}

\begin{itemize}
\item \textbf{\small (Reflection Operator)} there exists a real subspace, $\Re$, and an imaginary subspace, $\Im$,
  of the division algebra, $\mathbb{A}$
\end{itemize}

The fundamental form provides a base measure to develop a bilinear form to
determine similarities between elements of $\mathbb{A}$
and reflections in subspaces of $\mathbb{A}$, 

\begin{itemize}
\item \textbf{\small (Bilinear Form)} $\langle x,y\rangle={1\over 2}(\Phi (x-y)-\Phi(x)-\Phi(y)})$ 
\end{itemize}



Essentially each vector is a list of, $n$, field elements,
indexed from the set $\left[n\right]=\left\{1,\dots,n\right\}$.
In general the vector space, $V$, is a an algebra over a ring.
Typically the ring is precipitated by modular structures
to form a finite dimensional basis. 

Exampli gratia, a ring is restricted to the well behaved field of
real numbers, $\mathbb{R}$. 
$\left\{e_i\mid e_ie_j=\delta_{ij}e_i\right\}$ where $\delta_{ij}$ is the Kronecker delta function.
The representation of this group is the direct sum over the field, $K$.

\begin{equation}
V = \left\{v\mid\ v=\left(x_1\dots x_n\right),\,x_i\in K\right\}.
\end{equation}


Original terminology introduced by early authors such as Cartan and Weyl are referenced.
The linear concepts of a Cartesian coordinate frame and Sylvester's law of inertia underly
the field theory of general manifolds. In particular, the theorems of
Cartan-Dieudonn\'{e} provides a natural definition of basis, reflection and rotations.

\section{Appendix}

The following algebraic developments are detailed in Serge Lang's text, 
\cite{lang2002}, and Joseph Rotman's text, \cite{rotman1998}. 
Development begins with sets and the action of a binary operator on a set.
Greater development will be dedicated to compositional algebras that
combine multiple structures and operators.

Consider a set $G$ with a binary function, a binary operator:
\begin{center}
$\cdot\, {:}\, G \times G \to G$:
\end{center}

\begin{itemize}
\item \textbf{\small (Closure)} $a\cdot b\in G$ $\forall$ $a,b\in G$
\end{itemize}

A simplifying notation for binary operator, $\cdot$ , is used:

\begin{center}
 $\cdot(a,b)\equiv a\cdot b$
\end{center}

A binary operator that is invariant to the order of the input argument pair
is commutative:

\begin{itemize}
\item \textbf{\small (Commutative)} $a\cdot b=b\cdot a\in G$, $\forall$ $a,b\in G$
\end{itemize}

Algebraic structures with commutative algebraic operations are abelian.

A semigroup is a set, $G$, with an associative binary operator, $\cdot$ :

\begin{itemize}
\item \textbf{\small (Associative)}
  $a\cdot (b\cdot c) = (a\cdot b)\cdot c\in G$ $\forall$ $a,b,c\in G$
\end{itemize}

A semigroup, $G$, with an identity element, $e\in G$, is a monoid $(G,\cdot,e)$:

\begin{itemize}
\item \textbf{\small (Identity)} $a\cdot e=e\cdot a=a$, $\forall$ $a \in G$
\end{itemize}

Uniqueness of the identity element derives as a corollary to the monoid definition.
The trivial monoid consists of a single identity element, $G=\{e\}$.

A monoid, $(G,\cdot,e)$, is a group if for every element, $a\in G$, 
there exists an inverse element $a^{-1}\in G$ such that:

\begin{itemize}
\item \textbf{\small (Inverse)} $\forall$ $a\in G$ $\exists$ $a^{-1}\in G$ \\
  such that $e=a\cdot a^{-1}=a^{-1}\cdot a$
\end{itemize}

As a corollary, the inverse of any element of a group is unique.

To support the development of a class of composite algebras a
free group provides a construct.
A free group is constructed from a set, $X$, and a unit, $e$.
A vocabulary $W$ of words $w$ of length $\ell(w)\in\mathbb{N}$ are constructed
by listing all combinations elements of $X$ such that $W\in X^{\ell(w)}$.
The identity element, $e$, is the zero length word.
The product of words is fulfiled by the non-abelian operation of 
appending of words.
Assume the product identity is $e$.
The vocabulary is the set $W={\cup}_{i\in\mathbb{N}}X^i$.
A collection of words that reduce to $e$ of length 0 determine a rule set
to reduce the length of words in W. 
This group structure is familiar to Coxeter groups and
useful in developing bases for algebras

A module is developed as a free group over a field. The center of the module
is the field.
If the generating set of the free group is abelian then the module is a polynomial ring. 

Semigroups, monoids and groups are algebraic structures 
with a single binary operator.
A compositional algebra is a structure with more than one 
binary operator. 
The simplest compositional algebra is a ring.
A ring, $(R,+,\cdot,0)$, is a composition of an 
additive commutative group $(R,+,0)$
and a multiplicative semigroup, $(R,\cdot)$.
A ring with identity is a ring with a 
multiplicative monoid $(R,\cdot,1)$.
All rings are assumed to have a multiplicative identity 
unless stated otherwise.
The composition of ring operators, $+$ and $\cdot$, 
demonstrate distributive ring, $R$, properties:

\begin{itemize}
\item \textbf{\small (Left Distributive)}
  $a\cdot(b + c)=a\cdot b+a\cdot c\in R,$
\item \textbf{\small (Right Distributive)}
  $(a+b)\cdot c=a\cdot c+b\cdot c\in R,$
\end{itemize}

\begin{center}
  $\forall$ $a,b,c\in R$.
\end{center}

The trivial ring consists of one element with one property:

\begin{itemize}
\item \textbf{(\small Trivial Ring)} $\mathfrank{Z}=\{0\}$ where $1=0$.
\end{itemize}

Unless otherwise stated all rings are assumed to be not trivial.

The distributive properties of a ring assert the following corollaries:

\begin{itemize}
\item \textbf{\small (Zero)} $0\cdot a=a\cdot 0=0$, $\forall$ $a\in R$
\item \textbf{\small (Unit Reflection)} $(-1)+1=1+(-1)=1-1=0$
\item \textbf{\small (Element Reflection)} $(-1)\cdot a=a\cdot (-1)=-a$, 
$(-a)+a=a+(-a)=a-a=0$ $\forall$ $a\in R$
\end{itemize}

A ring, $R$, has at least one ideal subset. All ideals, 
$\mathfrak{I}\subset R$, have properties:

\begin{itemize}
\item \textbf{(Coset)} $a-b\in \mathfrak{I}\,\,\forall\, a,b\in\mathfrak{I}$
\item \textbf{(Closure)} $r\cdot a\in \mathfrak{I}$ and
  $a\cdot r\in\mathfrak{I}$ $\forall$ $a\in\mathfrak{I}$ and $r\in R$
\end{itemize}

All rings have the ideals, $\{0\}$, and $R$. If $R$ has an identity 
then $R=\{1\}$. In general denote a set of elements, $P$,
that generate and ideal $\mathfrak{I}$ as $\mathfrak{I}=\{P\}$ 
In general a prime ideal $\mathfrak{P}$ is not a proper subset 
of any ideal other than $R$. A maximal ideal is the largest 
proper ideal subset of the ring $R$.

If the multiplicative operator of a ring, $R$, is commutative then $R$ is commutative.

The zero divisor subset, $ZD$, of a ring, $R$,
consists of elements with the following property:

\begin{itemize}
\item \textbf{\small (Zero)} $0\in ZD$,
\item \textbf{\small (Zero Divisor)} $ZD\subset R$
  such that $\forall$ $a\in ZD$ there exists $b\in ZD$ where
  $a\cdot b=0$ or $b\cdot a=0$
\end{itemize}

A ring with a trivial zero divisor set, $ZD=\{0\}$, is an integral domain.

A nontrivial ring, $(R,+,\cdot,0,1)$, also has a nontrivial subset,
$R^*\subset R\backslash\{0\}$, of units.

\begin{itemize}
\item \textbf{\small (Unit)} $1\in R^*$, 
\item \textbf{\small (Closure)} $\forall$ $u\in R^*$ $\exists$ $u^{-1}\in R^*$
  such that $u\cdot u^{-1}=u^{-1}\cdot u=1\in R^*$
\end{itemize}

A ring consisting wholely of units and the additive identity is a division ring,
$R=\{0\}\cup R^*$. A commutative division ring is a field.

A more complex compositional structure is a module.
A left $R-module$ is a compositional algebra
of an additive group, $(A,+,0)$ and a ring $(R,+,\cdot,0)$
with a compositional function, $f{:}R\times A -> A$,
denote $f(r,a)=ra\in A$ $\forall$ $r\in R$ and $a\in A$.
Composition of ring and function operations demonstrate 
distibutive properties:

\begin{itemize}
\item $r(a+b)=ra + rb\in A$ $\forall$ $r\in R$ and $a,b\in A$
\item $(r+s)A=ra+sa\in A$ $\forall$ $r,s\in R$ and $a\in A$
\item $(r\cdot s)a=r\cdot (sa)\in A$ $\forall$ $r,s\in R$ and $a\in A$
\end{itemize}

A unitary $R-module$ incorporates a ring with identity for the multiplicative unit
of the ring with identity $R$ of the module $A$.

\begin{itemize}
\item \textbf{\small (Unit)} $1a=a$ for $1\in R$ and $\forall$ $a\in A$
\end{itemize}

A left vector space is a left $R-module$ over a division ring.

A right $R-module$ transposes the operational properties of the ring $R$
relative to the left $R-module$, $M$.

A module may have different structure to discriminate 
operations from the left and right, e.g. covarant/contravariant matrix algebra.
If the ring of the module is commutative the left and right module
operations are identical. A module with a commutative base ring centralizer
is a bimodule.

Modules may be constructed, extended or derived from the dependent ring. For instance
a module may be a basis of ideals, multivectors, derivations, matrices or tensors.

\begin{center}
$\alpha\cdot\prod_{x\in X}x^{i_x},\,i_x\in\mathbb{N},\,\alpha\in R$
\end{center}

A free group over a variable set may defines a module over
a division ring or a field. An abelian free group over a filed
is a polynomial.

Polynomial modules are rings. Hence, if the base ring is a domain
then the polynomial ring is also a domain.

polynomial domains have irreducuble elements.

\begin{itemize}
\item \textbf{\small (Irreducible)} $p\in M$ is irreducuble
  if $p=q\cdot r\,\Rightarrow$ either $q$ or $r$ is a unit of $M$.
\end{itemize}

A further journey in compositional structure is an algebra over a
commutative ring, $K$, a $K-algebra$, $\mathbb{A}$.

\begin{itemize}
\item \textbf{\small (Unitary)} $(\mathbb{A},+)$ is a unitary left $K$ module
\item \textbf{\small (Closure)} $k\cdot(ab)=(k\cdot a)b=a(k\cdot b)$
  $\forall$ $k\in K$ and $a,b\in\mathbb{A}$
\end{itemize}



\bibliography{sample-paper}

\bibliographystyle{prsty}
\begin{thebibliography}{99}
\bibitem{cartan1966}Cartan,\'{E}., ``The Theory of Spinors'', Hermann, [1966]
\bibitem{cartan1913}Cartan,\'{E}., ``Les Groupes Projectifs qui ne laissent
  invariante aucune multiplicit\'{e} plane'', Bull. Soc. Math. France, [1913]
\bibitem{brauerweyl1937}Brauer,R. and Weyl,H., ``Spinors in n Dimensions'', Am. J. Math., 57, pp. 425-449, [1937]
\bibitem{pauli1927}Pauli,W., ``Zur Quantenmechanik des Magnetischen Elektrons'', Z. Phys., 43, pp. 601-623, [1927]
\bibitem{dirac1928}Dirac,P.A.M., ``The Quantum Theory of the Electron'', Proc. R. Soc. \(London\), [1928]
\bibitem{lang2002}Lang,S., ``Algebra'', Springer-Verlag, NY, [2002]
\bibitem{hungerford1974}Hungerford,T.W., ``Algebra'', Springer-Verlag, NY, [1974]
\bibitem{rotman1998}Rotman,J., ``Galois Theory'', Springer-Verlag, NY, [1998]
\bibitem{fultonharris2004}Fulton,W., Harris,J., ``Representation Theory: A First Course'', Springer-Verlag, NY, [2004]
\bibitem{humphreys1972}Humphreys,J., ``Introduction to Lie Algebras and Representation Theory'', Springer-Verlag, NY, [1972]
\bibitem{brouwer1911}Brouwer,L.E.J., ``\"Uber Abbildung von Mannigfaltigkeiten'', Math. Ann., 71, pp. 97-115, [1911]
\bibitem{birkhoff1913}Birkhoff,G.D., ``Proof of Poincar\'e's Geometric Theorem'', Trans. Amer. Math. Soc., 14, pp. 14-22, [1913]
\bibitem{birkhoff1925}Birkhoff,G.D., ``An Extension of Poincar\'e's Last Geometric Theorem'', Acta Math., 47, pp. 297-311, [1925]
\bibitem{poincare1912}Poincar\'e,H., ``Sur un Theor\`eme de G\'eom\'etrie'', Rend. Circ. Mat. Palermo, 33, pp. 375-407, [1912]
\bibitem{brownnwumann977}Brown,H.,NeumannD., ``Proof of the Poincar\'e-Birkhoff Fixed-Point Theorem'', Michigan Math. J., 24, pp. 21-31, [1977]

\end{thebibliography}

\end{document}
